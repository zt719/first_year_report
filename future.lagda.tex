\chapter{Future Work Plan}

\begin{code}[hide]
{-# OPTIONS --allow-unsolved-metas #-}
module future where

open import Relation.Binary.PropositionalEquality
open import outcomes

infix 2 Πtm-syntax
Πtm-syntax : (I : Type) → (I → Tm Γ A) → Tm Γ A
Πtm-syntax = Πtm
syntax Πtm-syntax A (λ x → B) = Πtm[ x ∈ A ] B

infix 2 Σtm-syntax
Σtm-syntax : (I : Type) → (I → Tm Γ A) → Tm Γ A
Σtm-syntax = Σtm
syntax Σtm-syntax A (λ x → B) = Σtm[ x ∈ A ] B
\end{code}

\subsubsection*{Full Formalization}

There are many things left to do in our current research. The primary object at this stage is to finish the constructions and proofs as we promised before. As a summary, we wish to:

\begin{itemize}
  \item{construct the functorial meaning of higher containers}
  \item{fully interpret higher containers to higher functors}
  \item{define the semantics of morphisms of higher containers}
  \item{finish the proofs of higher containers as a model of simply typed \lambda-calculus}
\end{itemize}

Moreover, current formalization is carried out in plain Agda. We wish to strictly define and replicate the whole formalization in Cubical Agda.

\subsubsection*{Soundness and Completeness}

Another objective is to prove the completeness and soundness properties of the normalizer $nf$. To do that, we first define an embedding from $Nf(\Gamma,A)$ to $Tm(\Gamma,A)$:

\begin{code}[hide]
mutual
\end{code}

\subsubsection*{Embedding}

\begin{code}
  embNf : Nf Γ A → Tm Γ A
  embNf (lam t) = lam (embNf t)
  embNf (ne spr) = embNe spr

  embNe : Ne Γ A → Tm Γ A
  embNe (S ◃ P ◃ R) = Σtm[ s ∈ S ] 
    Πtm[ A ∈ Ty ] Πtm[ x ∈ Var _ A ] Πtm[ p ∈ P x s ] 
    embSp (R x s p) (var x)

  embSp : Sp Γ A B → Tm Γ A → Tm Γ B
  embSp ε u = u
  embSp (t , ts) u = embSp ts (u $ embNf t)
\end{code}

\begin{code}[hide]
data _βη≡_ : Tm Γ A → Tm Γ A → Type where
\end{code}

Here, the completeness means the embedding of any normalized term is \beta\eta-equivalent to itself:

\begin{code}
completeness : (t : Tm Γ A) → embNf (nf t) βη≡ t
\end{code}

where the \beta\eta-equivalence should be a relation of \lambda-terms which reflects the convertibility under \beta-equivalence and \eta-equivalence. We need to characterize this relation as well.

The soundness says, if two terms are convertible, then they should reduce to the same canonical normal forms:

\begin{code}
soundness : (t u : Tm Γ A) → t βη≡ u → nf t ≡ nf u
\end{code}

\begin{code}[hide]
completeness = {!   !}
soundness = {!   !}
\end{code}

Combining two results together, we wish to show that the ``convertibility of terms is decidable''.