\chapter{Prerequisites}

\section{Type Theory}

\section{Category Theory}

We will use the semantics of type theory to introduce basic category theory concepts. Categories, functors and natural transformations:

\begin{code}[hide]
{-# OPTIONS --cubical --type-in-type #-}
module Prerequisites where
open import Cubical.Foundations.Prelude
open import Cubical.Data.Unit renaming (Unit to ⊤)
\end{code}

\begin{code}
record Cat (Obj : Type₁) : Type₂ where
  infixr 9 _∘_
  field
    Hom : Obj → Obj → Type₁
    id : ∀ {X} → Hom X X
    _∘_ : ∀ {X Y Z} → Hom Y Z → Hom X Y → Hom X Z
    idl : ∀ {X Y} (f : Hom X Y) → id ∘ f ≡ f
    idr : ∀ {X Y} (f : Hom X Y) → f ∘ id ≡ f
    ass : ∀ {W X Y Z} (f : Hom X W) (g : Hom Y X) (h : Hom Z Y)
          → (f ∘ g) ∘ h ≡ f ∘ (g ∘ h)

record Func {A B : Type₁} (ℂ : Cat A) (𝔻 : Cat B) (F : A → B) : Type₁ where
  open Cat
  field
    F₁ : ∀ {X Y} → Hom ℂ X Y → Hom 𝔻 (F X) (F Y)
    F-id : ∀ {X} → F₁ {X} (ℂ .id) ≡ 𝔻 .id
    F-∘ : ∀ {X Y Z} (f : Hom ℂ Y Z) (g : Hom ℂ X Y)
          → F₁ (ℂ ._∘_ f g ) ≡ 𝔻 ._∘_ (F₁ f) (F₁ g)

record Nat {A B : Type₁} (ℂ : Cat A) (𝔻 : Cat B)
  (F G : A → B) (FF : Func ℂ 𝔻 F) (GG : Func ℂ 𝔻 G) : Type₁ where
  open Cat
  open Func
  field
    η : ∀ X → Hom 𝔻 (F X) (G X)
    nat : ∀ {X Y} (f : Hom ℂ X Y)
      → 𝔻 ._∘_ (GG .F₁ f) (η X) ≡ 𝔻 ._∘_ (η Y) (FF .F₁ f)
\end{code}