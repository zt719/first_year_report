% Abstract

\addcontentsline{toc}{chapter}{Abstract}

\begin{abstract}

Giving a short overview of the work in your project.\cite{ABBOTT20053}

\end{abstract}

\begin{code}
open import Relation.Binary.PropositionalEquality  
\end{code}

\section{Natural Numbers}
First, we define the type of natural numbers inductively:

\begin{code}
data ℕ : Set where
  zero : ℕ
  suc  : ℕ → ℕ
\end{code}

Here, \AgdaDatatype{ℕ} is the type of natural numbers, with two constructors:
\begin{itemize}
  \item \AgdaInductiveConstructor{zero} represents 0.
  \item \AgdaInductiveConstructor{suc} represents the successor function (i.e., $n+1$).
\end{itemize}

\section{Addition}
Next, we define addition recursively:

\begin{code}
_+_ : ℕ → ℕ → ℕ
zero  + n = n
suc m + n = suc (m + n)
\end{code}

This definition states:
\begin{itemize}
  \item \AgdaSymbol{0 + n = n} (base case).
  \item \AgdaSymbol{(m + 1) + n = (m + n) + 1} (recursive case).
\end{itemize}

\section{A Simple Proof}
We now prove that \AgdaSymbol{2 + 2 = 4}. First, we define the numbers:

\begin{code}
two : ℕ
two = suc (suc zero)

four : ℕ
four = suc (suc (suc (suc zero)))
\end{code}

Now, the proof reduces by computation:

\begin{code}
proof : two + two ≡ four
proof = refl
\end{code}

Since Agda's definitional equality handles reduction, \AgdaFunction{refl} suffices.

\section{Conclusion}
This example shows how Agda and LaTeX can be combined for formal proofs in papers. The full output is rendered with syntax highlighting.
