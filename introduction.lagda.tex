\chapter{Introduction}

\section{Background and Motivation}

TODO

\section{Aims and Objectives}

TODO

\section{Progress to Date}

Over the first few months, I engaged in a broad exploration of type theory and category theory. I have read fundational textbooks including Category Theory for Lazy Functional Programmers, Tao of Types, the HoTT book. I developed Agda and Cubical Agda programming skills by formalizing and reinterpreting notes and existing papers. After the learning phase, I shifted my focus toward learning and developing my current research area - containers. I met regularly with my supervisor for weekly discussions to track progress, discuss current problems, and plan next steps.

Beside my own research areas, I also learned ongoing research questions and outcomes by actively participating the Type Theory Cafe and Functional Programming Lunch, which are both internal seminars series within FP Lab. I also gave a talk about my master research - \'algebraic effects in Haskell\' on one of the FP Lunch seminars.

I had the opportunity to attend large-scale academic events. I attended the Midland Graduate School 2025 in Sheffield, where I followed the courses on coalgebras, the Curry-Howard correspondence, refinement type in Haskell. As part of MGS25, I also assisted in teaching a course on category theory by answering questions from the exercise sessions and preparing Latex solutions for the lecture notes. Before MGS25, I participated MGS24 in Leicester and MGS Christmas 24 in Sheffield, where I benefited from many courses and talks. Additionally, I also attended the TYPES 2025 in Glasgow, a five-day international conference covering a wide range of topics in type theory.

In the spring term, I worked as a teaching assistant for several modules. I helped marking exercises and exams, as well as running weekly tutorials for the module - Languages and Computation. I also acted as a lab tutor for Programming Paradigms, where I supported students on Haskell exercises during weekly lab sessions.