\documentclass{report}

% Preamble
\usepackage{graphicx}
\usepackage{xcolor}
\usepackage[hidelinks]{hyperref}
\usepackage{agda}
\usepackage{fontspec}
\usepackage{tikz-cd}
\usepackage{tikz}
\usepackage{amsmath}
\usepackage{amsfonts}
\usepackage{stmaryrd}
\usepackage{fontawesome5}
\usepackage{cancel}

% Mono-Space Font
\setmonofont{JuliaMono-Regular.ttf}[
  Path  = ./fonts/,
  Scale = 0.85
  ]

\linespread{1.2}

% Graphics Path
\graphicspath{ {./graphics/} }

\begin{document}

\pagenumbering{gobble}

% Title Page
\begin{titlepage}
    \centering
    
    \includegraphics[width=0.7\textwidth]{nottingham-logo.png} \\[1.5cm]
    
    {\huge \textbf{Progression Review Report}} \\[0.5cm]
    {\large Formalizing Higher-Order Containers} \\[1.0cm]
    
    {\large \textbf{Zhili Tian}} \\[0.2cm]
    {\large Supervised by Thorsten Altenkirch} \\ [0.2cm]
    {\large \& Ulrik Buchholtz} \\[2.0cm]

    \includegraphics[width=0.7\textwidth]{fplab-title7.png} \\[0.5cm]

    {\large Functional Programming Lab} \\[0.2cm]    
    {\large School of Computer Science} \\[0.2cm]
    {\large University of Nottingham} \\[1.0cm]
    
    {\large \today}
    
    \vfill

\end{titlepage}

\pagenumbering{roman}
\addtocounter{page}{1}

% Abstract
\begin{abstract}

Containers are semantic and categorical way to talk about strictly positive types. It is shown in previous work that the category of containers is closed under compositions, products, coproducts and exponentials. More extensions and variants, such as indexed containers, generalized containers are studied to capture a boarder class. In this project, we aim to develop the syntax and semantics of the higher-order containers and do formalization in Agda. The idea of higher containers is to provide a semantics for higher strictly positive types, such as monad transformers with kinds $(Type \to Type) \to Type \to Type$. We define a family of categories of containers indexed by their kinds and explore the properties. We also show that the higher containers give rise to a model of simply typed \lambda-calculus.

\end{abstract}

\addtocounter{page}{1}

% Contents Table
\tableofcontents

\cleardoublepage
\pagenumbering{arabic}

% Main Body
\chapter{Introduction}

\begin{code}[hide]%
\>[0]\AgdaKeyword{variable}\AgdaSpace{}%
\AgdaGeneralizable{X}\AgdaSpace{}%
\AgdaSymbol{:}\AgdaSpace{}%
\AgdaPrimitive{Set}\<%
\\
%
\\[\AgdaEmptyExtraSkip]%
\>[0]\AgdaFunction{Type}\AgdaSpace{}%
\AgdaSymbol{:}\AgdaSpace{}%
\AgdaPrimitive{Set₁}\<%
\\
\>[0]\AgdaFunction{Type}\AgdaSpace{}%
\AgdaSymbol{=}\AgdaSpace{}%
\AgdaPrimitive{Set}\<%
\end{code}

\section{Background and Motivation}

Martin-Löf Type Theory (MLTT) provides a constructive foundation for mathematics and programming. The essence of type theory is to represent propositions as types and their proofs as programs. Homotopy Type Theory (HoTT), a recent developed variant, extends MLTT with ideas from homotopy theory, interpreting types as spaces and equalities as paths. This richer interpretation enables reasoning about higher-dimensional structures, univalence, and equivalences between types.

Within these settings, we explore and generalize the concepts of containers, which offer a uniform way to represent a range of ``well-behaved'' inductive types. For many data types like trees, list, etc, container provides an alternative way to visualize and reason about their definition and data manipulating. It abstracts a data type into two components: a set of all possible shapes describing the overall form, and within each shape a position telling where data can be stored.

For example, the definition of \AgdaDatatype{List} \AgdaBound{X} can be visualized as a type with countable many shapes \AgdaDatatype{S} and, for each concrete shape \AgdaInductiveConstructor{sᵢ} \AgdaSymbol{:} \AgdaDatatype{S}, there are \AgdaInductiveConstructor{i} many data \AgdaBound{x : X} stored.

\begin{code}%
\>[0]\AgdaKeyword{data}\AgdaSpace{}%
\AgdaDatatype{List}\AgdaSpace{}%
\AgdaSymbol{(}\AgdaBound{X}\AgdaSpace{}%
\AgdaSymbol{:}\AgdaSpace{}%
\AgdaFunction{Type}\AgdaSymbol{)}\AgdaSpace{}%
\AgdaSymbol{:}\AgdaSpace{}%
\AgdaFunction{Type}\AgdaSpace{}%
\AgdaKeyword{where}\<%
\\
\>[0][@{}l@{\AgdaIndent{0}}]%
\>[2]\AgdaInductiveConstructor{s₀}\AgdaSpace{}%
\AgdaSymbol{:}\AgdaSpace{}%
\AgdaDatatype{List}\AgdaSpace{}%
\AgdaBound{X}\<%
\\
%
\>[2]\AgdaInductiveConstructor{s₁}\AgdaSpace{}%
\AgdaSymbol{:}\AgdaSpace{}%
\AgdaBound{X}\AgdaSpace{}%
\AgdaSymbol{→}\AgdaSpace{}%
\AgdaDatatype{List}\AgdaSpace{}%
\AgdaBound{X}\<%
\\
%
\>[2]\AgdaInductiveConstructor{s₂}\AgdaSpace{}%
\AgdaSymbol{:}\AgdaSpace{}%
\AgdaBound{X}\AgdaSpace{}%
\AgdaSymbol{→}\AgdaSpace{}%
\AgdaBound{X}\AgdaSpace{}%
\AgdaSymbol{→}\AgdaSpace{}%
\AgdaDatatype{List}\AgdaSpace{}%
\AgdaBound{X}\<%
\\
%
\>[2]\AgdaInductiveConstructor{s₃}\AgdaSpace{}%
\AgdaSymbol{:}\AgdaSpace{}%
\AgdaBound{X}\AgdaSpace{}%
\AgdaSymbol{→}\AgdaSpace{}%
\AgdaBound{X}\AgdaSpace{}%
\AgdaSymbol{→}\AgdaSpace{}%
\AgdaBound{X}\AgdaSpace{}%
\AgdaSymbol{→}\AgdaSpace{}%
\AgdaDatatype{List}\AgdaSpace{}%
\AgdaBound{X}\<%
\\
%
\>[2]\AgdaComment{\{-\ ...\ -\}}\<%
\end{code}

Data transfer can be understood and processed in a similar way. That is to to case analysis on shapes and, for each shape, specifying the target shape and how the target positions are obtained from the source positions. For example the \AgdaFunction{tail} function:

\begin{code}%
\>[0]\AgdaFunction{tail}\AgdaSpace{}%
\AgdaSymbol{:}\AgdaSpace{}%
\AgdaDatatype{List}\AgdaSpace{}%
\AgdaGeneralizable{X}\AgdaSpace{}%
\AgdaSymbol{→}\AgdaSpace{}%
\AgdaDatatype{List}\AgdaSpace{}%
\AgdaGeneralizable{X}\<%
\\
\>[0]\AgdaFunction{tail}\AgdaSpace{}%
\AgdaInductiveConstructor{s₀}\AgdaSpace{}%
\AgdaSymbol{=}\AgdaSpace{}%
\AgdaInductiveConstructor{s₀}\<%
\\
\>[0]\AgdaFunction{tail}\AgdaSpace{}%
\AgdaSymbol{(}\AgdaInductiveConstructor{s₁}\AgdaSpace{}%
\AgdaBound{x}\AgdaSymbol{)}\AgdaSpace{}%
\AgdaSymbol{=}\AgdaSpace{}%
\AgdaInductiveConstructor{s₀}\<%
\\
\>[0]\AgdaFunction{tail}\AgdaSpace{}%
\AgdaSymbol{(}\AgdaInductiveConstructor{s₂}\AgdaSpace{}%
\AgdaBound{x}\AgdaSpace{}%
\AgdaBound{y}\AgdaSymbol{)}\AgdaSpace{}%
\AgdaSymbol{=}\AgdaSpace{}%
\AgdaInductiveConstructor{s₁}\AgdaSpace{}%
\AgdaBound{y}\<%
\\
\>[0]\AgdaFunction{tail}\AgdaSpace{}%
\AgdaSymbol{(}\AgdaInductiveConstructor{s₃}\AgdaSpace{}%
\AgdaBound{x}\AgdaSpace{}%
\AgdaBound{y}\AgdaSpace{}%
\AgdaBound{z}\AgdaSymbol{)}\AgdaSpace{}%
\AgdaSymbol{=}\AgdaSpace{}%
\AgdaInductiveConstructor{s₂}\AgdaSpace{}%
\AgdaBound{y}\AgdaSpace{}%
\AgdaBound{z}\<%
\\
\>[0]\AgdaComment{\{-\ ...\ -\}}\<%
\end{code}

\section{Aims and Objectives}

Containers provide a compositional way to define and manipulate data types, making their semantics explicit and supporting generic programming, categorical analysis, and proofs about programs. In the Conducted Research section, we will present a detailed account of containers, outlining both their capabilities and their limitations. Our work generalize ordinary containers to higher-order containers, capturing and reinterpreting a broader range of concepts and structures in type theory. More specifically, we wish to:

\begin{itemize}
  \item{analyze higher(-order) data types with categorical methods}
  \item{define the meaning of higher functoriality}
  \item{define the syntax category of higher containers}
  \item{explore the categorical and algebraic properties of higher containers}
  \item{interpret the syntax to capture higher functoriality}
  \item{show that higher containers give rise to a simply typed category with families}
  \item{...}
\end{itemize}

\section{Progress to Date}

Over the first few months, I engaged in a broad exploration of type theory and category theory. I have read fundational textbooks including \textit{Categories for the Lazy Functional Programmer}, \textit{The Tao of Types}, \textit{Homotopy Type Theory - Univalent Foundations of Mathematics}. I deepened related background knowledge and developed Agda programming skills by formalizing and reinterpreting many existing papers. After the training phase, I shifted my focus toward learning and developing my current research area - containers. I met regularly with my supervisor for weekly discussions to track progress, discuss current problems, and plan next steps.

Beside my own research areas, I also learned ongoing research questions and outcomes by actively participating the \textit{Type Theory Cafe} and \textit{Functional Programming Lunch}, which are both internal seminars series within FP Lab. I also gave a talk about my master research - ``Algebraic Effects in Haskell'' on one of the FP Lunch seminars.

I had the opportunity to attend large-scale academic events. I attended \textit{Midland Graduate School 2025} in Sheffield, where I followed the courses on coalgebras, the Curry-Howard correspondence, refinement type in Haskell. As part of MGS25, I also assisted in teaching a course on category theory by answering questions from the exercise sessions and preparing Latex solutions for the lecture notes. Before MGS25, I participated MGS24 in Leicester and MGS Christmas 24 in Sheffield, where I benefited from many courses and talks. Additionally, I also attended the \textit{TYPES 2025} in Glasgow, a five-day international conference covering a wide range of topics in type theory.

In the spring term, I worked as a teaching assistant for several modules. I helped marking exercises and exams, as well as running weekly tutorials for the module \textit{Languages and Computation}. I also acted as a lab tutor for \textit{Programming Paradigms}, where I supported students on Haskell exercises during weekly lab sessions.
\chapter{Conducted Research}

\begin{code}[hide]%
\>[0]\AgdaSymbol{\{-\#}\AgdaSpace{}%
\AgdaKeyword{OPTIONS}\AgdaSpace{}%
\AgdaPragma{--guardedness}\AgdaSpace{}%
\AgdaSymbol{\#-\}}\<%
\\
%
\\[\AgdaEmptyExtraSkip]%
\>[0]\AgdaKeyword{open}\AgdaSpace{}%
\AgdaKeyword{import}\AgdaSpace{}%
\AgdaModule{Data.Empty}\<%
\\
\>[0]\AgdaKeyword{open}\AgdaSpace{}%
\AgdaKeyword{import}\AgdaSpace{}%
\AgdaModule{Data.Unit}\<%
\\
\>[0]\AgdaKeyword{open}\AgdaSpace{}%
\AgdaKeyword{import}\AgdaSpace{}%
\AgdaModule{Data.Sum}\<%
\\
\>[0]\AgdaKeyword{open}\AgdaSpace{}%
\AgdaKeyword{import}\AgdaSpace{}%
\AgdaModule{Data.Product}\<%
\\
\>[0]\AgdaKeyword{open}\AgdaSpace{}%
\AgdaKeyword{import}\AgdaSpace{}%
\AgdaModule{Function.Base}\<%
\\
\>[0]\AgdaKeyword{open}\AgdaSpace{}%
\AgdaKeyword{import}\AgdaSpace{}%
\AgdaModule{Relation.Binary.PropositionalEquality}\<%
\\
%
\\[\AgdaEmptyExtraSkip]%
\>[0]\AgdaFunction{Type}\AgdaSpace{}%
\AgdaSymbol{:}\AgdaSpace{}%
\AgdaPrimitive{Set₁}\<%
\\
\>[0]\AgdaFunction{Type}\AgdaSpace{}%
\AgdaSymbol{=}\AgdaSpace{}%
\AgdaPrimitive{Set}\<%
\\
%
\\[\AgdaEmptyExtraSkip]%
\>[0]\AgdaFunction{Type₁}\AgdaSpace{}%
\AgdaSymbol{:}\AgdaSpace{}%
\AgdaPrimitive{Set₂}\<%
\\
\>[0]\AgdaFunction{Type₁}\AgdaSpace{}%
\AgdaSymbol{=}\AgdaSpace{}%
\AgdaPrimitive{Set₁}\<%
\\
%
\\[\AgdaEmptyExtraSkip]%
\>[0]\AgdaKeyword{variable}\AgdaSpace{}%
\AgdaGeneralizable{X}\AgdaSpace{}%
\AgdaGeneralizable{Y}\AgdaSpace{}%
\AgdaGeneralizable{I}\AgdaSpace{}%
\AgdaSymbol{:}\AgdaSpace{}%
\AgdaFunction{Type}\<%
\end{code}

In this section, we begin by reviewing related literature. We then introduce a categorical view of types, containers, W-types and M-types. We also introduce category with families - a model of type theory, and hereditary substitutions - a normalization technic for \lambda-calculus. Finally, we outline the current challenges.

\section{Literature Review}
TODO

\section{Types as Algebras}

\subsection{Inductive Types are Initial Algebras}

We use natural number as our example through this sections. Natural number \texttt{ℕ = \{0, 1, 2, 3, ...\}} can be defined as inductive type:
\begin{code}%
\>[0]\AgdaKeyword{data}\AgdaSpace{}%
\AgdaDatatype{ℕ}\AgdaSpace{}%
\AgdaSymbol{:}\AgdaSpace{}%
\AgdaFunction{Type}\AgdaSpace{}%
\AgdaKeyword{where}\<%
\\
\>[0][@{}l@{\AgdaIndent{0}}]%
\>[2]\AgdaInductiveConstructor{zero}\AgdaSpace{}%
\AgdaSymbol{:}\AgdaSpace{}%
\AgdaDatatype{ℕ}\<%
\\
%
\>[2]\AgdaInductiveConstructor{suc}\AgdaSpace{}%
\AgdaSymbol{:}\AgdaSpace{}%
\AgdaDatatype{ℕ}\AgdaSpace{}%
\AgdaSymbol{→}\AgdaSpace{}%
\AgdaDatatype{ℕ}\<%
\end{code}

Given an endofunctor \AgdaBound{F} \AgdaSymbol{:} \AgdaPrimitive{Type} \AgdaFunction{⇒} \AgdaPrimitive{Type}, an algebra is defined as a carrier type \AgdaBound{A} \AgdaSymbol{:} \AgdaFunction{Type} and an evaluation function \AgdaBound{α} \AgdaSymbol{:} \AgdaBound{F} \AgdaBound{A} \AgdaSymbol{→} \AgdaBound{A}. The morphisms between algebras \AgdaSymbol{(}\AgdaBound{A} \AgdaInductiveConstructor{,} \AgdaBound{α}\AgdaSymbol{)} and \AgdaSymbol{(}\AgdaBound{B} \AgdaInductiveConstructor{,} \AgdaBound{β}\AgdaSymbol{)} are given by a function \AgdaBound{f} \AgdaSymbol{:} \AgdaBound{A} \AgdaSymbol{→} \AgdaBound{B} such that the following diagram commutes:

\[
\begin{tikzcd}[row sep=huge, column sep=huge]
  \AgdaBound{F A} \arrow[r, "\AgdaBound{F f}"] \arrow[d, "\AgdaBound{α}"]
  & \AgdaBound{F B} \arrow[d, "\AgdaBound{β}"] \\
  \AgdaBound{A} \arrow[r, "\AgdaBound{f}"]
  & \AgdaBound{B}
\end{tikzcd}
\]

Natural number (with its constructors) is the initial algebra of the \AgdaDatatype{⊤⊎\_} (\textbf{Maybe}) functor. To prove this, we first its constructors into an equivalent function form:
\begin{code}%
\>[0]\AgdaFunction{[z,s]}\AgdaSpace{}%
\AgdaSymbol{:}\AgdaSpace{}%
\AgdaRecord{⊤}\AgdaSpace{}%
\AgdaOperator{\AgdaDatatype{⊎}}\AgdaSpace{}%
\AgdaDatatype{ℕ}\AgdaSpace{}%
\AgdaSymbol{→}\AgdaSpace{}%
\AgdaDatatype{ℕ}\<%
\\
\>[0]\AgdaFunction{[z,s]}\AgdaSpace{}%
\AgdaSymbol{(}\AgdaInductiveConstructor{inj₁}\AgdaSpace{}%
\AgdaInductiveConstructor{tt}\AgdaSymbol{)}\AgdaSpace{}%
\AgdaSymbol{=}\AgdaSpace{}%
\AgdaInductiveConstructor{zero}\<%
\\
\>[0]\AgdaFunction{[z,s]}\AgdaSpace{}%
\AgdaSymbol{(}\AgdaInductiveConstructor{inj₂}\AgdaSpace{}%
\AgdaBound{n}\AgdaSymbol{)}\AgdaSpace{}%
\AgdaSymbol{=}\AgdaSpace{}%
\AgdaInductiveConstructor{suc}\AgdaSpace{}%
\AgdaBound{n}\<%
\end{code}

For the initiality, we show for any algebra there exists a unique morphism form algebra \texttt{ℕ}. That is to first undefine \AgdaFunction{fold}:

\begin{code}%
\>[0]\AgdaFunction{fold}\AgdaSpace{}%
\AgdaSymbol{:}\AgdaSpace{}%
\AgdaSymbol{(}\AgdaRecord{⊤}\AgdaSpace{}%
\AgdaOperator{\AgdaDatatype{⊎}}\AgdaSpace{}%
\AgdaGeneralizable{X}\AgdaSpace{}%
\AgdaSymbol{→}\AgdaSpace{}%
\AgdaGeneralizable{X}\AgdaSymbol{)}\AgdaSpace{}%
\AgdaSymbol{→}\AgdaSpace{}%
\AgdaDatatype{ℕ}\AgdaSpace{}%
\AgdaSymbol{→}\AgdaSpace{}%
\AgdaGeneralizable{X}\<%
\\
\>[0]\AgdaFunction{fold}\AgdaSpace{}%
\AgdaBound{α}\AgdaSpace{}%
\AgdaInductiveConstructor{zero}\AgdaSpace{}%
\AgdaSymbol{=}\AgdaSpace{}%
\AgdaBound{α}\AgdaSpace{}%
\AgdaSymbol{(}\AgdaInductiveConstructor{inj₁}\AgdaSpace{}%
\AgdaInductiveConstructor{tt}\AgdaSymbol{)}\<%
\\
\>[0]\AgdaFunction{fold}\AgdaSpace{}%
\AgdaBound{α}\AgdaSpace{}%
\AgdaSymbol{(}\AgdaInductiveConstructor{suc}\AgdaSpace{}%
\AgdaBound{n}\AgdaSymbol{)}\AgdaSpace{}%
\AgdaSymbol{=}\AgdaSpace{}%
\AgdaBound{α}\AgdaSpace{}%
\AgdaSymbol{(}\AgdaInductiveConstructor{inj₂}\AgdaSpace{}%
\AgdaSymbol{(}\AgdaFunction{fold}\AgdaSpace{}%
\AgdaBound{α}\AgdaSpace{}%
\AgdaBound{n}\AgdaSymbol{))}\<%
\end{code}

then construct the mapping morphism part of \AgdaDatatype{⊤⊎\_}:

\begin{code}[hide]%
\>[0]\AgdaFunction{⊤⊎₁}\AgdaSpace{}%
\AgdaSymbol{:}\AgdaSpace{}%
\AgdaSymbol{(}\AgdaGeneralizable{X}\AgdaSpace{}%
\AgdaSymbol{→}\AgdaSpace{}%
\AgdaGeneralizable{Y}\AgdaSymbol{)}\AgdaSpace{}%
\AgdaSymbol{→}\AgdaSpace{}%
\AgdaRecord{⊤}\AgdaSpace{}%
\AgdaOperator{\AgdaDatatype{⊎}}\AgdaSpace{}%
\AgdaGeneralizable{X}\AgdaSpace{}%
\AgdaSymbol{→}\AgdaSpace{}%
\AgdaRecord{⊤}\AgdaSpace{}%
\AgdaOperator{\AgdaDatatype{⊎}}\AgdaSpace{}%
\AgdaGeneralizable{Y}\<%
\\
\>[0]\AgdaFunction{⊤⊎₁}\AgdaSpace{}%
\AgdaBound{f}\AgdaSpace{}%
\AgdaSymbol{(}\AgdaInductiveConstructor{inj₁}\AgdaSpace{}%
\AgdaInductiveConstructor{tt}\AgdaSymbol{)}\AgdaSpace{}%
\AgdaSymbol{=}\AgdaSpace{}%
\AgdaInductiveConstructor{inj₁}\AgdaSpace{}%
\AgdaInductiveConstructor{tt}\<%
\\
\>[0]\AgdaFunction{⊤⊎₁}\AgdaSpace{}%
\AgdaBound{f}\AgdaSpace{}%
\AgdaSymbol{(}\AgdaInductiveConstructor{inj₂}\AgdaSpace{}%
\AgdaBound{x}\AgdaSymbol{)}\AgdaSpace{}%
\AgdaSymbol{=}\AgdaSpace{}%
\AgdaInductiveConstructor{inj₂}\AgdaSpace{}%
\AgdaSymbol{(}\AgdaBound{f}\AgdaSpace{}%
\AgdaBound{x}\AgdaSymbol{)}\<%
\end{code}

such that the following diagram commutes:

\[
\begin{tikzcd}[row sep=huge, column sep=5.0cm]
\AgdaDatatype{⊤ ⊎ ℕ} \arrow[r, "\AgdaFunction{⊤⊎₁ } \AgdaSymbol{(}\AgdaFunction{fold }\AgdaBound{α}\AgdaSymbol{)}"] \arrow[d, "\AgdaFunction{[z,s]}"]
& \AgdaDatatype{⊤ ⊎ }\AgdaBound{X} \arrow[d, "\AgdaBound{α}"] \\
\AgdaDatatype{ℕ} \arrow[r, "\AgdaFunction{fold }\AgdaBound{α}"]
& \AgdaBound{X}
\end{tikzcd}
\]

\subsection{Coinductive Types are Terminal Coalgebras}

One of the greatest power of category theory is that the opposite version of a theorem can always be derived for free. Now we inverse all the morphisms in above diagram and therefore dualize all concepts. We define conatural number as coinductive type and reverse \AgdaFunction{fold} to \AgdaFunction{unfold}:

\begin{code}%
\>[0]\AgdaKeyword{record}\AgdaSpace{}%
\AgdaRecord{ℕ∞}\AgdaSpace{}%
\AgdaSymbol{:}\AgdaSpace{}%
\AgdaFunction{Type}\AgdaSpace{}%
\AgdaKeyword{where}\<%
\\
\>[0][@{}l@{\AgdaIndent{0}}]%
\>[2]\AgdaKeyword{coinductive}\<%
\\
%
\>[2]\AgdaKeyword{field}\<%
\\
\>[2][@{}l@{\AgdaIndent{0}}]%
\>[4]\AgdaField{pred∞}\AgdaSpace{}%
\AgdaSymbol{:}\AgdaSpace{}%
\AgdaRecord{⊤}\AgdaSpace{}%
\AgdaOperator{\AgdaDatatype{⊎}}\AgdaSpace{}%
\AgdaRecord{ℕ∞}\<%
\\
\>[0]\AgdaKeyword{open}\AgdaSpace{}%
\AgdaModule{ℕ∞}\<%
\\
%
\\[\AgdaEmptyExtraSkip]%
\>[0]\AgdaFunction{unfold}\AgdaSpace{}%
\AgdaSymbol{:}\AgdaSpace{}%
\AgdaSymbol{(}\AgdaGeneralizable{X}\AgdaSpace{}%
\AgdaSymbol{→}\AgdaSpace{}%
\AgdaRecord{⊤}\AgdaSpace{}%
\AgdaOperator{\AgdaDatatype{⊎}}\AgdaSpace{}%
\AgdaGeneralizable{X}\AgdaSymbol{)}\AgdaSpace{}%
\AgdaSymbol{→}\AgdaSpace{}%
\AgdaGeneralizable{X}\AgdaSpace{}%
\AgdaSymbol{→}\AgdaSpace{}%
\AgdaRecord{ℕ∞}\<%
\\
\>[0]\AgdaField{pred∞}\AgdaSpace{}%
\AgdaSymbol{(}\AgdaFunction{unfold}\AgdaSpace{}%
\AgdaBound{α⁻}\AgdaSpace{}%
\AgdaBound{x}\AgdaSymbol{)}\AgdaSpace{}%
\AgdaKeyword{with}\AgdaSpace{}%
\AgdaBound{α⁻}\AgdaSpace{}%
\AgdaBound{x}\<%
\\
\>[0]\AgdaSymbol{...}\AgdaSpace{}%
\AgdaSymbol{|}\AgdaSpace{}%
\AgdaInductiveConstructor{inj₁}\AgdaSpace{}%
\AgdaInductiveConstructor{tt}\AgdaSpace{}%
\AgdaSymbol{=}\AgdaSpace{}%
\AgdaInductiveConstructor{inj₁}\AgdaSpace{}%
\AgdaInductiveConstructor{tt}\<%
\\
\>[0]\AgdaSymbol{...}\AgdaSpace{}%
\AgdaSymbol{|}\AgdaSpace{}%
\AgdaInductiveConstructor{inj₂}\AgdaSpace{}%
\AgdaBound{x'}\AgdaSpace{}%
\AgdaSymbol{=}\AgdaSpace{}%
\AgdaInductiveConstructor{inj₂}\AgdaSpace{}%
\AgdaSymbol{(}\AgdaFunction{unfold}\AgdaSpace{}%
\AgdaBound{α⁻}\AgdaSpace{}%
\AgdaBound{x'}\AgdaSymbol{)}\<%
\end{code}

such that the following diagram commutes:

\[
\begin{tikzcd}[row sep=huge, column sep=5.0cm]
\AgdaBound{X} \arrow[r, "\AgdaFunction{unfold }\AgdaBound{α⁻}"] \arrow[d, "\AgdaBound{α⁻}"]
& \AgdaDatatype{ℕ∞} \arrow[d, "\AgdaField{pred∞}"] \\
\AgdaDatatype{⊤ ⊎ }\AgdaBound{X} \arrow[r, "\AgdaFunction{⊤⊎₁ }\AgdaSymbol{(}\AgdaFunction{unfold }\AgdaBound{α⁻}\AgdaSymbol{)}"]
& \AgdaDatatype{⊤ ⊎ ℕ∞}
\end{tikzcd}
\]

Therefore we get the result of coinductive types are terminal coalgebras for free.

\section{Containers}

\subsection{Strict Positivity}

Containers are categorical and type-theoretical abstraction to describe strictly positive datatypes. A strictly positive type is the type where all data constructors do not include itself on the left-side of a function arrow. One typical counterexample is \AgdaDatatype{Weird}:

\begin{code}%
\>[0]\AgdaSymbol{\{-\#}\AgdaSpace{}%
\AgdaKeyword{NO\AgdaUnderscore{}POSITIVITY\AgdaUnderscore{}CHECK}\AgdaSpace{}%
\AgdaSymbol{\#-\}}\<%
\\
\>[0]\AgdaKeyword{data}\AgdaSpace{}%
\AgdaDatatype{Weird}\AgdaSpace{}%
\AgdaSymbol{:}\AgdaSpace{}%
\AgdaFunction{Type}\AgdaSpace{}%
\AgdaKeyword{where}\<%
\\
\>[0][@{}l@{\AgdaIndent{0}}]%
\>[2]\AgdaInductiveConstructor{foo}\AgdaSpace{}%
\AgdaSymbol{:}\AgdaSpace{}%
\AgdaSymbol{(}\AgdaDatatype{Weird}\AgdaSpace{}%
\AgdaSymbol{→}\AgdaSpace{}%
\AgdaFunction{⊥}\AgdaSymbol{)}\AgdaSpace{}%
\AgdaSymbol{→}\AgdaSpace{}%
\AgdaDatatype{Weird}\<%
\end{code}

\AgdaDatatype{Weird} is not strictly positive, as \AgdaDatatype{Weird} itself appears on the left-side of \AgdaSymbol{→} in \AgdaInductiveConstructor{foo}. Losing strict positivity can cause issues like non-normalizable, non-terminating, inconsistency, etc. In this case, a empty type term can be constructed using \AgdaFunction{¬weird} which is inconsistent to empty definition:

\begin{code}%
\>[0]\AgdaFunction{¬weird}\AgdaSpace{}%
\AgdaSymbol{:}\AgdaSpace{}%
\AgdaDatatype{Weird}\AgdaSpace{}%
\AgdaSymbol{→}\AgdaSpace{}%
\AgdaFunction{⊥}\<%
\\
\>[0]\AgdaFunction{¬weird}\AgdaSpace{}%
\AgdaSymbol{(}\AgdaInductiveConstructor{foo}\AgdaSpace{}%
\AgdaBound{x}\AgdaSymbol{)}\AgdaSpace{}%
\AgdaSymbol{=}\AgdaSpace{}%
\AgdaBound{x}\AgdaSpace{}%
\AgdaSymbol{(}\AgdaInductiveConstructor{foo}\AgdaSpace{}%
\AgdaBound{x}\AgdaSymbol{)}\<%
\\
%
\\[\AgdaEmptyExtraSkip]%
\>[0]\AgdaFunction{bad}\AgdaSpace{}%
\AgdaSymbol{:}\AgdaSpace{}%
\AgdaFunction{⊥}\<%
\\
\>[0]\AgdaFunction{bad}\AgdaSpace{}%
\AgdaSymbol{=}\AgdaSpace{}%
\AgdaFunction{¬weird}\AgdaSpace{}%
\AgdaSymbol{(}\AgdaInductiveConstructor{foo}\AgdaSpace{}%
\AgdaFunction{¬weird}\AgdaSymbol{)}\<%
\end{code}

\subsection{Syntax and Semantics}

A container is given by a shape \AgdaBound{S} and with each shape a position \AgdaBound{P}:

\begin{code}%
\>[0]\AgdaKeyword{record}\AgdaSpace{}%
\AgdaRecord{Cont}\AgdaSpace{}%
\AgdaSymbol{:}\AgdaSpace{}%
\AgdaFunction{Type₁}\AgdaSpace{}%
\AgdaKeyword{where}\<%
\\
\>[0][@{}l@{\AgdaIndent{0}}]%
\>[2]\AgdaKeyword{constructor}\AgdaSpace{}%
\AgdaOperator{\AgdaInductiveConstructor{\AgdaUnderscore{}◃\AgdaUnderscore{}}}\<%
\\
%
\>[2]\AgdaKeyword{field}\<%
\\
\>[2][@{}l@{\AgdaIndent{0}}]%
\>[4]\AgdaField{S}\AgdaSpace{}%
\AgdaSymbol{:}\AgdaSpace{}%
\AgdaFunction{Type}\<%
\\
%
\>[4]\AgdaField{P}\AgdaSpace{}%
\AgdaSymbol{:}\AgdaSpace{}%
\AgdaField{S}\AgdaSpace{}%
\AgdaSymbol{→}\AgdaSpace{}%
\AgdaFunction{Type}\<%
\end{code}

\begin{code}[hide]%
\>[0]\AgdaKeyword{variable}\AgdaSpace{}%
\AgdaGeneralizable{SP}\AgdaSpace{}%
\AgdaGeneralizable{TQ}\AgdaSpace{}%
\AgdaSymbol{:}\AgdaSpace{}%
\AgdaRecord{Cont}\<%
\end{code}

A container should give rise to a endofunctor \AgdaPrimitive{Type} \AgdaSymbol{⇒} \AgdaPrimitive{Type}. Again, we explicitly distinguish mapping objects part and mapping morphisms part of a functor:

\begin{code}%
\>[0]\AgdaKeyword{record}\AgdaSpace{}%
\AgdaOperator{\AgdaRecord{⟦\AgdaUnderscore{}⟧₀}}\AgdaSpace{}%
\AgdaSymbol{(}\AgdaBound{SP}\AgdaSpace{}%
\AgdaSymbol{:}\AgdaSpace{}%
\AgdaRecord{Cont}\AgdaSymbol{)}\AgdaSpace{}%
\AgdaSymbol{(}\AgdaBound{X}\AgdaSpace{}%
\AgdaSymbol{:}\AgdaSpace{}%
\AgdaFunction{Type}\AgdaSymbol{)}\AgdaSpace{}%
\AgdaSymbol{:}\AgdaSpace{}%
\AgdaFunction{Type}\AgdaSpace{}%
\AgdaKeyword{where}\<%
\\
\>[0][@{}l@{\AgdaIndent{0}}]%
\>[2]\AgdaKeyword{constructor}\AgdaSpace{}%
\AgdaOperator{\AgdaInductiveConstructor{\AgdaUnderscore{},\AgdaUnderscore{}}}\<%
\\
%
\>[2]\AgdaKeyword{open}\AgdaSpace{}%
\AgdaModule{Cont}\AgdaSpace{}%
\AgdaBound{SP}\<%
\\
%
\>[2]\AgdaKeyword{field}\<%
\\
\>[2][@{}l@{\AgdaIndent{0}}]%
\>[4]\AgdaField{s}\AgdaSpace{}%
\AgdaSymbol{:}\AgdaSpace{}%
\AgdaFunction{S}\<%
\\
%
\>[4]\AgdaField{k}\AgdaSpace{}%
\AgdaSymbol{:}\AgdaSpace{}%
\AgdaFunction{P}\AgdaSpace{}%
\AgdaField{s}\AgdaSpace{}%
\AgdaSymbol{→}\AgdaSpace{}%
\AgdaBound{X}\<%
\\
%
\\[\AgdaEmptyExtraSkip]%
\>[0]\AgdaOperator{\AgdaFunction{⟦\AgdaUnderscore{}⟧₁}}\AgdaSpace{}%
\AgdaSymbol{:}\AgdaSpace{}%
\AgdaSymbol{(}\AgdaBound{SP}\AgdaSpace{}%
\AgdaSymbol{:}\AgdaSpace{}%
\AgdaRecord{Cont}\AgdaSymbol{)}\AgdaSpace{}%
\AgdaSymbol{→}\AgdaSpace{}%
\AgdaSymbol{(}\AgdaGeneralizable{X}\AgdaSpace{}%
\AgdaSymbol{→}\AgdaSpace{}%
\AgdaGeneralizable{Y}\AgdaSymbol{)}\AgdaSpace{}%
\AgdaSymbol{→}\AgdaSpace{}%
\AgdaOperator{\AgdaRecord{⟦}}\AgdaSpace{}%
\AgdaBound{SP}\AgdaSpace{}%
\AgdaOperator{\AgdaRecord{⟧₀}}\AgdaSpace{}%
\AgdaGeneralizable{X}\AgdaSpace{}%
\AgdaSymbol{→}\AgdaSpace{}%
\AgdaOperator{\AgdaRecord{⟦}}\AgdaSpace{}%
\AgdaBound{SP}\AgdaSpace{}%
\AgdaOperator{\AgdaRecord{⟧₀}}\AgdaSpace{}%
\AgdaGeneralizable{Y}\<%
\\
\>[0]\AgdaOperator{\AgdaFunction{⟦}}\AgdaSpace{}%
\AgdaBound{SP}\AgdaSpace{}%
\AgdaOperator{\AgdaFunction{⟧₁}}\AgdaSpace{}%
\AgdaBound{f}\AgdaSpace{}%
\AgdaSymbol{(}\AgdaBound{s}\AgdaSpace{}%
\AgdaOperator{\AgdaInductiveConstructor{,}}\AgdaSpace{}%
\AgdaBound{k}\AgdaSymbol{)}\AgdaSpace{}%
\AgdaSymbol{=}\AgdaSpace{}%
\AgdaBound{s}\AgdaSpace{}%
\AgdaOperator{\AgdaInductiveConstructor{,}}\AgdaSpace{}%
\AgdaBound{f}\AgdaSpace{}%
\AgdaOperator{\AgdaFunction{∘}}\AgdaSpace{}%
\AgdaBound{k}\<%
\end{code}

\subsection{Categorical Structures}

Containers and their morphisms form a category. The morphisms are defined as follow:

\begin{code}%
\>[0]\AgdaKeyword{record}\AgdaSpace{}%
\AgdaRecord{ContHom}\AgdaSpace{}%
\AgdaSymbol{(}\AgdaBound{SP}\AgdaSpace{}%
\AgdaBound{TQ}\AgdaSpace{}%
\AgdaSymbol{:}\AgdaSpace{}%
\AgdaRecord{Cont}\AgdaSymbol{)}\AgdaSpace{}%
\AgdaSymbol{:}\AgdaSpace{}%
\AgdaFunction{Type}\AgdaSpace{}%
\AgdaKeyword{where}\<%
\\
\>[0][@{}l@{\AgdaIndent{0}}]%
\>[2]\AgdaKeyword{constructor}\AgdaSpace{}%
\AgdaOperator{\AgdaInductiveConstructor{\AgdaUnderscore{}◃\AgdaUnderscore{}}}\<%
\\
%
\>[2]\AgdaKeyword{open}\AgdaSpace{}%
\AgdaModule{Cont}\AgdaSpace{}%
\AgdaBound{SP}\<%
\\
%
\>[2]\AgdaKeyword{open}\AgdaSpace{}%
\AgdaModule{Cont}\AgdaSpace{}%
\AgdaBound{TQ}\AgdaSpace{}%
\AgdaKeyword{renaming}\AgdaSpace{}%
\AgdaSymbol{(}\AgdaField{S}\AgdaSpace{}%
\AgdaSymbol{to}\AgdaSpace{}%
\AgdaField{T}\AgdaSymbol{;}\AgdaSpace{}%
\AgdaField{P}\AgdaSpace{}%
\AgdaSymbol{to}\AgdaSpace{}%
\AgdaField{Q}\AgdaSymbol{)}\<%
\\
%
\>[2]\AgdaKeyword{field}\<%
\\
\>[2][@{}l@{\AgdaIndent{0}}]%
\>[4]\AgdaField{f}\AgdaSpace{}%
\AgdaSymbol{:}\AgdaSpace{}%
\AgdaFunction{S}\AgdaSpace{}%
\AgdaSymbol{→}\AgdaSpace{}%
\AgdaFunction{T}\<%
\\
%
\>[4]\AgdaField{g}\AgdaSpace{}%
\AgdaSymbol{:}\AgdaSpace{}%
\AgdaSymbol{(}\AgdaBound{s}\AgdaSpace{}%
\AgdaSymbol{:}\AgdaSpace{}%
\AgdaFunction{S}\AgdaSymbol{)}\AgdaSpace{}%
\AgdaSymbol{→}\AgdaSpace{}%
\AgdaFunction{Q}\AgdaSpace{}%
\AgdaSymbol{(}\AgdaField{f}\AgdaSpace{}%
\AgdaBound{s}\AgdaSymbol{)}\AgdaSpace{}%
\AgdaSymbol{→}\AgdaSpace{}%
\AgdaFunction{P}\AgdaSpace{}%
\AgdaBound{s}\<%
\end{code}

As containers give rise to functors, the morphisms of containers should naturally give rise to the morphisms of functors - the natural transformations:

\begin{code}%
\>[0]\AgdaOperator{\AgdaFunction{⟦\AgdaUnderscore{}⟧Hom}}\AgdaSpace{}%
\AgdaSymbol{:}\AgdaSpace{}%
\AgdaRecord{ContHom}\AgdaSpace{}%
\AgdaGeneralizable{SP}\AgdaSpace{}%
\AgdaGeneralizable{TQ}\AgdaSpace{}%
\AgdaSymbol{→}\AgdaSpace{}%
\AgdaSymbol{(}\AgdaBound{X}\AgdaSpace{}%
\AgdaSymbol{:}\AgdaSpace{}%
\AgdaFunction{Type}\AgdaSymbol{)}\AgdaSpace{}%
\AgdaSymbol{→}\AgdaSpace{}%
\AgdaOperator{\AgdaRecord{⟦}}\AgdaSpace{}%
\AgdaGeneralizable{SP}\AgdaSpace{}%
\AgdaOperator{\AgdaRecord{⟧₀}}\AgdaSpace{}%
\AgdaBound{X}\AgdaSpace{}%
\AgdaSymbol{→}\AgdaSpace{}%
\AgdaOperator{\AgdaRecord{⟦}}\AgdaSpace{}%
\AgdaGeneralizable{TQ}\AgdaSpace{}%
\AgdaOperator{\AgdaRecord{⟧₀}}\AgdaSpace{}%
\AgdaBound{X}\<%
\\
\>[0]\AgdaOperator{\AgdaFunction{⟦}}\AgdaSpace{}%
\AgdaBound{f}\AgdaSpace{}%
\AgdaOperator{\AgdaInductiveConstructor{◃}}\AgdaSpace{}%
\AgdaBound{g}\AgdaSpace{}%
\AgdaOperator{\AgdaFunction{⟧Hom}}\AgdaSpace{}%
\AgdaBound{X}\AgdaSpace{}%
\AgdaSymbol{(}\AgdaBound{s}\AgdaSpace{}%
\AgdaOperator{\AgdaInductiveConstructor{,}}\AgdaSpace{}%
\AgdaBound{k}\AgdaSymbol{)}\AgdaSpace{}%
\AgdaSymbol{=}\AgdaSpace{}%
\AgdaBound{f}\AgdaSpace{}%
\AgdaBound{s}\AgdaSpace{}%
\AgdaOperator{\AgdaInductiveConstructor{,}}\AgdaSpace{}%
\AgdaBound{k}\AgdaSpace{}%
\AgdaOperator{\AgdaFunction{∘}}\AgdaSpace{}%
\AgdaBound{g}\AgdaSpace{}%
\AgdaBound{s}\<%
\end{code}

\subsection{Algebraic Structures}

Containers are also known as \textbf{polynomial functors} as they forms a semiring structure. Namely, we can define one, zero, multiplication and addition for containers:

\begin{code}%
\>[0]\AgdaFunction{oneC}\AgdaSpace{}%
\AgdaSymbol{:}\AgdaSpace{}%
\AgdaRecord{Cont}\<%
\\
\>[0]\AgdaFunction{oneC}\AgdaSpace{}%
\AgdaSymbol{=}\AgdaSpace{}%
\AgdaRecord{⊤}\AgdaSpace{}%
\AgdaOperator{\AgdaInductiveConstructor{◃}}\AgdaSpace{}%
\AgdaSymbol{λ}\AgdaSpace{}%
\AgdaBound{x}\AgdaSpace{}%
\AgdaSymbol{→}\AgdaSpace{}%
\AgdaFunction{⊥}\<%
\\
%
\\[\AgdaEmptyExtraSkip]%
\>[0]\AgdaFunction{zeroC}\AgdaSpace{}%
\AgdaSymbol{:}\AgdaSpace{}%
\AgdaRecord{Cont}\<%
\\
\>[0]\AgdaFunction{zeroC}\AgdaSpace{}%
\AgdaSymbol{=}\AgdaSpace{}%
\AgdaFunction{⊥}\AgdaSpace{}%
\AgdaOperator{\AgdaInductiveConstructor{◃}}\AgdaSpace{}%
\AgdaSymbol{λ}\AgdaSpace{}%
\AgdaSymbol{()}\<%
\\
%
\\[\AgdaEmptyExtraSkip]%
\>[0]\AgdaOperator{\AgdaFunction{\AgdaUnderscore{}×C\AgdaUnderscore{}}}\AgdaSpace{}%
\AgdaSymbol{:}\AgdaSpace{}%
\AgdaRecord{Cont}\AgdaSpace{}%
\AgdaSymbol{→}\AgdaSpace{}%
\AgdaRecord{Cont}\AgdaSpace{}%
\AgdaSymbol{→}\AgdaSpace{}%
\AgdaRecord{Cont}\<%
\\
\>[0]\AgdaSymbol{(}\AgdaBound{S}\AgdaSpace{}%
\AgdaOperator{\AgdaInductiveConstructor{◃}}\AgdaSpace{}%
\AgdaBound{P}\AgdaSymbol{)}\AgdaSpace{}%
\AgdaOperator{\AgdaFunction{×C}}\AgdaSpace{}%
\AgdaSymbol{(}\AgdaBound{T}\AgdaSpace{}%
\AgdaOperator{\AgdaInductiveConstructor{◃}}\AgdaSpace{}%
\AgdaBound{Q}\AgdaSymbol{)}\AgdaSpace{}%
\AgdaSymbol{=}\AgdaSpace{}%
\AgdaSymbol{(}\AgdaBound{S}\AgdaSpace{}%
\AgdaOperator{\AgdaFunction{×}}\AgdaSpace{}%
\AgdaBound{T}\AgdaSymbol{)}\AgdaSpace{}%
\AgdaOperator{\AgdaInductiveConstructor{◃}}\AgdaSpace{}%
\AgdaSymbol{λ}\AgdaSpace{}%
\AgdaSymbol{(}\AgdaBound{s}\AgdaSpace{}%
\AgdaOperator{\AgdaInductiveConstructor{,}}\AgdaSpace{}%
\AgdaBound{t}\AgdaSymbol{)}\AgdaSpace{}%
\AgdaSymbol{→}\AgdaSpace{}%
\AgdaBound{P}\AgdaSpace{}%
\AgdaBound{s}\AgdaSpace{}%
\AgdaOperator{\AgdaDatatype{⊎}}\AgdaSpace{}%
\AgdaBound{Q}\AgdaSpace{}%
\AgdaBound{t}\<%
\\
%
\\[\AgdaEmptyExtraSkip]%
\>[0]\AgdaOperator{\AgdaFunction{\AgdaUnderscore{}⊎C\AgdaUnderscore{}}}\AgdaSpace{}%
\AgdaSymbol{:}\AgdaSpace{}%
\AgdaRecord{Cont}\AgdaSpace{}%
\AgdaSymbol{→}\AgdaSpace{}%
\AgdaRecord{Cont}\AgdaSpace{}%
\AgdaSymbol{→}\AgdaSpace{}%
\AgdaRecord{Cont}\<%
\\
\>[0]\AgdaSymbol{(}\AgdaBound{S}\AgdaSpace{}%
\AgdaOperator{\AgdaInductiveConstructor{◃}}\AgdaSpace{}%
\AgdaBound{P}\AgdaSymbol{)}\AgdaSpace{}%
\AgdaOperator{\AgdaFunction{⊎C}}\AgdaSpace{}%
\AgdaSymbol{(}\AgdaBound{T}\AgdaSpace{}%
\AgdaOperator{\AgdaInductiveConstructor{◃}}\AgdaSpace{}%
\AgdaBound{Q}\AgdaSymbol{)}\AgdaSpace{}%
\AgdaSymbol{=}\AgdaSpace{}%
\AgdaSymbol{(}\AgdaBound{S}\AgdaSpace{}%
\AgdaOperator{\AgdaDatatype{⊎}}\AgdaSpace{}%
\AgdaBound{T}\AgdaSymbol{)}\AgdaSpace{}%
\AgdaOperator{\AgdaInductiveConstructor{◃}}\AgdaSpace{}%
\AgdaSymbol{λ\{}\AgdaSpace{}%
\AgdaSymbol{(}\AgdaInductiveConstructor{inj₁}\AgdaSpace{}%
\AgdaBound{s}\AgdaSymbol{)}\AgdaSpace{}%
\AgdaSymbol{→}\AgdaSpace{}%
\AgdaBound{P}\AgdaSpace{}%
\AgdaBound{s}\AgdaSpace{}%
\AgdaSymbol{;}\AgdaSpace{}%
\AgdaSymbol{(}\AgdaInductiveConstructor{inj₂}\AgdaSpace{}%
\AgdaBound{t}\AgdaSymbol{)}\AgdaSpace{}%
\AgdaSymbol{→}\AgdaSpace{}%
\AgdaBound{Q}\AgdaSpace{}%
\AgdaBound{t}\AgdaSpace{}%
\AgdaSymbol{\}}\<%
\end{code}

such that the semiring laws should hold. In fact, both multiplication and addition are commutative, associative and left- and right-annihilated by their units. It turns out they also exactly correspond to the initial, terminal, product and coproduct of the containers. 

Even better, we can generalize \AgdaRecord{×} and \AgdaFunction{⊎} to \AgdaFunction{Π} and \AgdaRecord{Σ} for containers, which are finitary product and coproduct objects.

\begin{code}[hide]%
\>[0]\AgdaKeyword{module}\AgdaSpace{}%
\AgdaModule{\AgdaUnderscore{}}\AgdaSpace{}%
\AgdaKeyword{where}\<%
\\
\>[0][@{}l@{\AgdaIndent{0}}]%
\>[2]\AgdaKeyword{open}\AgdaSpace{}%
\AgdaModule{Cont}\<%
\end{code}

\begin{code}%
%
\>[2]\AgdaFunction{ΠC}\AgdaSpace{}%
\AgdaSymbol{:}\AgdaSpace{}%
\AgdaSymbol{(}\AgdaGeneralizable{I}\AgdaSpace{}%
\AgdaSymbol{→}\AgdaSpace{}%
\AgdaRecord{Cont}\AgdaSymbol{)}\AgdaSpace{}%
\AgdaSymbol{→}\AgdaSpace{}%
\AgdaRecord{Cont}\<%
\\
%
\>[2]\AgdaFunction{ΠC}\AgdaSpace{}%
\AgdaSymbol{\{}\AgdaBound{I}\AgdaSymbol{\}}\AgdaSpace{}%
\AgdaBound{SPs}\AgdaSpace{}%
\AgdaSymbol{=}\AgdaSpace{}%
\AgdaSymbol{((}\AgdaBound{i}\AgdaSpace{}%
\AgdaSymbol{:}\AgdaSpace{}%
\AgdaBound{I}\AgdaSymbol{)}\AgdaSpace{}%
\AgdaSymbol{→}\AgdaSpace{}%
\AgdaBound{SPs}\AgdaSpace{}%
\AgdaBound{i}\AgdaSpace{}%
\AgdaSymbol{.}\AgdaField{S}\AgdaSymbol{)}\AgdaSpace{}%
\AgdaOperator{\AgdaInductiveConstructor{◃}}\AgdaSpace{}%
\AgdaSymbol{λ}\AgdaSpace{}%
\AgdaBound{f}\AgdaSpace{}%
\AgdaSymbol{→}\AgdaSpace{}%
\AgdaFunction{Σ[}\AgdaSpace{}%
\AgdaBound{i}\AgdaSpace{}%
\AgdaFunction{∈}\AgdaSpace{}%
\AgdaBound{I}\AgdaSpace{}%
\AgdaFunction{]}\AgdaSpace{}%
\AgdaBound{SPs}\AgdaSpace{}%
\AgdaBound{i}\AgdaSpace{}%
\AgdaSymbol{.}\AgdaField{P}\AgdaSpace{}%
\AgdaSymbol{(}\AgdaBound{f}\AgdaSpace{}%
\AgdaBound{i}\AgdaSymbol{)}\<%
\\
%
\\[\AgdaEmptyExtraSkip]%
%
\>[2]\AgdaFunction{ΣC}\AgdaSpace{}%
\AgdaSymbol{:}\AgdaSpace{}%
\AgdaSymbol{(}\AgdaGeneralizable{I}\AgdaSpace{}%
\AgdaSymbol{→}\AgdaSpace{}%
\AgdaRecord{Cont}\AgdaSymbol{)}\AgdaSpace{}%
\AgdaSymbol{→}\AgdaSpace{}%
\AgdaRecord{Cont}\<%
\\
%
\>[2]\AgdaFunction{ΣC}\AgdaSpace{}%
\AgdaSymbol{\{}\AgdaBound{I}\AgdaSymbol{\}}\AgdaSpace{}%
\AgdaBound{SPs}\AgdaSpace{}%
\AgdaSymbol{=}\AgdaSpace{}%
\AgdaSymbol{(}\AgdaFunction{Σ[}\AgdaSpace{}%
\AgdaBound{i}\AgdaSpace{}%
\AgdaFunction{∈}\AgdaSpace{}%
\AgdaBound{I}\AgdaSpace{}%
\AgdaFunction{]}\AgdaSpace{}%
\AgdaBound{SPs}\AgdaSpace{}%
\AgdaBound{i}\AgdaSpace{}%
\AgdaSymbol{.}\AgdaField{S}\AgdaSymbol{)}\AgdaSpace{}%
\AgdaOperator{\AgdaInductiveConstructor{◃}}\AgdaSpace{}%
\AgdaSymbol{λ}\AgdaSpace{}%
\AgdaSymbol{(}\AgdaBound{i}\AgdaSpace{}%
\AgdaOperator{\AgdaInductiveConstructor{,}}\AgdaSpace{}%
\AgdaBound{s}\AgdaSymbol{)}\AgdaSpace{}%
\AgdaSymbol{→}\AgdaSpace{}%
\AgdaBound{SPs}\AgdaSpace{}%
\AgdaBound{i}\AgdaSpace{}%
\AgdaSymbol{.}\AgdaField{P}\AgdaSpace{}%
\AgdaBound{s}\<%
\end{code}

\subsection{W and M}

With containers, we now give the definition of general form of inductive types, which is the \AgdaDatatype{W} type:

\begin{code}%
\>[0]\AgdaKeyword{data}\AgdaSpace{}%
\AgdaDatatype{W}\AgdaSpace{}%
\AgdaSymbol{(}\AgdaBound{SP}\AgdaSpace{}%
\AgdaSymbol{:}\AgdaSpace{}%
\AgdaRecord{Cont}\AgdaSymbol{)}\AgdaSpace{}%
\AgdaSymbol{:}\AgdaSpace{}%
\AgdaFunction{Type}\AgdaSpace{}%
\AgdaKeyword{where}\<%
\\
\>[0][@{}l@{\AgdaIndent{0}}]%
\>[2]\AgdaInductiveConstructor{sup}\AgdaSpace{}%
\AgdaSymbol{:}\AgdaSpace{}%
\AgdaOperator{\AgdaRecord{⟦}}\AgdaSpace{}%
\AgdaBound{SP}\AgdaSpace{}%
\AgdaOperator{\AgdaRecord{⟧₀}}\AgdaSpace{}%
\AgdaSymbol{(}\AgdaDatatype{W}\AgdaSpace{}%
\AgdaBound{SP}\AgdaSymbol{)}\AgdaSpace{}%
\AgdaSymbol{→}\AgdaSpace{}%
\AgdaDatatype{W}\AgdaSpace{}%
\AgdaBound{SP}\<%
\end{code}

From the definition, \AgdaDatatype{W} is an inductive type that specified by a container. In another word, any inductive type can be specified by a strictly positive functor. We can retrieve \AgdaDatatype{ℕ} through the \AgdaDatatype{⊤ ⊎} container:

\begin{code}%
\>[0]\AgdaFunction{⊤⊎Cont}\AgdaSpace{}%
\AgdaSymbol{:}\AgdaSpace{}%
\AgdaRecord{Cont}\<%
\\
\>[0]\AgdaFunction{⊤⊎Cont}\AgdaSpace{}%
\AgdaSymbol{=}\AgdaSpace{}%
\AgdaSymbol{(}\AgdaRecord{⊤}\AgdaSpace{}%
\AgdaOperator{\AgdaDatatype{⊎}}\AgdaSpace{}%
\AgdaRecord{⊤}\AgdaSymbol{)}\AgdaSpace{}%
\AgdaOperator{\AgdaInductiveConstructor{◃}}\AgdaSpace{}%
\AgdaSymbol{λ\{}\AgdaSpace{}%
\AgdaSymbol{(}\AgdaInductiveConstructor{inj₁}\AgdaSpace{}%
\AgdaInductiveConstructor{tt}\AgdaSymbol{)}\AgdaSpace{}%
\AgdaSymbol{→}\AgdaSpace{}%
\AgdaFunction{⊥}\AgdaSpace{}%
\AgdaSymbol{;}\AgdaSpace{}%
\AgdaSymbol{(}\AgdaInductiveConstructor{inj₂}\AgdaSpace{}%
\AgdaBound{y}\AgdaSymbol{)}\AgdaSpace{}%
\AgdaSymbol{→}\AgdaSpace{}%
\AgdaRecord{⊤}\AgdaSpace{}%
\AgdaSymbol{\}}\<%
\\
%
\\[\AgdaEmptyExtraSkip]%
\>[0]\AgdaFunction{ℕW}\AgdaSpace{}%
\AgdaSymbol{:}\AgdaSpace{}%
\AgdaFunction{Type}\<%
\\
\>[0]\AgdaFunction{ℕW}\AgdaSpace{}%
\AgdaSymbol{=}\AgdaSpace{}%
\AgdaDatatype{W}\AgdaSpace{}%
\AgdaFunction{⊤⊎Cont}\<%
\end{code}

Dually, the general form of coinductive types - \AgdaRecord{M} type is just the terminal coalgebra of containers. \AgdaRecord{W} and \AgdaRecord{ℕ∞} are defined as follow:

\begin{code}%
\>[0]\AgdaKeyword{record}\AgdaSpace{}%
\AgdaRecord{M}\AgdaSpace{}%
\AgdaSymbol{(}\AgdaBound{SP}\AgdaSpace{}%
\AgdaSymbol{:}\AgdaSpace{}%
\AgdaRecord{Cont}\AgdaSymbol{)}\AgdaSpace{}%
\AgdaSymbol{:}\AgdaSpace{}%
\AgdaFunction{Type}\AgdaSpace{}%
\AgdaKeyword{where}\<%
\\
\>[0][@{}l@{\AgdaIndent{0}}]%
\>[2]\AgdaKeyword{coinductive}\<%
\\
%
\>[2]\AgdaKeyword{field}\<%
\\
\>[2][@{}l@{\AgdaIndent{0}}]%
\>[4]\AgdaField{inf}\AgdaSpace{}%
\AgdaSymbol{:}\AgdaSpace{}%
\AgdaOperator{\AgdaRecord{⟦}}\AgdaSpace{}%
\AgdaBound{SP}\AgdaSpace{}%
\AgdaOperator{\AgdaRecord{⟧₀}}\AgdaSpace{}%
\AgdaSymbol{(}\AgdaRecord{M}\AgdaSpace{}%
\AgdaBound{SP}\AgdaSymbol{)}\<%
\\
%
\\[\AgdaEmptyExtraSkip]%
\>[0]\AgdaFunction{ℕ∞M}\AgdaSpace{}%
\AgdaSymbol{:}\AgdaSpace{}%
\AgdaFunction{Type}\<%
\\
\>[0]\AgdaFunction{ℕ∞M}\AgdaSpace{}%
\AgdaSymbol{=}\AgdaSpace{}%
\AgdaRecord{M}\AgdaSpace{}%
\AgdaFunction{⊤⊎Cont}\<%
\end{code}

Finally, we have the following commutative diagrams for \AgdaDatatype{W} and \AgdaRecord{M}:

\[
\begin{tikzcd}[row sep=huge, column sep=5.0cm]
\AgdaRecord{⟦ }\AgdaBound{SP }\AgdaRecord{⟧₀ }\AgdaSymbol{(}\AgdaDatatype{W }\AgdaBound{SP}\AgdaSymbol{)} \arrow[r, "\AgdaFunction{⟦ }\AgdaBound{SP }\AgdaFunction{⟧₁ }\AgdaSymbol{(}\AgdaFunction{fold }\AgdaBound{α}\AgdaSymbol{)}"] \arrow[d, "\AgdaInductiveConstructor{sup}"]
& \AgdaRecord{⟦ }\AgdaBound{SP }\AgdaRecord{⟧₀ }\AgdaBound{X} \arrow[d, "\AgdaBound{α}"] \\
\AgdaDatatype{W }\AgdaBound{SP} \arrow[r, "\AgdaFunction{fold }\AgdaBound{α}"]
& \AgdaBound{X}
\end{tikzcd}
\]

\[
\begin{tikzcd}[row sep=huge, column sep=5.0cm]
\AgdaBound{X} \arrow[r, "\AgdaFunction{unfold }\AgdaBound{α⁻}"] \arrow[d, "\AgdaBound{α⁻}"]
& \AgdaRecord{M } \AgdaBound{SP}\arrow[d, "\AgdaField{inf}"] \\
\AgdaRecord{⟦ }\AgdaBound{SP }\AgdaRecord{⟧₀ }\AgdaBound{X} \arrow[r, "\AgdaFunction{⟦ }\AgdaBound{SP }\AgdaFunction{⟧₁ }\AgdaSymbol{(}\AgdaFunction{unfold }\AgdaBound{α⁻}\AgdaSymbol{)}"]
& \AgdaRecord{⟦ }\AgdaBound{SP }\AgdaRecord{⟧₀ }\AgdaSymbol{(}\AgdaRecord{M }\AgdaBound{SP}\AgdaSymbol{)}
\end{tikzcd}
\]

We have now showed inductive types are initial algebras and coinductive types are terminal coalgebras.

\section{Simply-typed Categories with Families}

TODO : What is CwFs?

\section{Normalization by Hereditary Substitutions}

We introduce the hereditary substitutions for \lambda-calculus normalization. To build the syntax of \lambda-calculus, we first define types and contexts:

\begin{code}%
\>[0]\AgdaKeyword{data}\AgdaSpace{}%
\AgdaDatatype{Ty}\AgdaSpace{}%
\AgdaSymbol{:}\AgdaSpace{}%
\AgdaFunction{Type}\AgdaSpace{}%
\AgdaKeyword{where}\<%
\\
\>[0][@{}l@{\AgdaIndent{0}}]%
\>[2]\AgdaInductiveConstructor{*}%
\>[6]\AgdaSymbol{:}\AgdaSpace{}%
\AgdaDatatype{Ty}\<%
\\
%
\>[2]\AgdaOperator{\AgdaInductiveConstructor{\AgdaUnderscore{}⇒\AgdaUnderscore{}}}\AgdaSpace{}%
\AgdaSymbol{:}\AgdaSpace{}%
\AgdaDatatype{Ty}\AgdaSpace{}%
\AgdaSymbol{→}\AgdaSpace{}%
\AgdaDatatype{Ty}\AgdaSpace{}%
\AgdaSymbol{→}\AgdaSpace{}%
\AgdaDatatype{Ty}\<%
\\
%
\\[\AgdaEmptyExtraSkip]%
\>[0]\AgdaKeyword{data}\AgdaSpace{}%
\AgdaDatatype{Con}\AgdaSpace{}%
\AgdaSymbol{:}\AgdaSpace{}%
\AgdaFunction{Type}\AgdaSpace{}%
\AgdaKeyword{where}\<%
\\
\>[0][@{}l@{\AgdaIndent{0}}]%
\>[2]\AgdaInductiveConstructor{∙}%
\>[6]\AgdaSymbol{:}\AgdaSpace{}%
\AgdaDatatype{Con}\<%
\\
%
\>[2]\AgdaOperator{\AgdaInductiveConstructor{\AgdaUnderscore{}▹\AgdaUnderscore{}}}\AgdaSpace{}%
\AgdaSymbol{:}\AgdaSpace{}%
\AgdaDatatype{Con}\AgdaSpace{}%
\AgdaSymbol{→}\AgdaSpace{}%
\AgdaDatatype{Ty}\AgdaSpace{}%
\AgdaSymbol{→}\AgdaSpace{}%
\AgdaDatatype{Con}\<%
\end{code}

\begin{code}[hide]%
\>[0]\AgdaKeyword{variable}\AgdaSpace{}%
\AgdaGeneralizable{A}\AgdaSpace{}%
\AgdaGeneralizable{B}\AgdaSpace{}%
\AgdaGeneralizable{C}\AgdaSpace{}%
\AgdaSymbol{:}\AgdaSpace{}%
\AgdaDatatype{Ty}\<%
\\
\>[0]\AgdaKeyword{variable}\AgdaSpace{}%
\AgdaGeneralizable{Γ}\AgdaSpace{}%
\AgdaGeneralizable{Δ}\AgdaSpace{}%
\AgdaSymbol{:}\AgdaSpace{}%
\AgdaDatatype{Con}\<%
\end{code}

We adopt the De Bruijn indices to represent bound variables without explicit naming.

\begin{code}%
\>[0]\AgdaKeyword{data}\AgdaSpace{}%
\AgdaDatatype{Var}\AgdaSpace{}%
\AgdaSymbol{:}\AgdaSpace{}%
\AgdaDatatype{Con}\AgdaSpace{}%
\AgdaSymbol{→}\AgdaSpace{}%
\AgdaDatatype{Ty}\AgdaSpace{}%
\AgdaSymbol{→}\AgdaSpace{}%
\AgdaFunction{Type}\AgdaSpace{}%
\AgdaKeyword{where}\<%
\\
\>[0][@{}l@{\AgdaIndent{0}}]%
\>[2]\AgdaInductiveConstructor{vz}\AgdaSpace{}%
\AgdaSymbol{:}\AgdaSpace{}%
\AgdaDatatype{Var}\AgdaSpace{}%
\AgdaSymbol{(}\AgdaGeneralizable{Γ}\AgdaSpace{}%
\AgdaOperator{\AgdaInductiveConstructor{▹}}\AgdaSpace{}%
\AgdaGeneralizable{A}\AgdaSymbol{)}\AgdaSpace{}%
\AgdaGeneralizable{A}\<%
\\
%
\>[2]\AgdaInductiveConstructor{vs}\AgdaSpace{}%
\AgdaSymbol{:}\AgdaSpace{}%
\AgdaDatatype{Var}\AgdaSpace{}%
\AgdaGeneralizable{Γ}\AgdaSpace{}%
\AgdaGeneralizable{A}\AgdaSpace{}%
\AgdaSymbol{→}\AgdaSpace{}%
\AgdaDatatype{Var}\AgdaSpace{}%
\AgdaSymbol{(}\AgdaGeneralizable{Γ}\AgdaSpace{}%
\AgdaOperator{\AgdaInductiveConstructor{▹}}\AgdaSpace{}%
\AgdaGeneralizable{B}\AgdaSymbol{)}\AgdaSpace{}%
\AgdaGeneralizable{A}\<%
\end{code}

\begin{code}[hide]%
\>[0]\AgdaKeyword{mutual}\<%
\end{code}

The normal form \lambda-terms \AgdaDatatype{Nf} are defined as either:

\begin{itemize}
  \item{\lambda-abstractions of \AgdaDatatype{Nf};}
  \item{Or neutral terms \AgdaDatatype{Ne}, where \AgdaDatatype{Ne} are variables \AgdaDatatype{Var} applied to lists of \AgdaDatatype{Nf}, called spine \AgdaDatatype{Sp}}.
\end{itemize}

\begin{code}%
\>[0][@{}l@{\AgdaIndent{1}}]%
\>[2]\AgdaKeyword{data}\AgdaSpace{}%
\AgdaDatatype{Nf}\AgdaSpace{}%
\AgdaSymbol{:}\AgdaSpace{}%
\AgdaDatatype{Con}\AgdaSpace{}%
\AgdaSymbol{→}\AgdaSpace{}%
\AgdaDatatype{Ty}\AgdaSpace{}%
\AgdaSymbol{→}\AgdaSpace{}%
\AgdaFunction{Type}\AgdaSpace{}%
\AgdaKeyword{where}\<%
\\
\>[2][@{}l@{\AgdaIndent{0}}]%
\>[4]\AgdaInductiveConstructor{lam}\AgdaSpace{}%
\AgdaSymbol{:}\AgdaSpace{}%
\AgdaDatatype{Nf}\AgdaSpace{}%
\AgdaSymbol{(}\AgdaGeneralizable{Γ}\AgdaSpace{}%
\AgdaOperator{\AgdaInductiveConstructor{▹}}\AgdaSpace{}%
\AgdaGeneralizable{A}\AgdaSymbol{)}\AgdaSpace{}%
\AgdaGeneralizable{B}\AgdaSpace{}%
\AgdaSymbol{→}\AgdaSpace{}%
\AgdaDatatype{Nf}\AgdaSpace{}%
\AgdaGeneralizable{Γ}\AgdaSpace{}%
\AgdaSymbol{(}\AgdaGeneralizable{A}\AgdaSpace{}%
\AgdaOperator{\AgdaInductiveConstructor{⇒}}\AgdaSpace{}%
\AgdaGeneralizable{B}\AgdaSymbol{)}\<%
\\
%
\>[4]\AgdaInductiveConstructor{ne}%
\>[8]\AgdaSymbol{:}\AgdaSpace{}%
\AgdaDatatype{Ne}\AgdaSpace{}%
\AgdaGeneralizable{Γ}\AgdaSpace{}%
\AgdaInductiveConstructor{*}\AgdaSpace{}%
\AgdaSymbol{→}\AgdaSpace{}%
\AgdaDatatype{Nf}\AgdaSpace{}%
\AgdaGeneralizable{Γ}\AgdaSpace{}%
\AgdaInductiveConstructor{*}\<%
\\
%
\\[\AgdaEmptyExtraSkip]%
%
\>[2]\AgdaKeyword{data}\AgdaSpace{}%
\AgdaDatatype{Ne}\AgdaSpace{}%
\AgdaSymbol{:}\AgdaSpace{}%
\AgdaDatatype{Con}\AgdaSpace{}%
\AgdaSymbol{→}\AgdaSpace{}%
\AgdaDatatype{Ty}\AgdaSpace{}%
\AgdaSymbol{→}\AgdaSpace{}%
\AgdaFunction{Type}\AgdaSpace{}%
\AgdaKeyword{where}\<%
\\
\>[2][@{}l@{\AgdaIndent{0}}]%
\>[4]\AgdaOperator{\AgdaInductiveConstructor{\AgdaUnderscore{},\AgdaUnderscore{}}}\AgdaSpace{}%
\AgdaSymbol{:}\AgdaSpace{}%
\AgdaDatatype{Var}\AgdaSpace{}%
\AgdaGeneralizable{Γ}\AgdaSpace{}%
\AgdaGeneralizable{A}\AgdaSpace{}%
\AgdaSymbol{→}\AgdaSpace{}%
\AgdaDatatype{Sp}\AgdaSpace{}%
\AgdaGeneralizable{Γ}\AgdaSpace{}%
\AgdaGeneralizable{A}\AgdaSpace{}%
\AgdaGeneralizable{B}\AgdaSpace{}%
\AgdaSymbol{→}\AgdaSpace{}%
\AgdaDatatype{Ne}\AgdaSpace{}%
\AgdaGeneralizable{Γ}\AgdaSpace{}%
\AgdaGeneralizable{B}\<%
\\
%
\\[\AgdaEmptyExtraSkip]%
%
\>[2]\AgdaKeyword{data}\AgdaSpace{}%
\AgdaDatatype{Sp}\AgdaSpace{}%
\AgdaSymbol{:}\AgdaSpace{}%
\AgdaDatatype{Con}\AgdaSpace{}%
\AgdaSymbol{→}\AgdaSpace{}%
\AgdaDatatype{Ty}\AgdaSpace{}%
\AgdaSymbol{→}\AgdaSpace{}%
\AgdaDatatype{Ty}\AgdaSpace{}%
\AgdaSymbol{→}\AgdaSpace{}%
\AgdaFunction{Type}\AgdaSpace{}%
\AgdaKeyword{where}\<%
\\
\>[2][@{}l@{\AgdaIndent{0}}]%
\>[4]\AgdaInductiveConstructor{ε}%
\>[8]\AgdaSymbol{:}\AgdaSpace{}%
\AgdaDatatype{Sp}\AgdaSpace{}%
\AgdaGeneralizable{Γ}\AgdaSpace{}%
\AgdaGeneralizable{A}\AgdaSpace{}%
\AgdaGeneralizable{A}\<%
\\
%
\>[4]\AgdaOperator{\AgdaInductiveConstructor{\AgdaUnderscore{},\AgdaUnderscore{}}}\AgdaSpace{}%
\AgdaSymbol{:}\AgdaSpace{}%
\AgdaDatatype{Nf}\AgdaSpace{}%
\AgdaGeneralizable{Γ}\AgdaSpace{}%
\AgdaGeneralizable{A}\AgdaSpace{}%
\AgdaSymbol{→}\AgdaSpace{}%
\AgdaDatatype{Sp}\AgdaSpace{}%
\AgdaGeneralizable{Γ}\AgdaSpace{}%
\AgdaGeneralizable{B}\AgdaSpace{}%
\AgdaGeneralizable{C}\AgdaSpace{}%
\AgdaSymbol{→}\AgdaSpace{}%
\AgdaDatatype{Sp}\AgdaSpace{}%
\AgdaGeneralizable{Γ}\AgdaSpace{}%
\AgdaSymbol{(}\AgdaGeneralizable{A}\AgdaSpace{}%
\AgdaOperator{\AgdaInductiveConstructor{⇒}}\AgdaSpace{}%
\AgdaGeneralizable{B}\AgdaSymbol{)}\AgdaSpace{}%
\AgdaGeneralizable{C}\<%
\end{code}

\section{Questions}

TODO

\subsection{Bush}

\begin{code}%
\>[0]\AgdaKeyword{record}\AgdaSpace{}%
\AgdaRecord{Bush₀}\AgdaSpace{}%
\AgdaSymbol{(}\AgdaBound{X}\AgdaSpace{}%
\AgdaSymbol{:}\AgdaSpace{}%
\AgdaFunction{Type}\AgdaSymbol{)}\AgdaSpace{}%
\AgdaSymbol{:}\AgdaSpace{}%
\AgdaFunction{Type}\AgdaSpace{}%
\AgdaKeyword{where}\<%
\\
\>[0][@{}l@{\AgdaIndent{0}}]%
\>[2]\AgdaKeyword{coinductive}\<%
\\
%
\>[2]\AgdaKeyword{field}\<%
\\
\>[2][@{}l@{\AgdaIndent{0}}]%
\>[4]\AgdaField{head}\AgdaSpace{}%
\AgdaSymbol{:}\AgdaSpace{}%
\AgdaBound{X}\<%
\\
%
\>[4]\AgdaField{tail}\AgdaSpace{}%
\AgdaSymbol{:}\AgdaSpace{}%
\AgdaRecord{Bush₀}\AgdaSpace{}%
\AgdaSymbol{(}\AgdaRecord{Bush₀}\AgdaSpace{}%
\AgdaBound{X}\AgdaSymbol{)}\<%
\\
\>[0]\AgdaKeyword{open}\AgdaSpace{}%
\AgdaModule{Bush₀}\<%
\\
%
\\[\AgdaEmptyExtraSkip]%
\>[0]\AgdaSymbol{\{-\#}\AgdaSpace{}%
\AgdaKeyword{TERMINATING}\AgdaSpace{}%
\AgdaSymbol{\#-\}}\<%
\\
\>[0]\AgdaFunction{Bush₁}\AgdaSpace{}%
\AgdaSymbol{:}\AgdaSpace{}%
\AgdaSymbol{(}\AgdaGeneralizable{X}\AgdaSpace{}%
\AgdaSymbol{→}\AgdaSpace{}%
\AgdaGeneralizable{Y}\AgdaSymbol{)}\AgdaSpace{}%
\AgdaSymbol{→}\AgdaSpace{}%
\AgdaRecord{Bush₀}\AgdaSpace{}%
\AgdaGeneralizable{X}\AgdaSpace{}%
\AgdaSymbol{→}\AgdaSpace{}%
\AgdaRecord{Bush₀}\AgdaSpace{}%
\AgdaGeneralizable{Y}\<%
\\
\>[0]\AgdaField{head}\AgdaSpace{}%
\AgdaSymbol{(}\AgdaFunction{Bush₁}\AgdaSpace{}%
\AgdaBound{f}\AgdaSpace{}%
\AgdaBound{a}\AgdaSymbol{)}\AgdaSpace{}%
\AgdaSymbol{=}\AgdaSpace{}%
\AgdaBound{f}\AgdaSpace{}%
\AgdaSymbol{(}\AgdaField{head}\AgdaSpace{}%
\AgdaBound{a}\AgdaSymbol{)}\<%
\\
\>[0]\AgdaField{tail}\AgdaSpace{}%
\AgdaSymbol{(}\AgdaFunction{Bush₁}\AgdaSpace{}%
\AgdaBound{f}\AgdaSpace{}%
\AgdaBound{a}\AgdaSymbol{)}\AgdaSpace{}%
\AgdaSymbol{=}\AgdaSpace{}%
\AgdaFunction{Bush₁}\AgdaSpace{}%
\AgdaSymbol{(}\AgdaFunction{Bush₁}\AgdaSpace{}%
\AgdaBound{f}\AgdaSymbol{)}\AgdaSpace{}%
\AgdaSymbol{(}\AgdaField{tail}\AgdaSpace{}%
\AgdaBound{a}\AgdaSymbol{)}\<%
\end{code}

How do we represent \AgdaRecord{Bush} \AgdaBound{X} as a \AgdaRecord{M} type? It turns out that previews scheme is no longer applicable.

The key observation is to lift the space from \AgdaPrimitive{Type} to \AgdaPrimitive{Type} \AgdaSymbol{→} \AgdaPrimitive{Type}. We can show \AgdaRecord{Bush} is the terminal coalgebra of a ``higher'' endofunctor. We need to also define a ``higher'' algebra:

\begin{code}%
\>[0]\AgdaFunction{H}\AgdaSpace{}%
\AgdaSymbol{:}\AgdaSpace{}%
\AgdaSymbol{(}\AgdaFunction{Type}\AgdaSpace{}%
\AgdaSymbol{→}\AgdaSpace{}%
\AgdaFunction{Type}\AgdaSymbol{)}\AgdaSpace{}%
\AgdaSymbol{→}\AgdaSpace{}%
\AgdaFunction{Type}\AgdaSpace{}%
\AgdaSymbol{→}\AgdaSpace{}%
\AgdaFunction{Type}\<%
\\
\>[0]\AgdaFunction{H}\AgdaSpace{}%
\AgdaBound{F}\AgdaSpace{}%
\AgdaBound{X}\AgdaSpace{}%
\AgdaSymbol{=}\AgdaSpace{}%
\AgdaBound{X}\AgdaSpace{}%
\AgdaOperator{\AgdaFunction{×}}\AgdaSpace{}%
\AgdaBound{F}\AgdaSpace{}%
\AgdaSymbol{(}\AgdaBound{F}\AgdaSpace{}%
\AgdaBound{X}\AgdaSymbol{)}\<%
\end{code}

\[
\begin{tikzcd}[row sep=huge, column sep=5.0cm]
\AgdaBound{F} \arrow[r, "\AgdaFunction{unfold }\AgdaBound{α⁻}"] \arrow[d, "\AgdaBound{α⁻}"]
& \AgdaRecord{Bush₀} \arrow[d, "\AgdaFunction{< }\AgdaField{head }\AgdaFunction{, }\AgdaField{tail }\AgdaFunction{>}"] \\
\AgdaFunction{H }\AgdaBound{F} \arrow[r, "\AgdaDatatype{H₁ }\AgdaSymbol{(}\AgdaFunction{unfold }\AgdaBound{α⁻}\AgdaSymbol{)}"]
& \AgdaFunction{H }\AgdaRecord{Bush₀}
\end{tikzcd}
\]
\chapter{Research Outcomes}

\begin{code}[hide]%
\>[0]\AgdaKeyword{open}\AgdaSpace{}%
\AgdaKeyword{import}\AgdaSpace{}%
\AgdaModule{Data.Empty}\<%
\\
\>[0]\AgdaKeyword{open}\AgdaSpace{}%
\AgdaKeyword{import}\AgdaSpace{}%
\AgdaModule{Data.Unit}\<%
\\
\>[0]\AgdaKeyword{open}\AgdaSpace{}%
\AgdaKeyword{import}\AgdaSpace{}%
\AgdaModule{Data.Sum}\<%
\\
\>[0]\AgdaKeyword{open}\AgdaSpace{}%
\AgdaKeyword{import}\AgdaSpace{}%
\AgdaModule{Data.Product}\<%
\\
%
\\[\AgdaEmptyExtraSkip]%
\>[0]\AgdaKeyword{open}\AgdaSpace{}%
\AgdaKeyword{import}\AgdaSpace{}%
\AgdaModule{Level}\AgdaSpace{}%
\AgdaKeyword{renaming}\AgdaSpace{}%
\AgdaSymbol{(}\AgdaPrimitive{zero}\AgdaSpace{}%
\AgdaSymbol{to}\AgdaSpace{}%
\AgdaPrimitive{lzero}\AgdaSymbol{;}\AgdaSpace{}%
\AgdaPrimitive{suc}\AgdaSpace{}%
\AgdaSymbol{to}\AgdaSpace{}%
\AgdaPrimitive{lsuc}\AgdaSymbol{)}\<%
\\
%
\\[\AgdaEmptyExtraSkip]%
\>[0]\AgdaFunction{Type}\AgdaSpace{}%
\AgdaSymbol{:}\AgdaSpace{}%
\AgdaPrimitive{Set₁}\<%
\\
\>[0]\AgdaFunction{Type}\AgdaSpace{}%
\AgdaSymbol{=}\AgdaSpace{}%
\AgdaPrimitive{Set}\<%
\\
%
\\[\AgdaEmptyExtraSkip]%
\>[0]\AgdaFunction{Type₁}\AgdaSpace{}%
\AgdaSymbol{:}\AgdaSpace{}%
\AgdaPrimitive{Set₂}\<%
\\
\>[0]\AgdaFunction{Type₁}\AgdaSpace{}%
\AgdaSymbol{=}\AgdaSpace{}%
\AgdaPrimitive{Set₁}\<%
\end{code}

\section{Higher-Order Containers}

\subsection{Syntax}

\begin{code}[hide]%
\>[0]\AgdaKeyword{infixr}\AgdaSpace{}%
\AgdaNumber{20}\AgdaSpace{}%
\AgdaOperator{\AgdaInductiveConstructor{\AgdaUnderscore{}⇒\AgdaUnderscore{}}}\<%
\\
\>[0]\AgdaKeyword{data}\AgdaSpace{}%
\AgdaDatatype{Ty}\AgdaSpace{}%
\AgdaSymbol{:}\AgdaSpace{}%
\AgdaFunction{Type}\AgdaSpace{}%
\AgdaKeyword{where}\<%
\\
\>[0][@{}l@{\AgdaIndent{0}}]%
\>[2]\AgdaInductiveConstructor{*}\AgdaSpace{}%
\AgdaSymbol{:}\AgdaSpace{}%
\AgdaDatatype{Ty}\<%
\\
%
\>[2]\AgdaOperator{\AgdaInductiveConstructor{\AgdaUnderscore{}⇒\AgdaUnderscore{}}}\AgdaSpace{}%
\AgdaSymbol{:}\AgdaSpace{}%
\AgdaDatatype{Ty}\AgdaSpace{}%
\AgdaSymbol{→}\AgdaSpace{}%
\AgdaDatatype{Ty}\AgdaSpace{}%
\AgdaSymbol{→}\AgdaSpace{}%
\AgdaDatatype{Ty}\<%
\\
%
\\[\AgdaEmptyExtraSkip]%
\>[0]\AgdaKeyword{variable}\AgdaSpace{}%
\AgdaGeneralizable{A}\AgdaSpace{}%
\AgdaGeneralizable{B}\AgdaSpace{}%
\AgdaGeneralizable{C}\AgdaSpace{}%
\AgdaSymbol{:}\AgdaSpace{}%
\AgdaDatatype{Ty}\<%
\\
%
\\[\AgdaEmptyExtraSkip]%
\>[0]\AgdaKeyword{infixl}\AgdaSpace{}%
\AgdaNumber{5}\AgdaSpace{}%
\AgdaOperator{\AgdaInductiveConstructor{\AgdaUnderscore{}▹\AgdaUnderscore{}}}\<%
\\
\>[0]\AgdaKeyword{data}\AgdaSpace{}%
\AgdaDatatype{Con}\AgdaSpace{}%
\AgdaSymbol{:}\AgdaSpace{}%
\AgdaFunction{Type}\AgdaSpace{}%
\AgdaKeyword{where}\<%
\\
\>[0][@{}l@{\AgdaIndent{0}}]%
\>[2]\AgdaInductiveConstructor{∙}%
\>[6]\AgdaSymbol{:}\AgdaSpace{}%
\AgdaDatatype{Con}\<%
\\
%
\>[2]\AgdaOperator{\AgdaInductiveConstructor{\AgdaUnderscore{}▹\AgdaUnderscore{}}}\AgdaSpace{}%
\AgdaSymbol{:}\AgdaSpace{}%
\AgdaDatatype{Con}\AgdaSpace{}%
\AgdaSymbol{→}\AgdaSpace{}%
\AgdaDatatype{Ty}\AgdaSpace{}%
\AgdaSymbol{→}\AgdaSpace{}%
\AgdaDatatype{Con}\<%
\\
%
\\[\AgdaEmptyExtraSkip]%
\>[0]\AgdaKeyword{variable}\AgdaSpace{}%
\AgdaGeneralizable{Γ}\AgdaSpace{}%
\AgdaGeneralizable{Δ}\AgdaSpace{}%
\AgdaGeneralizable{Θ}\AgdaSpace{}%
\AgdaSymbol{:}\AgdaSpace{}%
\AgdaDatatype{Con}\<%
\\
%
\\[\AgdaEmptyExtraSkip]%
\>[0]\AgdaKeyword{data}\AgdaSpace{}%
\AgdaDatatype{Var}\AgdaSpace{}%
\AgdaSymbol{:}\AgdaSpace{}%
\AgdaDatatype{Con}\AgdaSpace{}%
\AgdaSymbol{→}\AgdaSpace{}%
\AgdaDatatype{Ty}\AgdaSpace{}%
\AgdaSymbol{→}\AgdaSpace{}%
\AgdaFunction{Type}\AgdaSpace{}%
\AgdaKeyword{where}\<%
\\
\>[0][@{}l@{\AgdaIndent{0}}]%
\>[2]\AgdaInductiveConstructor{vz}\AgdaSpace{}%
\AgdaSymbol{:}\AgdaSpace{}%
\AgdaDatatype{Var}\AgdaSpace{}%
\AgdaSymbol{(}\AgdaGeneralizable{Γ}\AgdaSpace{}%
\AgdaOperator{\AgdaInductiveConstructor{▹}}\AgdaSpace{}%
\AgdaGeneralizable{A}\AgdaSymbol{)}\AgdaSpace{}%
\AgdaGeneralizable{A}\<%
\\
%
\>[2]\AgdaInductiveConstructor{vs}\AgdaSpace{}%
\AgdaSymbol{:}\AgdaSpace{}%
\AgdaDatatype{Var}\AgdaSpace{}%
\AgdaGeneralizable{Γ}\AgdaSpace{}%
\AgdaGeneralizable{A}\AgdaSpace{}%
\AgdaSymbol{→}\AgdaSpace{}%
\AgdaDatatype{Var}\AgdaSpace{}%
\AgdaSymbol{(}\AgdaGeneralizable{Γ}\AgdaSpace{}%
\AgdaOperator{\AgdaInductiveConstructor{▹}}\AgdaSpace{}%
\AgdaGeneralizable{B}\AgdaSymbol{)}\AgdaSpace{}%
\AgdaGeneralizable{A}\<%
\\
%
\\[\AgdaEmptyExtraSkip]%
\>[0]\AgdaKeyword{variable}\AgdaSpace{}%
\AgdaGeneralizable{x}\AgdaSpace{}%
\AgdaGeneralizable{y}\AgdaSpace{}%
\AgdaSymbol{:}\AgdaSpace{}%
\AgdaDatatype{Var}\AgdaSpace{}%
\AgdaGeneralizable{Γ}\AgdaSpace{}%
\AgdaGeneralizable{A}\<%
\end{code}

To establish the syntax of higher containers, we begin by reviewing the first-order case. For example, list can be viewed as \texttt{ℕ}-nary coproducts of \texttt{Fin n}-nary (for \texttt{n : ℕ}) products of \texttt{X}. The essence of coproducts and products are captured by shapes and positions correspondingly.

\begin{code}%
\>[0]\AgdaFunction{List}\AgdaSpace{}%
\AgdaSymbol{:}\AgdaSpace{}%
\AgdaFunction{Type}\AgdaSpace{}%
\AgdaSymbol{→}\AgdaSpace{}%
\AgdaFunction{Type}\<%
\\
\>[0]\AgdaFunction{List}\AgdaSpace{}%
\AgdaBound{X}\AgdaSpace{}%
\AgdaSymbol{=}\AgdaSpace{}%
\AgdaRecord{⊤}\AgdaSpace{}%
\AgdaOperator{\AgdaDatatype{⊎}}\AgdaSpace{}%
\AgdaBound{X}\AgdaSpace{}%
\AgdaOperator{\AgdaDatatype{⊎}}\AgdaSpace{}%
\AgdaBound{X}\AgdaSpace{}%
\AgdaOperator{\AgdaFunction{×}}\AgdaSpace{}%
\AgdaBound{X}\AgdaSpace{}%
\AgdaOperator{\AgdaDatatype{⊎}}\AgdaSpace{}%
\AgdaBound{X}\AgdaSpace{}%
\AgdaOperator{\AgdaFunction{×}}\AgdaSpace{}%
\AgdaBound{X}\AgdaSpace{}%
\AgdaOperator{\AgdaFunction{×}}\AgdaSpace{}%
\AgdaBound{X}\AgdaSpace{}%
\AgdaComment{\{-\ ⊎\ ...\ -\}}\<%
\end{code}

Now observe a random second-order function \texttt{H}.

\begin{code}%
\>[0]\AgdaFunction{H}\AgdaSpace{}%
\AgdaSymbol{:}\AgdaSpace{}%
\AgdaSymbol{(}\AgdaFunction{Type}\AgdaSpace{}%
\AgdaSymbol{→}\AgdaSpace{}%
\AgdaFunction{Type}\AgdaSymbol{)}\AgdaSpace{}%
\AgdaSymbol{→}\AgdaSpace{}%
\AgdaFunction{Type}\AgdaSpace{}%
\AgdaSymbol{→}\AgdaSpace{}%
\AgdaFunction{Type}\<%
\\
\>[0]\AgdaFunction{H}\AgdaSpace{}%
\AgdaBound{F}\AgdaSpace{}%
\AgdaBound{X}\AgdaSpace{}%
\AgdaSymbol{=}\AgdaSpace{}%
\AgdaRecord{⊤}\AgdaSpace{}%
\AgdaOperator{\AgdaDatatype{⊎}}\AgdaSpace{}%
\AgdaBound{X}\AgdaSpace{}%
\AgdaOperator{\AgdaFunction{×}}\AgdaSpace{}%
\AgdaBound{X}\AgdaSpace{}%
\AgdaOperator{\AgdaDatatype{⊎}}\AgdaSpace{}%
\AgdaBound{F}\AgdaSpace{}%
\AgdaSymbol{(}\AgdaBound{F}\AgdaSpace{}%
\AgdaBound{X}\AgdaSymbol{)}\<%
\end{code}

Similar to a first-order term, \texttt{H} is also a coproduct of sub-terms, so we need to keep the shape. The difference is that, under each shape we now need to keep track of the positions for both input variables \texttt{X} and \texttt{F}. Notably, the third sub-term \texttt{F (F X)} is defined as nested term applications. It implies that, we need to add recursive structures for each occurrence of term that is not fully saturated.

It leads to the syntax of higher containers. They are much like (normalized) simply typed \lambda-terms closed under products and coproducts. We therefore borrow the syntax of normal forms from previous section.

\begin{code}[hide]%
\>[0]\AgdaKeyword{mutual}\<%
\end{code}

\begin{code}%
\>[0][@{}l@{\AgdaIndent{1}}]%
\>[2]\AgdaKeyword{data}\AgdaSpace{}%
\AgdaDatatype{Nf}\AgdaSpace{}%
\AgdaSymbol{:}\AgdaSpace{}%
\AgdaDatatype{Con}\AgdaSpace{}%
\AgdaSymbol{→}\AgdaSpace{}%
\AgdaDatatype{Ty}\AgdaSpace{}%
\AgdaSymbol{→}\AgdaSpace{}%
\AgdaFunction{Type₁}\AgdaSpace{}%
\AgdaKeyword{where}\<%
\\
\>[2][@{}l@{\AgdaIndent{0}}]%
\>[4]\AgdaInductiveConstructor{lam}\AgdaSpace{}%
\AgdaSymbol{:}\AgdaSpace{}%
\AgdaDatatype{Nf}\AgdaSpace{}%
\AgdaSymbol{(}\AgdaGeneralizable{Γ}\AgdaSpace{}%
\AgdaOperator{\AgdaInductiveConstructor{▹}}\AgdaSpace{}%
\AgdaGeneralizable{A}\AgdaSymbol{)}\AgdaSpace{}%
\AgdaGeneralizable{B}\AgdaSpace{}%
\AgdaSymbol{→}\AgdaSpace{}%
\AgdaDatatype{Nf}\AgdaSpace{}%
\AgdaGeneralizable{Γ}\AgdaSpace{}%
\AgdaSymbol{(}\AgdaGeneralizable{A}\AgdaSpace{}%
\AgdaOperator{\AgdaInductiveConstructor{⇒}}\AgdaSpace{}%
\AgdaGeneralizable{B}\AgdaSymbol{)}\<%
\\
%
\>[4]\AgdaInductiveConstructor{ne}%
\>[8]\AgdaSymbol{:}\AgdaSpace{}%
\AgdaRecord{Ne}\AgdaSpace{}%
\AgdaGeneralizable{Γ}\AgdaSpace{}%
\AgdaInductiveConstructor{*}\AgdaSpace{}%
\AgdaSymbol{→}\AgdaSpace{}%
\AgdaDatatype{Nf}\AgdaSpace{}%
\AgdaGeneralizable{Γ}\AgdaSpace{}%
\AgdaInductiveConstructor{*}\<%
\end{code}

The purpose of \texttt{Nf} is just to push types into the context by \texttt{lam}, or to go to container definition by \texttt{ne}.

\begin{code}%
%
\>[2]\AgdaKeyword{record}\AgdaSpace{}%
\AgdaRecord{Ne}\AgdaSpace{}%
\AgdaSymbol{(}\AgdaBound{Γ}\AgdaSpace{}%
\AgdaSymbol{:}\AgdaSpace{}%
\AgdaDatatype{Con}\AgdaSymbol{)}\AgdaSpace{}%
\AgdaSymbol{(}\AgdaBound{B}\AgdaSpace{}%
\AgdaSymbol{:}\AgdaSpace{}%
\AgdaDatatype{Ty}\AgdaSymbol{)}\AgdaSpace{}%
\AgdaSymbol{:}\AgdaSpace{}%
\AgdaFunction{Type₁}\AgdaSpace{}%
\AgdaKeyword{where}\<%
\\
\>[2][@{}l@{\AgdaIndent{0}}]%
\>[4]\AgdaKeyword{constructor}\AgdaSpace{}%
\AgdaOperator{\AgdaInductiveConstructor{\AgdaUnderscore{}◃\AgdaUnderscore{}◃\AgdaUnderscore{}}}\<%
\\
%
\>[4]\AgdaKeyword{inductive}\<%
\\
%
\>[4]\AgdaKeyword{field}\<%
\\
\>[4][@{}l@{\AgdaIndent{0}}]%
\>[6]\AgdaField{S}\AgdaSpace{}%
\AgdaSymbol{:}\AgdaSpace{}%
\AgdaFunction{Type}\<%
\\
%
\>[6]\AgdaField{P}\AgdaSpace{}%
\AgdaSymbol{:}\AgdaSpace{}%
\AgdaDatatype{Var}\AgdaSpace{}%
\AgdaBound{Γ}\AgdaSpace{}%
\AgdaGeneralizable{A}\AgdaSpace{}%
\AgdaSymbol{→}\AgdaSpace{}%
\AgdaField{S}\AgdaSpace{}%
\AgdaSymbol{→}\AgdaSpace{}%
\AgdaFunction{Type}\<%
\\
%
\>[6]\AgdaField{R}\AgdaSpace{}%
\AgdaSymbol{:}\AgdaSpace{}%
\AgdaSymbol{(}\AgdaBound{x}\AgdaSpace{}%
\AgdaSymbol{:}\AgdaSpace{}%
\AgdaDatatype{Var}\AgdaSpace{}%
\AgdaBound{Γ}\AgdaSpace{}%
\AgdaGeneralizable{A}\AgdaSymbol{)}\AgdaSpace{}%
\AgdaSymbol{(}\AgdaBound{s}\AgdaSpace{}%
\AgdaSymbol{:}\AgdaSpace{}%
\AgdaField{S}\AgdaSymbol{)}\AgdaSpace{}%
\AgdaSymbol{(}\AgdaBound{p}\AgdaSpace{}%
\AgdaSymbol{:}\AgdaSpace{}%
\AgdaField{P}\AgdaSpace{}%
\AgdaBound{x}\AgdaSpace{}%
\AgdaBound{s}\AgdaSymbol{)}\AgdaSpace{}%
\AgdaSymbol{→}\AgdaSpace{}%
\AgdaDatatype{Sp}\AgdaSpace{}%
\AgdaBound{Γ}\AgdaSpace{}%
\AgdaGeneralizable{A}\AgdaSpace{}%
\AgdaBound{B}\<%
\end{code}

\texttt{Ne} is the core definition of containers, which is defined as a shape, and for each variable in the context and each shape a position, and for each position of variable a spine. Where a spine is list of normal forms applied to variable to saturate a neutral term.

\begin{code}%
%
\>[2]\AgdaKeyword{data}\AgdaSpace{}%
\AgdaDatatype{Sp}\AgdaSpace{}%
\AgdaSymbol{:}\AgdaSpace{}%
\AgdaDatatype{Con}\AgdaSpace{}%
\AgdaSymbol{→}\AgdaSpace{}%
\AgdaDatatype{Ty}\AgdaSpace{}%
\AgdaSymbol{→}\AgdaSpace{}%
\AgdaDatatype{Ty}\AgdaSpace{}%
\AgdaSymbol{→}\AgdaSpace{}%
\AgdaFunction{Type₁}\AgdaSpace{}%
\AgdaKeyword{where}\<%
\\
\>[2][@{}l@{\AgdaIndent{0}}]%
\>[4]\AgdaInductiveConstructor{ε}%
\>[8]\AgdaSymbol{:}\AgdaSpace{}%
\AgdaDatatype{Sp}\AgdaSpace{}%
\AgdaGeneralizable{Γ}\AgdaSpace{}%
\AgdaGeneralizable{A}\AgdaSpace{}%
\AgdaGeneralizable{A}\<%
\\
%
\>[4]\AgdaOperator{\AgdaInductiveConstructor{\AgdaUnderscore{},\AgdaUnderscore{}}}\AgdaSpace{}%
\AgdaSymbol{:}\AgdaSpace{}%
\AgdaDatatype{Nf}\AgdaSpace{}%
\AgdaGeneralizable{Γ}\AgdaSpace{}%
\AgdaGeneralizable{A}\AgdaSpace{}%
\AgdaSymbol{→}\AgdaSpace{}%
\AgdaDatatype{Sp}\AgdaSpace{}%
\AgdaGeneralizable{Γ}\AgdaSpace{}%
\AgdaGeneralizable{B}\AgdaSpace{}%
\AgdaGeneralizable{C}\AgdaSpace{}%
\AgdaSymbol{→}\AgdaSpace{}%
\AgdaDatatype{Sp}\AgdaSpace{}%
\AgdaGeneralizable{Γ}\AgdaSpace{}%
\AgdaSymbol{(}\AgdaGeneralizable{A}\AgdaSpace{}%
\AgdaOperator{\AgdaInductiveConstructor{⇒}}\AgdaSpace{}%
\AgdaGeneralizable{B}\AgdaSymbol{)}\AgdaSpace{}%
\AgdaGeneralizable{C}\<%
\end{code}

\begin{code}[hide]%
%
\>[2]\AgdaKeyword{variable}\AgdaSpace{}%
\AgdaGeneralizable{t}\AgdaSpace{}%
\AgdaGeneralizable{u}\AgdaSpace{}%
\AgdaGeneralizable{v}\AgdaSpace{}%
\AgdaSymbol{:}\AgdaSpace{}%
\AgdaDatatype{Nf}\AgdaSpace{}%
\AgdaGeneralizable{Γ}\AgdaSpace{}%
\AgdaGeneralizable{A}\<%
\\
%
\>[2]\AgdaKeyword{variable}\AgdaSpace{}%
\AgdaGeneralizable{spr}\AgdaSpace{}%
\AgdaGeneralizable{tql}\AgdaSpace{}%
\AgdaSymbol{:}\AgdaSpace{}%
\AgdaRecord{Ne}\AgdaSpace{}%
\AgdaGeneralizable{Γ}\AgdaSpace{}%
\AgdaGeneralizable{A}\<%
\\
%
\>[2]\AgdaKeyword{variable}\AgdaSpace{}%
\AgdaGeneralizable{ts}\AgdaSpace{}%
\AgdaGeneralizable{us}\AgdaSpace{}%
\AgdaSymbol{:}\AgdaSpace{}%
\AgdaDatatype{Sp}\AgdaSpace{}%
\AgdaGeneralizable{Γ}\AgdaSpace{}%
\AgdaGeneralizable{A}\AgdaSpace{}%
\AgdaGeneralizable{B}\<%
\end{code}

Finally, higher-order containers are defined as closed (with empty context) normal forms. 

\begin{code}%
\>[0]\AgdaFunction{HCont}\AgdaSpace{}%
\AgdaSymbol{:}\AgdaSpace{}%
\AgdaDatatype{Ty}\AgdaSpace{}%
\AgdaSymbol{→}\AgdaSpace{}%
\AgdaFunction{Type₁}\<%
\\
\>[0]\AgdaFunction{HCont}\AgdaSpace{}%
\AgdaBound{A}\AgdaSpace{}%
\AgdaSymbol{=}\AgdaSpace{}%
\AgdaDatatype{Nf}\AgdaSpace{}%
\AgdaInductiveConstructor{∙}\AgdaSpace{}%
\AgdaBound{A}\<%
\end{code}

From this point onward, we focus on normal forms in general, as higher-order containers are merely special cases of the former.

\subsubsection*{Normalization}

\begin{code}[hide]%
\>[0]\AgdaFunction{app}\AgdaSpace{}%
\AgdaSymbol{:}\AgdaSpace{}%
\AgdaDatatype{Nf}\AgdaSpace{}%
\AgdaGeneralizable{Γ}\AgdaSpace{}%
\AgdaSymbol{(}\AgdaGeneralizable{A}\AgdaSpace{}%
\AgdaOperator{\AgdaInductiveConstructor{⇒}}\AgdaSpace{}%
\AgdaGeneralizable{B}\AgdaSymbol{)}\AgdaSpace{}%
\AgdaSymbol{→}\AgdaSpace{}%
\AgdaDatatype{Nf}\AgdaSpace{}%
\AgdaSymbol{(}\AgdaGeneralizable{Γ}\AgdaSpace{}%
\AgdaOperator{\AgdaInductiveConstructor{▹}}\AgdaSpace{}%
\AgdaGeneralizable{A}\AgdaSymbol{)}\AgdaSpace{}%
\AgdaGeneralizable{B}\<%
\\
\>[0]\AgdaFunction{app}\AgdaSpace{}%
\AgdaSymbol{(}\AgdaInductiveConstructor{lam}\AgdaSpace{}%
\AgdaBound{t}\AgdaSymbol{)}\AgdaSpace{}%
\AgdaSymbol{=}\AgdaSpace{}%
\AgdaBound{t}\<%
\\
%
\\[\AgdaEmptyExtraSkip]%
\>[0]\AgdaFunction{en}\AgdaSpace{}%
\AgdaSymbol{:}\AgdaSpace{}%
\AgdaDatatype{Nf}\AgdaSpace{}%
\AgdaGeneralizable{Γ}\AgdaSpace{}%
\AgdaInductiveConstructor{*}\AgdaSpace{}%
\AgdaSymbol{→}\AgdaSpace{}%
\AgdaRecord{Ne}\AgdaSpace{}%
\AgdaGeneralizable{Γ}\AgdaSpace{}%
\AgdaInductiveConstructor{*}\<%
\\
\>[0]\AgdaFunction{en}\AgdaSpace{}%
\AgdaSymbol{(}\AgdaInductiveConstructor{ne}\AgdaSpace{}%
\AgdaBound{spr}\AgdaSymbol{)}\AgdaSpace{}%
\AgdaSymbol{=}\AgdaSpace{}%
\AgdaBound{spr}\<%
\\
%
\\[\AgdaEmptyExtraSkip]%
\>[0]\AgdaOperator{\AgdaFunction{\AgdaUnderscore{}-\AgdaUnderscore{}}}\AgdaSpace{}%
\AgdaSymbol{:}\AgdaSpace{}%
\AgdaSymbol{(}\AgdaBound{Γ}\AgdaSpace{}%
\AgdaSymbol{:}\AgdaSpace{}%
\AgdaDatatype{Con}\AgdaSymbol{)}\AgdaSpace{}%
\AgdaSymbol{→}\AgdaSpace{}%
\AgdaDatatype{Var}\AgdaSpace{}%
\AgdaBound{Γ}\AgdaSpace{}%
\AgdaGeneralizable{A}\AgdaSpace{}%
\AgdaSymbol{→}\AgdaSpace{}%
\AgdaDatatype{Con}\<%
\\
\>[0]\AgdaInductiveConstructor{∙}\AgdaSpace{}%
\AgdaOperator{\AgdaFunction{-}}\AgdaSpace{}%
\AgdaSymbol{()}\<%
\\
\>[0]\AgdaSymbol{(}\AgdaBound{Γ}\AgdaSpace{}%
\AgdaOperator{\AgdaInductiveConstructor{▹}}\AgdaSpace{}%
\AgdaBound{A}\AgdaSymbol{)}\AgdaSpace{}%
\AgdaOperator{\AgdaFunction{-}}\AgdaSpace{}%
\AgdaInductiveConstructor{vz}\AgdaSpace{}%
\AgdaSymbol{=}\AgdaSpace{}%
\AgdaBound{Γ}\<%
\\
\>[0]\AgdaSymbol{(}\AgdaBound{Γ}\AgdaSpace{}%
\AgdaOperator{\AgdaInductiveConstructor{▹}}\AgdaSpace{}%
\AgdaBound{A}\AgdaSymbol{)}\AgdaSpace{}%
\AgdaOperator{\AgdaFunction{-}}\AgdaSpace{}%
\AgdaSymbol{(}\AgdaInductiveConstructor{vs}\AgdaSpace{}%
\AgdaBound{x}\AgdaSymbol{)}\AgdaSpace{}%
\AgdaSymbol{=}\AgdaSpace{}%
\AgdaSymbol{(}\AgdaBound{Γ}\AgdaSpace{}%
\AgdaOperator{\AgdaFunction{-}}\AgdaSpace{}%
\AgdaBound{x}\AgdaSymbol{)}\AgdaSpace{}%
\AgdaOperator{\AgdaInductiveConstructor{▹}}\AgdaSpace{}%
\AgdaBound{A}\<%
\\
%
\\[\AgdaEmptyExtraSkip]%
\>[0]\AgdaFunction{wkVar}\AgdaSpace{}%
\AgdaSymbol{:}\AgdaSpace{}%
\AgdaSymbol{(}\AgdaBound{x}\AgdaSpace{}%
\AgdaSymbol{:}\AgdaSpace{}%
\AgdaDatatype{Var}\AgdaSpace{}%
\AgdaGeneralizable{Γ}\AgdaSpace{}%
\AgdaGeneralizable{A}\AgdaSymbol{)}\AgdaSpace{}%
\AgdaSymbol{→}\AgdaSpace{}%
\AgdaDatatype{Var}\AgdaSpace{}%
\AgdaSymbol{(}\AgdaGeneralizable{Γ}\AgdaSpace{}%
\AgdaOperator{\AgdaFunction{-}}\AgdaSpace{}%
\AgdaBound{x}\AgdaSymbol{)}\AgdaSpace{}%
\AgdaGeneralizable{B}\AgdaSpace{}%
\AgdaSymbol{→}\AgdaSpace{}%
\AgdaDatatype{Var}\AgdaSpace{}%
\AgdaGeneralizable{Γ}\AgdaSpace{}%
\AgdaGeneralizable{B}\<%
\\
\>[0]\AgdaFunction{wkVar}\AgdaSpace{}%
\AgdaInductiveConstructor{vz}\AgdaSpace{}%
\AgdaBound{y}\AgdaSpace{}%
\AgdaSymbol{=}\AgdaSpace{}%
\AgdaInductiveConstructor{vs}\AgdaSpace{}%
\AgdaBound{y}\<%
\\
\>[0]\AgdaFunction{wkVar}\AgdaSpace{}%
\AgdaSymbol{(}\AgdaInductiveConstructor{vs}\AgdaSpace{}%
\AgdaBound{x}\AgdaSymbol{)}\AgdaSpace{}%
\AgdaInductiveConstructor{vz}\AgdaSpace{}%
\AgdaSymbol{=}\AgdaSpace{}%
\AgdaInductiveConstructor{vz}\<%
\\
\>[0]\AgdaFunction{wkVar}\AgdaSpace{}%
\AgdaSymbol{(}\AgdaInductiveConstructor{vs}\AgdaSpace{}%
\AgdaBound{x}\AgdaSymbol{)}\AgdaSpace{}%
\AgdaSymbol{(}\AgdaInductiveConstructor{vs}\AgdaSpace{}%
\AgdaBound{y}\AgdaSymbol{)}\AgdaSpace{}%
\AgdaSymbol{=}\AgdaSpace{}%
\AgdaInductiveConstructor{vs}\AgdaSpace{}%
\AgdaSymbol{(}\AgdaFunction{wkVar}\AgdaSpace{}%
\AgdaBound{x}\AgdaSpace{}%
\AgdaBound{y}\AgdaSymbol{)}\<%
\\
%
\\[\AgdaEmptyExtraSkip]%
\>[0]\AgdaKeyword{data}\AgdaSpace{}%
\AgdaDatatype{EqVar}\AgdaSpace{}%
\AgdaSymbol{:}\AgdaSpace{}%
\AgdaDatatype{Var}\AgdaSpace{}%
\AgdaGeneralizable{Γ}\AgdaSpace{}%
\AgdaGeneralizable{A}\AgdaSpace{}%
\AgdaSymbol{→}\AgdaSpace{}%
\AgdaDatatype{Var}\AgdaSpace{}%
\AgdaGeneralizable{Γ}\AgdaSpace{}%
\AgdaGeneralizable{B}\AgdaSpace{}%
\AgdaSymbol{→}\AgdaSpace{}%
\AgdaFunction{Type}\AgdaSpace{}%
\AgdaKeyword{where}\<%
\\
\>[0][@{}l@{\AgdaIndent{0}}]%
\>[2]\AgdaInductiveConstructor{same}\AgdaSpace{}%
\AgdaSymbol{:}\AgdaSpace{}%
\AgdaDatatype{EqVar}\AgdaSpace{}%
\AgdaGeneralizable{x}\AgdaSpace{}%
\AgdaGeneralizable{x}\<%
\\
%
\>[2]\AgdaInductiveConstructor{diff}\AgdaSpace{}%
\AgdaSymbol{:}\AgdaSpace{}%
\AgdaSymbol{(}\AgdaBound{x}\AgdaSpace{}%
\AgdaSymbol{:}\AgdaSpace{}%
\AgdaDatatype{Var}\AgdaSpace{}%
\AgdaGeneralizable{Γ}\AgdaSpace{}%
\AgdaGeneralizable{A}\AgdaSymbol{)}\AgdaSpace{}%
\AgdaSymbol{(}\AgdaBound{y}\AgdaSpace{}%
\AgdaSymbol{:}\AgdaSpace{}%
\AgdaDatatype{Var}\AgdaSpace{}%
\AgdaSymbol{(}\AgdaGeneralizable{Γ}\AgdaSpace{}%
\AgdaOperator{\AgdaFunction{-}}\AgdaSpace{}%
\AgdaBound{x}\AgdaSymbol{)}\AgdaSpace{}%
\AgdaGeneralizable{B}\AgdaSymbol{)}\AgdaSpace{}%
\AgdaSymbol{→}\AgdaSpace{}%
\AgdaDatatype{EqVar}\AgdaSpace{}%
\AgdaBound{x}\AgdaSpace{}%
\AgdaSymbol{(}\AgdaFunction{wkVar}\AgdaSpace{}%
\AgdaBound{x}\AgdaSpace{}%
\AgdaBound{y}\AgdaSymbol{)}\<%
\\
%
\\[\AgdaEmptyExtraSkip]%
\>[0]\AgdaFunction{eq}\AgdaSpace{}%
\AgdaSymbol{:}\AgdaSpace{}%
\AgdaSymbol{(}\AgdaBound{x}\AgdaSpace{}%
\AgdaSymbol{:}\AgdaSpace{}%
\AgdaDatatype{Var}\AgdaSpace{}%
\AgdaGeneralizable{Γ}\AgdaSpace{}%
\AgdaGeneralizable{A}\AgdaSymbol{)}\AgdaSpace{}%
\AgdaSymbol{(}\AgdaBound{y}\AgdaSpace{}%
\AgdaSymbol{:}\AgdaSpace{}%
\AgdaDatatype{Var}\AgdaSpace{}%
\AgdaGeneralizable{Γ}\AgdaSpace{}%
\AgdaGeneralizable{B}\AgdaSymbol{)}\AgdaSpace{}%
\AgdaSymbol{→}\AgdaSpace{}%
\AgdaDatatype{EqVar}\AgdaSpace{}%
\AgdaBound{x}\AgdaSpace{}%
\AgdaBound{y}\<%
\\
\>[0]\AgdaFunction{eq}\AgdaSpace{}%
\AgdaInductiveConstructor{vz}\AgdaSpace{}%
\AgdaInductiveConstructor{vz}\AgdaSpace{}%
\AgdaSymbol{=}\AgdaSpace{}%
\AgdaInductiveConstructor{same}\<%
\\
\>[0]\AgdaFunction{eq}\AgdaSpace{}%
\AgdaInductiveConstructor{vz}\AgdaSpace{}%
\AgdaSymbol{(}\AgdaInductiveConstructor{vs}\AgdaSpace{}%
\AgdaBound{y}\AgdaSymbol{)}\AgdaSpace{}%
\AgdaSymbol{=}\AgdaSpace{}%
\AgdaInductiveConstructor{diff}\AgdaSpace{}%
\AgdaInductiveConstructor{vz}\AgdaSpace{}%
\AgdaBound{y}\<%
\\
\>[0]\AgdaFunction{eq}\AgdaSpace{}%
\AgdaSymbol{(}\AgdaInductiveConstructor{vs}\AgdaSpace{}%
\AgdaBound{x}\AgdaSymbol{)}\AgdaSpace{}%
\AgdaInductiveConstructor{vz}\AgdaSpace{}%
\AgdaSymbol{=}\AgdaSpace{}%
\AgdaInductiveConstructor{diff}\AgdaSpace{}%
\AgdaSymbol{(}\AgdaInductiveConstructor{vs}\AgdaSpace{}%
\AgdaBound{x}\AgdaSymbol{)}\AgdaSpace{}%
\AgdaInductiveConstructor{vz}\<%
\\
\>[0]\AgdaFunction{eq}\AgdaSpace{}%
\AgdaSymbol{(}\AgdaInductiveConstructor{vs}\AgdaSpace{}%
\AgdaBound{x}\AgdaSymbol{)}\AgdaSpace{}%
\AgdaSymbol{(}\AgdaInductiveConstructor{vs}\AgdaSpace{}%
\AgdaBound{y}\AgdaSymbol{)}\AgdaSpace{}%
\AgdaKeyword{with}\AgdaSpace{}%
\AgdaFunction{eq}\AgdaSpace{}%
\AgdaBound{x}\AgdaSpace{}%
\AgdaBound{y}\<%
\\
\>[0]\AgdaFunction{eq}\AgdaSpace{}%
\AgdaSymbol{(}\AgdaInductiveConstructor{vs}\AgdaSpace{}%
\AgdaBound{x}\AgdaSymbol{)}\AgdaSpace{}%
\AgdaSymbol{(}\AgdaInductiveConstructor{vs}\AgdaSpace{}%
\AgdaDottedPattern{\AgdaSymbol{.}}\AgdaDottedPattern{\AgdaBound{x}}\AgdaSymbol{)}%
\>[29]\AgdaSymbol{|}\AgdaSpace{}%
\AgdaInductiveConstructor{same}\AgdaSpace{}%
\AgdaSymbol{=}\AgdaSpace{}%
\AgdaInductiveConstructor{same}\<%
\\
\>[0]\AgdaFunction{eq}\AgdaSpace{}%
\AgdaSymbol{(}\AgdaInductiveConstructor{vs}\AgdaSpace{}%
\AgdaBound{x}\AgdaSymbol{)}\AgdaSpace{}%
\AgdaSymbol{(}\AgdaInductiveConstructor{vs}\AgdaSpace{}%
\AgdaDottedPattern{\AgdaSymbol{.(}}\AgdaDottedPattern{\AgdaFunction{wkVar}}\AgdaSpace{}%
\AgdaDottedPattern{\AgdaBound{x}}\AgdaSpace{}%
\AgdaDottedPattern{\AgdaBound{y'}}\AgdaDottedPattern{\AgdaSymbol{)}}\AgdaSymbol{)}\AgdaSpace{}%
\AgdaSymbol{|}\AgdaSpace{}%
\AgdaInductiveConstructor{diff}\AgdaSpace{}%
\AgdaDottedPattern{\AgdaSymbol{.}}\AgdaDottedPattern{\AgdaBound{x}}\AgdaSpace{}%
\AgdaBound{y'}\AgdaSpace{}%
\AgdaSymbol{=}\AgdaSpace{}%
\AgdaInductiveConstructor{diff}\AgdaSpace{}%
\AgdaSymbol{(}\AgdaInductiveConstructor{vs}\AgdaSpace{}%
\AgdaBound{x}\AgdaSymbol{)}\AgdaSpace{}%
\AgdaSymbol{(}\AgdaInductiveConstructor{vs}\AgdaSpace{}%
\AgdaBound{y'}\AgdaSymbol{)}\<%
\\
%
\\[\AgdaEmptyExtraSkip]%
\>[0]\AgdaKeyword{mutual}\<%
\\
\>[0][@{}l@{\AgdaIndent{0}}]%
\>[2]\AgdaFunction{wkNf}\AgdaSpace{}%
\AgdaSymbol{:}\AgdaSpace{}%
\AgdaSymbol{(}\AgdaBound{x}\AgdaSpace{}%
\AgdaSymbol{:}\AgdaSpace{}%
\AgdaDatatype{Var}\AgdaSpace{}%
\AgdaGeneralizable{Γ}\AgdaSpace{}%
\AgdaGeneralizable{A}\AgdaSymbol{)}\AgdaSpace{}%
\AgdaSymbol{→}\AgdaSpace{}%
\AgdaDatatype{Nf}\AgdaSpace{}%
\AgdaSymbol{(}\AgdaGeneralizable{Γ}\AgdaSpace{}%
\AgdaOperator{\AgdaFunction{-}}\AgdaSpace{}%
\AgdaBound{x}\AgdaSymbol{)}\AgdaSpace{}%
\AgdaGeneralizable{B}\AgdaSpace{}%
\AgdaSymbol{→}\AgdaSpace{}%
\AgdaDatatype{Nf}\AgdaSpace{}%
\AgdaGeneralizable{Γ}\AgdaSpace{}%
\AgdaGeneralizable{B}\<%
\\
%
\>[2]\AgdaFunction{wkNf}\AgdaSpace{}%
\AgdaBound{x}\AgdaSpace{}%
\AgdaSymbol{(}\AgdaInductiveConstructor{lam}\AgdaSpace{}%
\AgdaBound{t}\AgdaSymbol{)}\AgdaSpace{}%
\AgdaSymbol{=}\AgdaSpace{}%
\AgdaInductiveConstructor{lam}\AgdaSpace{}%
\AgdaSymbol{(}\AgdaFunction{wkNf}\AgdaSpace{}%
\AgdaSymbol{(}\AgdaInductiveConstructor{vs}\AgdaSpace{}%
\AgdaBound{x}\AgdaSymbol{)}\AgdaSpace{}%
\AgdaBound{t}\AgdaSymbol{)}\<%
\\
%
\>[2]\AgdaFunction{wkNf}\AgdaSpace{}%
\AgdaBound{x}\AgdaSpace{}%
\AgdaSymbol{(}\AgdaInductiveConstructor{ne}\AgdaSpace{}%
\AgdaBound{spr}\AgdaSymbol{)}\AgdaSpace{}%
\AgdaSymbol{=}\AgdaSpace{}%
\AgdaInductiveConstructor{ne}\AgdaSpace{}%
\AgdaSymbol{(}\AgdaFunction{wkNe}\AgdaSpace{}%
\AgdaBound{x}\AgdaSpace{}%
\AgdaBound{spr}\AgdaSymbol{)}\<%
\\
%
\\[\AgdaEmptyExtraSkip]%
%
\>[2]\AgdaFunction{wkNe}\AgdaSpace{}%
\AgdaSymbol{:}\AgdaSpace{}%
\AgdaSymbol{(}\AgdaBound{x}\AgdaSpace{}%
\AgdaSymbol{:}\AgdaSpace{}%
\AgdaDatatype{Var}\AgdaSpace{}%
\AgdaGeneralizable{Γ}\AgdaSpace{}%
\AgdaGeneralizable{A}\AgdaSymbol{)}\AgdaSpace{}%
\AgdaSymbol{→}\AgdaSpace{}%
\AgdaRecord{Ne}\AgdaSpace{}%
\AgdaSymbol{(}\AgdaGeneralizable{Γ}\AgdaSpace{}%
\AgdaOperator{\AgdaFunction{-}}\AgdaSpace{}%
\AgdaBound{x}\AgdaSymbol{)}\AgdaSpace{}%
\AgdaGeneralizable{B}\AgdaSpace{}%
\AgdaSymbol{→}\AgdaSpace{}%
\AgdaRecord{Ne}\AgdaSpace{}%
\AgdaGeneralizable{Γ}\AgdaSpace{}%
\AgdaGeneralizable{B}\<%
\\
%
\>[2]\AgdaFunction{wkNe}\AgdaSpace{}%
\AgdaSymbol{\{}\AgdaBound{Γ}\AgdaSymbol{\}}\AgdaSpace{}%
\AgdaSymbol{\{}\AgdaBound{A}\AgdaSymbol{\}}\AgdaSpace{}%
\AgdaSymbol{\{}\AgdaBound{C}\AgdaSymbol{\}}\AgdaSpace{}%
\AgdaBound{x}\AgdaSpace{}%
\AgdaSymbol{(}\AgdaBound{S}\AgdaSpace{}%
\AgdaOperator{\AgdaInductiveConstructor{◃}}\AgdaSpace{}%
\AgdaBound{P}\AgdaSpace{}%
\AgdaOperator{\AgdaInductiveConstructor{◃}}\AgdaSpace{}%
\AgdaBound{R}\AgdaSymbol{)}\AgdaSpace{}%
\AgdaSymbol{=}\AgdaSpace{}%
\AgdaBound{S}\AgdaSpace{}%
\AgdaOperator{\AgdaInductiveConstructor{◃}}\AgdaSpace{}%
\AgdaFunction{P'}\AgdaSpace{}%
\AgdaOperator{\AgdaInductiveConstructor{◃}}\AgdaSpace{}%
\AgdaFunction{R'}\<%
\\
\>[2][@{}l@{\AgdaIndent{0}}]%
\>[4]\AgdaKeyword{where}\<%
\\
%
\>[4]\AgdaFunction{P'}\AgdaSpace{}%
\AgdaSymbol{:}\AgdaSpace{}%
\AgdaDatatype{Var}\AgdaSpace{}%
\AgdaBound{Γ}\AgdaSpace{}%
\AgdaGeneralizable{B}\AgdaSpace{}%
\AgdaSymbol{→}\AgdaSpace{}%
\AgdaBound{S}\AgdaSpace{}%
\AgdaSymbol{→}\AgdaSpace{}%
\AgdaFunction{Type}\<%
\\
%
\>[4]\AgdaFunction{P'}\AgdaSpace{}%
\AgdaBound{y}%
\>[10]\AgdaBound{s}\AgdaSpace{}%
\AgdaKeyword{with}\AgdaSpace{}%
\AgdaFunction{eq}\AgdaSpace{}%
\AgdaBound{x}\AgdaSpace{}%
\AgdaBound{y}\<%
\\
%
\>[4]\AgdaFunction{P'}\AgdaSpace{}%
\AgdaDottedPattern{\AgdaSymbol{.}}\AgdaDottedPattern{\AgdaBound{x}}\AgdaSpace{}%
\AgdaBound{s}\AgdaSpace{}%
\AgdaSymbol{|}\AgdaSpace{}%
\AgdaInductiveConstructor{same}\AgdaSpace{}%
\AgdaSymbol{=}\AgdaSpace{}%
\AgdaFunction{⊥}\<%
\\
%
\>[4]\AgdaFunction{P'}\AgdaSpace{}%
\AgdaBound{y}%
\>[10]\AgdaBound{s}\AgdaSpace{}%
\AgdaSymbol{|}\AgdaSpace{}%
\AgdaInductiveConstructor{diff}\AgdaSpace{}%
\AgdaDottedPattern{\AgdaSymbol{.}}\AgdaDottedPattern{\AgdaBound{x}}\AgdaSpace{}%
\AgdaBound{y'}\AgdaSpace{}%
\AgdaSymbol{=}\AgdaSpace{}%
\AgdaBound{P}\AgdaSpace{}%
\AgdaBound{y'}\AgdaSpace{}%
\AgdaBound{s}\<%
\\
%
\\[\AgdaEmptyExtraSkip]%
%
\>[4]\AgdaFunction{R'}\AgdaSpace{}%
\AgdaSymbol{:}\AgdaSpace{}%
\AgdaSymbol{(}\AgdaBound{y}\AgdaSpace{}%
\AgdaSymbol{:}\AgdaSpace{}%
\AgdaDatatype{Var}\AgdaSpace{}%
\AgdaBound{Γ}\AgdaSpace{}%
\AgdaGeneralizable{B}\AgdaSymbol{)}\AgdaSpace{}%
\AgdaSymbol{(}\AgdaBound{s}\AgdaSpace{}%
\AgdaSymbol{:}\AgdaSpace{}%
\AgdaBound{S}\AgdaSymbol{)}\AgdaSpace{}%
\AgdaSymbol{→}\AgdaSpace{}%
\AgdaFunction{P'}\AgdaSpace{}%
\AgdaBound{y}\AgdaSpace{}%
\AgdaBound{s}\AgdaSpace{}%
\AgdaSymbol{→}\AgdaSpace{}%
\AgdaDatatype{Sp}\AgdaSpace{}%
\AgdaBound{Γ}\AgdaSpace{}%
\AgdaGeneralizable{B}\AgdaSpace{}%
\AgdaBound{C}\<%
\\
%
\>[4]\AgdaFunction{R'}\AgdaSpace{}%
\AgdaBound{y}\AgdaSpace{}%
\AgdaBound{s}\AgdaSpace{}%
\AgdaBound{p}\AgdaSpace{}%
\AgdaKeyword{with}\AgdaSpace{}%
\AgdaFunction{eq}\AgdaSpace{}%
\AgdaBound{x}\AgdaSpace{}%
\AgdaBound{y}\<%
\\
%
\>[4]\AgdaFunction{R'}\AgdaSpace{}%
\AgdaBound{y}\AgdaSpace{}%
\AgdaBound{s}\AgdaSpace{}%
\AgdaBound{p}\AgdaSpace{}%
\AgdaSymbol{|}\AgdaSpace{}%
\AgdaInductiveConstructor{diff}\AgdaSpace{}%
\AgdaDottedPattern{\AgdaSymbol{.}}\AgdaDottedPattern{\AgdaBound{x}}\AgdaSpace{}%
\AgdaBound{y'}\AgdaSpace{}%
\AgdaSymbol{=}\AgdaSpace{}%
\AgdaFunction{wkSp}\AgdaSpace{}%
\AgdaBound{x}\AgdaSpace{}%
\AgdaSymbol{(}\AgdaBound{R}\AgdaSpace{}%
\AgdaBound{y'}\AgdaSpace{}%
\AgdaBound{s}\AgdaSpace{}%
\AgdaBound{p}\AgdaSymbol{)}\<%
\\
%
\\[\AgdaEmptyExtraSkip]%
%
\>[2]\AgdaFunction{wkSp}\AgdaSpace{}%
\AgdaSymbol{:}\AgdaSpace{}%
\AgdaSymbol{(}\AgdaBound{x}\AgdaSpace{}%
\AgdaSymbol{:}\AgdaSpace{}%
\AgdaDatatype{Var}\AgdaSpace{}%
\AgdaGeneralizable{Γ}\AgdaSpace{}%
\AgdaGeneralizable{A}\AgdaSymbol{)}\AgdaSpace{}%
\AgdaSymbol{→}\AgdaSpace{}%
\AgdaDatatype{Sp}\AgdaSpace{}%
\AgdaSymbol{(}\AgdaGeneralizable{Γ}\AgdaSpace{}%
\AgdaOperator{\AgdaFunction{-}}\AgdaSpace{}%
\AgdaBound{x}\AgdaSymbol{)}\AgdaSpace{}%
\AgdaGeneralizable{B}\AgdaSpace{}%
\AgdaGeneralizable{C}\AgdaSpace{}%
\AgdaSymbol{→}\AgdaSpace{}%
\AgdaDatatype{Sp}\AgdaSpace{}%
\AgdaGeneralizable{Γ}\AgdaSpace{}%
\AgdaGeneralizable{B}\AgdaSpace{}%
\AgdaGeneralizable{C}\<%
\\
%
\>[2]\AgdaFunction{wkSp}\AgdaSpace{}%
\AgdaBound{x}\AgdaSpace{}%
\AgdaInductiveConstructor{ε}\AgdaSpace{}%
\AgdaSymbol{=}\AgdaSpace{}%
\AgdaInductiveConstructor{ε}\<%
\\
%
\>[2]\AgdaFunction{wkSp}\AgdaSpace{}%
\AgdaBound{x}\AgdaSpace{}%
\AgdaSymbol{(}\AgdaBound{t}\AgdaSpace{}%
\AgdaOperator{\AgdaInductiveConstructor{,}}\AgdaSpace{}%
\AgdaBound{ts}\AgdaSymbol{)}\AgdaSpace{}%
\AgdaSymbol{=}\AgdaSpace{}%
\AgdaFunction{wkNf}\AgdaSpace{}%
\AgdaBound{x}\AgdaSpace{}%
\AgdaBound{t}\AgdaSpace{}%
\AgdaOperator{\AgdaInductiveConstructor{,}}\AgdaSpace{}%
\AgdaFunction{wkSp}\AgdaSpace{}%
\AgdaBound{x}\AgdaSpace{}%
\AgdaBound{ts}\<%
\\
%
\\[\AgdaEmptyExtraSkip]%
\>[0]\AgdaFunction{appSp}\AgdaSpace{}%
\AgdaSymbol{:}\AgdaSpace{}%
\AgdaDatatype{Sp}\AgdaSpace{}%
\AgdaGeneralizable{Γ}\AgdaSpace{}%
\AgdaGeneralizable{A}\AgdaSpace{}%
\AgdaSymbol{(}\AgdaGeneralizable{B}\AgdaSpace{}%
\AgdaOperator{\AgdaInductiveConstructor{⇒}}\AgdaSpace{}%
\AgdaGeneralizable{C}\AgdaSymbol{)}\AgdaSpace{}%
\AgdaSymbol{→}\AgdaSpace{}%
\AgdaDatatype{Nf}\AgdaSpace{}%
\AgdaGeneralizable{Γ}\AgdaSpace{}%
\AgdaGeneralizable{B}\AgdaSpace{}%
\AgdaSymbol{→}\AgdaSpace{}%
\AgdaDatatype{Sp}\AgdaSpace{}%
\AgdaGeneralizable{Γ}\AgdaSpace{}%
\AgdaGeneralizable{A}\AgdaSpace{}%
\AgdaGeneralizable{C}\<%
\\
\>[0]\AgdaFunction{appSp}\AgdaSpace{}%
\AgdaInductiveConstructor{ε}\AgdaSpace{}%
\AgdaBound{u}\AgdaSpace{}%
\AgdaSymbol{=}\AgdaSpace{}%
\AgdaBound{u}\AgdaSpace{}%
\AgdaOperator{\AgdaInductiveConstructor{,}}\AgdaSpace{}%
\AgdaInductiveConstructor{ε}\<%
\\
\>[0]\AgdaFunction{appSp}\AgdaSpace{}%
\AgdaSymbol{(}\AgdaBound{t}\AgdaSpace{}%
\AgdaOperator{\AgdaInductiveConstructor{,}}\AgdaSpace{}%
\AgdaBound{ts}\AgdaSymbol{)}\AgdaSpace{}%
\AgdaBound{u}\AgdaSpace{}%
\AgdaSymbol{=}\AgdaSpace{}%
\AgdaBound{t}\AgdaSpace{}%
\AgdaOperator{\AgdaInductiveConstructor{,}}\AgdaSpace{}%
\AgdaFunction{appSp}\AgdaSpace{}%
\AgdaBound{ts}\AgdaSpace{}%
\AgdaBound{u}\<%
\\
%
\\[\AgdaEmptyExtraSkip]%
\>[0]\AgdaFunction{nvar}\AgdaSpace{}%
\AgdaSymbol{:}\AgdaSpace{}%
\AgdaDatatype{Var}\AgdaSpace{}%
\AgdaGeneralizable{Γ}\AgdaSpace{}%
\AgdaGeneralizable{A}\AgdaSpace{}%
\AgdaSymbol{→}\AgdaSpace{}%
\AgdaDatatype{Nf}\AgdaSpace{}%
\AgdaGeneralizable{Γ}\AgdaSpace{}%
\AgdaGeneralizable{A}\<%
\\
%
\\[\AgdaEmptyExtraSkip]%
\>[0]\AgdaFunction{ne2nf}\AgdaSpace{}%
\AgdaSymbol{:}\AgdaSpace{}%
\AgdaRecord{Ne}\AgdaSpace{}%
\AgdaGeneralizable{Γ}\AgdaSpace{}%
\AgdaGeneralizable{A}\AgdaSpace{}%
\AgdaSymbol{→}\AgdaSpace{}%
\AgdaDatatype{Nf}\AgdaSpace{}%
\AgdaGeneralizable{Γ}\AgdaSpace{}%
\AgdaGeneralizable{A}\<%
\\
%
\\[\AgdaEmptyExtraSkip]%
\>[0]\AgdaFunction{nvar}\AgdaSpace{}%
\AgdaSymbol{\{}\AgdaBound{Γ}\AgdaSymbol{\}}\AgdaSpace{}%
\AgdaSymbol{\{}\AgdaBound{B}\AgdaSymbol{\}}\AgdaSpace{}%
\AgdaBound{x}\AgdaSpace{}%
\AgdaSymbol{=}\AgdaSpace{}%
\AgdaFunction{ne2nf}\AgdaSpace{}%
\AgdaSymbol{(}\AgdaFunction{S}\AgdaSpace{}%
\AgdaOperator{\AgdaInductiveConstructor{◃}}\AgdaSpace{}%
\AgdaFunction{P}\AgdaSpace{}%
\AgdaOperator{\AgdaInductiveConstructor{◃}}\AgdaSpace{}%
\AgdaFunction{R}\AgdaSymbol{)}\<%
\\
\>[0][@{}l@{\AgdaIndent{0}}]%
\>[2]\AgdaKeyword{where}\<%
\\
%
\>[2]\AgdaFunction{S}\AgdaSpace{}%
\AgdaSymbol{:}\AgdaSpace{}%
\AgdaFunction{Type}\<%
\\
%
\>[2]\AgdaFunction{S}\AgdaSpace{}%
\AgdaSymbol{=}\AgdaSpace{}%
\AgdaRecord{⊤}\<%
\\
%
\\[\AgdaEmptyExtraSkip]%
%
\>[2]\AgdaFunction{P}\AgdaSpace{}%
\AgdaSymbol{:}\AgdaSpace{}%
\AgdaDatatype{Var}\AgdaSpace{}%
\AgdaBound{Γ}\AgdaSpace{}%
\AgdaGeneralizable{A}\AgdaSpace{}%
\AgdaSymbol{→}\AgdaSpace{}%
\AgdaFunction{S}\AgdaSpace{}%
\AgdaSymbol{→}\AgdaSpace{}%
\AgdaFunction{Type}\<%
\\
%
\>[2]\AgdaFunction{P}\AgdaSpace{}%
\AgdaBound{y}%
\>[7]\AgdaInductiveConstructor{tt}\AgdaSpace{}%
\AgdaKeyword{with}\AgdaSpace{}%
\AgdaFunction{eq}\AgdaSpace{}%
\AgdaBound{x}\AgdaSpace{}%
\AgdaBound{y}\<%
\\
%
\>[2]\AgdaFunction{P}\AgdaSpace{}%
\AgdaDottedPattern{\AgdaSymbol{.}}\AgdaDottedPattern{\AgdaBound{x}}\AgdaSpace{}%
\AgdaInductiveConstructor{tt}\AgdaSpace{}%
\AgdaSymbol{|}\AgdaSpace{}%
\AgdaInductiveConstructor{same}\AgdaSpace{}%
\AgdaSymbol{=}\AgdaSpace{}%
\AgdaRecord{⊤}\<%
\\
%
\>[2]\AgdaFunction{P}\AgdaSpace{}%
\AgdaBound{y}%
\>[7]\AgdaInductiveConstructor{tt}\AgdaSpace{}%
\AgdaSymbol{|}\AgdaSpace{}%
\AgdaInductiveConstructor{diff}\AgdaSpace{}%
\AgdaDottedPattern{\AgdaSymbol{.}}\AgdaDottedPattern{\AgdaBound{x}}\AgdaSpace{}%
\AgdaBound{y'}\AgdaSpace{}%
\AgdaSymbol{=}\AgdaSpace{}%
\AgdaFunction{⊥}\<%
\\
%
\\[\AgdaEmptyExtraSkip]%
%
\>[2]\AgdaFunction{R}\AgdaSpace{}%
\AgdaSymbol{:}\AgdaSpace{}%
\AgdaSymbol{(}\AgdaBound{y}\AgdaSpace{}%
\AgdaSymbol{:}\AgdaSpace{}%
\AgdaDatatype{Var}\AgdaSpace{}%
\AgdaBound{Γ}\AgdaSpace{}%
\AgdaGeneralizable{A}\AgdaSymbol{)}\AgdaSpace{}%
\AgdaSymbol{(}\AgdaBound{s}\AgdaSpace{}%
\AgdaSymbol{:}\AgdaSpace{}%
\AgdaFunction{S}\AgdaSymbol{)}\AgdaSpace{}%
\AgdaSymbol{→}\AgdaSpace{}%
\AgdaFunction{P}\AgdaSpace{}%
\AgdaBound{y}\AgdaSpace{}%
\AgdaBound{s}\AgdaSpace{}%
\AgdaSymbol{→}\AgdaSpace{}%
\AgdaDatatype{Sp}\AgdaSpace{}%
\AgdaBound{Γ}\AgdaSpace{}%
\AgdaGeneralizable{A}\AgdaSpace{}%
\AgdaBound{B}\<%
\\
%
\>[2]\AgdaFunction{R}\AgdaSpace{}%
\AgdaBound{y}\AgdaSpace{}%
\AgdaInductiveConstructor{tt}\AgdaSpace{}%
\AgdaBound{p}\AgdaSpace{}%
\AgdaKeyword{with}\AgdaSpace{}%
\AgdaFunction{eq}\AgdaSpace{}%
\AgdaBound{x}\AgdaSpace{}%
\AgdaBound{y}\<%
\\
%
\>[2]\AgdaFunction{R}\AgdaSpace{}%
\AgdaDottedPattern{\AgdaSymbol{.}}\AgdaDottedPattern{\AgdaBound{x}}\AgdaSpace{}%
\AgdaInductiveConstructor{tt}\AgdaSpace{}%
\AgdaBound{p}\AgdaSpace{}%
\AgdaSymbol{|}\AgdaSpace{}%
\AgdaInductiveConstructor{same}\AgdaSpace{}%
\AgdaSymbol{=}\AgdaSpace{}%
\AgdaInductiveConstructor{ε}\<%
\\
%
\>[2]\AgdaFunction{R}\AgdaSpace{}%
\AgdaBound{y}\AgdaSpace{}%
\AgdaInductiveConstructor{tt}\AgdaSpace{}%
\AgdaSymbol{()}\AgdaSpace{}%
\AgdaSymbol{|}\AgdaSpace{}%
\AgdaInductiveConstructor{diff}\AgdaSpace{}%
\AgdaDottedPattern{\AgdaSymbol{.}}\AgdaDottedPattern{\AgdaBound{x}}\AgdaSpace{}%
\AgdaBound{y'}\<%
\\
%
\\[\AgdaEmptyExtraSkip]%
\>[0]\AgdaFunction{ne2nf}\AgdaSpace{}%
\AgdaSymbol{\{}\AgdaBound{Γ}\AgdaSymbol{\}}\AgdaSpace{}%
\AgdaSymbol{\{}\AgdaInductiveConstructor{*}\AgdaSymbol{\}}\AgdaSpace{}%
\AgdaBound{spr}\AgdaSpace{}%
\AgdaSymbol{=}\AgdaSpace{}%
\AgdaInductiveConstructor{ne}\AgdaSpace{}%
\AgdaBound{spr}\<%
\\
\>[0]\AgdaFunction{ne2nf}\AgdaSpace{}%
\AgdaSymbol{\{}\AgdaBound{Γ}\AgdaSymbol{\}}\AgdaSpace{}%
\AgdaSymbol{\{}\AgdaBound{A}\AgdaSpace{}%
\AgdaOperator{\AgdaInductiveConstructor{⇒}}\AgdaSpace{}%
\AgdaBound{C}\AgdaSymbol{\}}\AgdaSpace{}%
\AgdaSymbol{(}\AgdaBound{S}\AgdaSpace{}%
\AgdaOperator{\AgdaInductiveConstructor{◃}}\AgdaSpace{}%
\AgdaBound{P}\AgdaSpace{}%
\AgdaOperator{\AgdaInductiveConstructor{◃}}\AgdaSpace{}%
\AgdaBound{R}\AgdaSymbol{)}\AgdaSpace{}%
\AgdaSymbol{=}\AgdaSpace{}%
\AgdaInductiveConstructor{lam}\AgdaSpace{}%
\AgdaSymbol{(}\AgdaFunction{ne2nf}\AgdaSpace{}%
\AgdaSymbol{(}\AgdaBound{S}\AgdaSpace{}%
\AgdaOperator{\AgdaInductiveConstructor{◃}}\AgdaSpace{}%
\AgdaFunction{P'}\AgdaSpace{}%
\AgdaOperator{\AgdaInductiveConstructor{◃}}\AgdaSpace{}%
\AgdaFunction{R'}\AgdaSymbol{))}\<%
\\
\>[0][@{}l@{\AgdaIndent{0}}]%
\>[2]\AgdaKeyword{where}\<%
\\
%
\>[2]\AgdaFunction{P'}\AgdaSpace{}%
\AgdaSymbol{:}\AgdaSpace{}%
\AgdaDatatype{Var}\AgdaSpace{}%
\AgdaSymbol{(}\AgdaBound{Γ}\AgdaSpace{}%
\AgdaOperator{\AgdaInductiveConstructor{▹}}\AgdaSpace{}%
\AgdaBound{A}\AgdaSymbol{)}\AgdaSpace{}%
\AgdaGeneralizable{B}\AgdaSpace{}%
\AgdaSymbol{→}\AgdaSpace{}%
\AgdaBound{S}\AgdaSpace{}%
\AgdaSymbol{→}\AgdaSpace{}%
\AgdaFunction{Type}\<%
\\
%
\>[2]\AgdaFunction{P'}\AgdaSpace{}%
\AgdaInductiveConstructor{vz}\AgdaSpace{}%
\AgdaBound{s}\AgdaSpace{}%
\AgdaSymbol{=}\AgdaSpace{}%
\AgdaFunction{⊥}\<%
\\
%
\>[2]\AgdaFunction{P'}\AgdaSpace{}%
\AgdaSymbol{(}\AgdaInductiveConstructor{vs}\AgdaSpace{}%
\AgdaBound{x}\AgdaSymbol{)}\AgdaSpace{}%
\AgdaBound{s}\AgdaSpace{}%
\AgdaSymbol{=}\AgdaSpace{}%
\AgdaBound{P}\AgdaSpace{}%
\AgdaBound{x}\AgdaSpace{}%
\AgdaBound{s}\<%
\\
%
\\[\AgdaEmptyExtraSkip]%
%
\>[2]\AgdaFunction{R'}\AgdaSpace{}%
\AgdaSymbol{:}\AgdaSpace{}%
\AgdaSymbol{(}\AgdaBound{x}\AgdaSpace{}%
\AgdaSymbol{:}\AgdaSpace{}%
\AgdaDatatype{Var}\AgdaSpace{}%
\AgdaSymbol{(}\AgdaBound{Γ}\AgdaSpace{}%
\AgdaOperator{\AgdaInductiveConstructor{▹}}\AgdaSpace{}%
\AgdaBound{A}\AgdaSymbol{)}\AgdaSpace{}%
\AgdaGeneralizable{B}\AgdaSymbol{)}\AgdaSpace{}%
\AgdaSymbol{(}\AgdaBound{s}\AgdaSpace{}%
\AgdaSymbol{:}\AgdaSpace{}%
\AgdaBound{S}\AgdaSymbol{)}\AgdaSpace{}%
\AgdaSymbol{→}\AgdaSpace{}%
\AgdaFunction{P'}\AgdaSpace{}%
\AgdaBound{x}\AgdaSpace{}%
\AgdaBound{s}\AgdaSpace{}%
\AgdaSymbol{→}\AgdaSpace{}%
\AgdaDatatype{Sp}\AgdaSpace{}%
\AgdaSymbol{(}\AgdaBound{Γ}\AgdaSpace{}%
\AgdaOperator{\AgdaInductiveConstructor{▹}}\AgdaSpace{}%
\AgdaBound{A}\AgdaSymbol{)}\AgdaSpace{}%
\AgdaGeneralizable{B}\AgdaSpace{}%
\AgdaBound{C}\<%
\\
%
\>[2]\AgdaFunction{R'}\AgdaSpace{}%
\AgdaInductiveConstructor{vz}\AgdaSpace{}%
\AgdaBound{s}\AgdaSpace{}%
\AgdaSymbol{()}\<%
\\
%
\>[2]\AgdaFunction{R'}\AgdaSpace{}%
\AgdaSymbol{(}\AgdaInductiveConstructor{vs}\AgdaSpace{}%
\AgdaBound{x}\AgdaSymbol{)}\AgdaSpace{}%
\AgdaBound{s}\AgdaSpace{}%
\AgdaBound{p}\AgdaSpace{}%
\AgdaSymbol{=}\AgdaSpace{}%
\AgdaFunction{appSp}\AgdaSpace{}%
\AgdaSymbol{(}\AgdaFunction{wkSp}\AgdaSpace{}%
\AgdaInductiveConstructor{vz}\AgdaSpace{}%
\AgdaSymbol{(}\AgdaBound{R}\AgdaSpace{}%
\AgdaBound{x}\AgdaSpace{}%
\AgdaBound{s}\AgdaSpace{}%
\AgdaBound{p}\AgdaSymbol{))}\AgdaSpace{}%
\AgdaSymbol{(}\AgdaFunction{nvar}\AgdaSpace{}%
\AgdaInductiveConstructor{vz}\AgdaSymbol{)}\<%
\\
%
\\[\AgdaEmptyExtraSkip]%
\>[0]\AgdaComment{\{-\ Normalization\ -\}}\<%
\\
%
\\[\AgdaEmptyExtraSkip]%
\>[0]\AgdaKeyword{mutual}\<%
\\
\>[0][@{}l@{\AgdaIndent{0}}]%
\>[2]\AgdaOperator{\AgdaFunction{\AgdaUnderscore{}[\AgdaUnderscore{}:=\AgdaUnderscore{}]}}\AgdaSpace{}%
\AgdaSymbol{:}\AgdaSpace{}%
\AgdaDatatype{Nf}\AgdaSpace{}%
\AgdaGeneralizable{Γ}\AgdaSpace{}%
\AgdaGeneralizable{B}\AgdaSpace{}%
\AgdaSymbol{→}\AgdaSpace{}%
\AgdaSymbol{(}\AgdaBound{x}\AgdaSpace{}%
\AgdaSymbol{:}\AgdaSpace{}%
\AgdaDatatype{Var}\AgdaSpace{}%
\AgdaGeneralizable{Γ}\AgdaSpace{}%
\AgdaGeneralizable{A}\AgdaSymbol{)}\AgdaSpace{}%
\AgdaSymbol{→}\AgdaSpace{}%
\AgdaDatatype{Nf}\AgdaSpace{}%
\AgdaSymbol{(}\AgdaGeneralizable{Γ}\AgdaSpace{}%
\AgdaOperator{\AgdaFunction{-}}\AgdaSpace{}%
\AgdaBound{x}\AgdaSymbol{)}\AgdaSpace{}%
\AgdaGeneralizable{A}\AgdaSpace{}%
\AgdaSymbol{→}\AgdaSpace{}%
\AgdaDatatype{Nf}\AgdaSpace{}%
\AgdaSymbol{(}\AgdaGeneralizable{Γ}\AgdaSpace{}%
\AgdaOperator{\AgdaFunction{-}}\AgdaSpace{}%
\AgdaBound{x}\AgdaSymbol{)}\AgdaSpace{}%
\AgdaGeneralizable{B}\<%
\\
%
\>[2]\AgdaInductiveConstructor{lam}\AgdaSpace{}%
\AgdaBound{t}\AgdaSpace{}%
\AgdaOperator{\AgdaFunction{[}}\AgdaSpace{}%
\AgdaBound{x}\AgdaSpace{}%
\AgdaOperator{\AgdaFunction{:=}}\AgdaSpace{}%
\AgdaBound{u}\AgdaSpace{}%
\AgdaOperator{\AgdaFunction{]}}\AgdaSpace{}%
\AgdaSymbol{=}\AgdaSpace{}%
\AgdaInductiveConstructor{lam}\AgdaSpace{}%
\AgdaSymbol{(}\AgdaBound{t}\AgdaSpace{}%
\AgdaOperator{\AgdaFunction{[}}\AgdaSpace{}%
\AgdaInductiveConstructor{vs}\AgdaSpace{}%
\AgdaBound{x}\AgdaSpace{}%
\AgdaOperator{\AgdaFunction{:=}}\AgdaSpace{}%
\AgdaFunction{wkNf}\AgdaSpace{}%
\AgdaInductiveConstructor{vz}\AgdaSpace{}%
\AgdaBound{u}\AgdaSpace{}%
\AgdaOperator{\AgdaFunction{]}}\AgdaSymbol{)}\<%
\\
%
\>[2]\AgdaInductiveConstructor{ne}\AgdaSpace{}%
\AgdaSymbol{\{}\AgdaBound{Γ}\AgdaSymbol{\}}\AgdaSpace{}%
\AgdaSymbol{(}\AgdaBound{S}\AgdaSpace{}%
\AgdaOperator{\AgdaInductiveConstructor{◃}}\AgdaSpace{}%
\AgdaBound{P}\AgdaSpace{}%
\AgdaOperator{\AgdaInductiveConstructor{◃}}\AgdaSpace{}%
\AgdaBound{R}\AgdaSymbol{)}\AgdaSpace{}%
\AgdaOperator{\AgdaFunction{[}}\AgdaSpace{}%
\AgdaBound{x}\AgdaSpace{}%
\AgdaOperator{\AgdaFunction{:=}}\AgdaSpace{}%
\AgdaBound{u}\AgdaSpace{}%
\AgdaOperator{\AgdaFunction{]}}\AgdaSpace{}%
\AgdaSymbol{=}\AgdaSpace{}%
\AgdaInductiveConstructor{ne}\AgdaSpace{}%
\AgdaSymbol{(}\AgdaBound{S}\AgdaSpace{}%
\AgdaOperator{\AgdaInductiveConstructor{◃}}\AgdaSpace{}%
\AgdaFunction{P'}\AgdaSpace{}%
\AgdaOperator{\AgdaInductiveConstructor{◃}}\AgdaSpace{}%
\AgdaFunction{R'}\AgdaSymbol{)}\<%
\\
\>[2][@{}l@{\AgdaIndent{0}}]%
\>[4]\AgdaKeyword{where}\<%
\\
%
\>[4]\AgdaFunction{P'}\AgdaSpace{}%
\AgdaSymbol{:}\AgdaSpace{}%
\AgdaDatatype{Var}\AgdaSpace{}%
\AgdaSymbol{(}\AgdaBound{Γ}\AgdaSpace{}%
\AgdaOperator{\AgdaFunction{-}}\AgdaSpace{}%
\AgdaBound{x}\AgdaSymbol{)}\AgdaSpace{}%
\AgdaGeneralizable{A}\AgdaSpace{}%
\AgdaSymbol{→}\AgdaSpace{}%
\AgdaBound{S}\AgdaSpace{}%
\AgdaSymbol{→}\AgdaSpace{}%
\AgdaFunction{Type}\<%
\\
%
\>[4]\AgdaFunction{P'}\AgdaSpace{}%
\AgdaBound{y}\AgdaSpace{}%
\AgdaBound{s}\AgdaSpace{}%
\AgdaSymbol{=}\AgdaSpace{}%
\AgdaBound{P}\AgdaSpace{}%
\AgdaSymbol{(}\AgdaFunction{wkVar}\AgdaSpace{}%
\AgdaBound{x}\AgdaSpace{}%
\AgdaBound{y}\AgdaSymbol{)}\AgdaSpace{}%
\AgdaBound{s}\<%
\\
%
\\[\AgdaEmptyExtraSkip]%
%
\>[4]\AgdaFunction{R'}\AgdaSpace{}%
\AgdaSymbol{:}\AgdaSpace{}%
\AgdaSymbol{(}\AgdaBound{y}\AgdaSpace{}%
\AgdaSymbol{:}\AgdaSpace{}%
\AgdaDatatype{Var}\AgdaSpace{}%
\AgdaSymbol{(}\AgdaBound{Γ}\AgdaSpace{}%
\AgdaOperator{\AgdaFunction{-}}\AgdaSpace{}%
\AgdaBound{x}\AgdaSymbol{)}\AgdaSpace{}%
\AgdaGeneralizable{A}\AgdaSymbol{)}\AgdaSpace{}%
\AgdaSymbol{(}\AgdaBound{s}\AgdaSpace{}%
\AgdaSymbol{:}\AgdaSpace{}%
\AgdaBound{S}\AgdaSymbol{)}\AgdaSpace{}%
\AgdaSymbol{→}\AgdaSpace{}%
\AgdaFunction{P'}\AgdaSpace{}%
\AgdaBound{y}\AgdaSpace{}%
\AgdaBound{s}\AgdaSpace{}%
\AgdaSymbol{→}\AgdaSpace{}%
\AgdaDatatype{Sp}\AgdaSpace{}%
\AgdaSymbol{(}\AgdaBound{Γ}\AgdaSpace{}%
\AgdaOperator{\AgdaFunction{-}}\AgdaSpace{}%
\AgdaBound{x}\AgdaSymbol{)}\AgdaSpace{}%
\AgdaGeneralizable{A}\AgdaSpace{}%
\AgdaInductiveConstructor{*}\<%
\\
%
\>[4]\AgdaFunction{R'}\AgdaSpace{}%
\AgdaBound{y}\AgdaSpace{}%
\AgdaBound{s}\AgdaSpace{}%
\AgdaBound{p}\AgdaSpace{}%
\AgdaSymbol{=}\AgdaSpace{}%
\AgdaBound{R}\AgdaSpace{}%
\AgdaSymbol{(}\AgdaFunction{wkVar}\AgdaSpace{}%
\AgdaBound{x}\AgdaSpace{}%
\AgdaBound{y}\AgdaSymbol{)}\AgdaSpace{}%
\AgdaBound{s}\AgdaSpace{}%
\AgdaBound{p}\AgdaSpace{}%
\AgdaOperator{\AgdaFunction{<}}\AgdaSpace{}%
\AgdaBound{x}\AgdaSpace{}%
\AgdaOperator{\AgdaFunction{:=}}\AgdaSpace{}%
\AgdaBound{u}\AgdaSpace{}%
\AgdaOperator{\AgdaFunction{>}}\<%
\\
%
\\[\AgdaEmptyExtraSkip]%
%
\>[2]\AgdaOperator{\AgdaFunction{\AgdaUnderscore{}<\AgdaUnderscore{}:=\AgdaUnderscore{}>}}\AgdaSpace{}%
\AgdaSymbol{:}\AgdaSpace{}%
\AgdaDatatype{Sp}\AgdaSpace{}%
\AgdaGeneralizable{Γ}\AgdaSpace{}%
\AgdaGeneralizable{B}\AgdaSpace{}%
\AgdaGeneralizable{C}\AgdaSpace{}%
\AgdaSymbol{→}\AgdaSpace{}%
\AgdaSymbol{(}\AgdaBound{x}\AgdaSpace{}%
\AgdaSymbol{:}\AgdaSpace{}%
\AgdaDatatype{Var}\AgdaSpace{}%
\AgdaGeneralizable{Γ}\AgdaSpace{}%
\AgdaGeneralizable{A}\AgdaSymbol{)}\AgdaSpace{}%
\AgdaSymbol{→}\AgdaSpace{}%
\AgdaDatatype{Nf}\AgdaSpace{}%
\AgdaSymbol{(}\AgdaGeneralizable{Γ}\AgdaSpace{}%
\AgdaOperator{\AgdaFunction{-}}\AgdaSpace{}%
\AgdaBound{x}\AgdaSymbol{)}\AgdaSpace{}%
\AgdaGeneralizable{A}\AgdaSpace{}%
\AgdaSymbol{→}\AgdaSpace{}%
\AgdaDatatype{Sp}\AgdaSpace{}%
\AgdaSymbol{(}\AgdaGeneralizable{Γ}\AgdaSpace{}%
\AgdaOperator{\AgdaFunction{-}}\AgdaSpace{}%
\AgdaBound{x}\AgdaSymbol{)}\AgdaSpace{}%
\AgdaGeneralizable{B}\AgdaSpace{}%
\AgdaGeneralizable{C}\<%
\\
%
\>[2]\AgdaInductiveConstructor{ε}\AgdaSpace{}%
\AgdaOperator{\AgdaFunction{<}}\AgdaSpace{}%
\AgdaBound{x}\AgdaSpace{}%
\AgdaOperator{\AgdaFunction{:=}}\AgdaSpace{}%
\AgdaBound{u}\AgdaSpace{}%
\AgdaOperator{\AgdaFunction{>}}\AgdaSpace{}%
\AgdaSymbol{=}\AgdaSpace{}%
\AgdaInductiveConstructor{ε}\<%
\\
%
\>[2]\AgdaSymbol{(}\AgdaBound{t}\AgdaSpace{}%
\AgdaOperator{\AgdaInductiveConstructor{,}}\AgdaSpace{}%
\AgdaBound{ts}\AgdaSymbol{)}\AgdaSpace{}%
\AgdaOperator{\AgdaFunction{<}}\AgdaSpace{}%
\AgdaBound{x}\AgdaSpace{}%
\AgdaOperator{\AgdaFunction{:=}}\AgdaSpace{}%
\AgdaBound{u}\AgdaSpace{}%
\AgdaOperator{\AgdaFunction{>}}\AgdaSpace{}%
\AgdaSymbol{=}\AgdaSpace{}%
\AgdaSymbol{(}\AgdaBound{t}\AgdaSpace{}%
\AgdaOperator{\AgdaFunction{[}}\AgdaSpace{}%
\AgdaBound{x}\AgdaSpace{}%
\AgdaOperator{\AgdaFunction{:=}}\AgdaSpace{}%
\AgdaBound{u}\AgdaSpace{}%
\AgdaOperator{\AgdaFunction{]}}\AgdaSymbol{)}\AgdaSpace{}%
\AgdaOperator{\AgdaInductiveConstructor{,}}\AgdaSpace{}%
\AgdaSymbol{(}\AgdaBound{ts}\AgdaSpace{}%
\AgdaOperator{\AgdaFunction{<}}\AgdaSpace{}%
\AgdaBound{x}\AgdaSpace{}%
\AgdaOperator{\AgdaFunction{:=}}\AgdaSpace{}%
\AgdaBound{u}\AgdaSpace{}%
\AgdaOperator{\AgdaFunction{>}}\AgdaSymbol{)}\<%
\\
%
\\[\AgdaEmptyExtraSkip]%
%
\>[2]\AgdaOperator{\AgdaFunction{\AgdaUnderscore{}◇\AgdaUnderscore{}}}\AgdaSpace{}%
\AgdaSymbol{:}\AgdaSpace{}%
\AgdaDatatype{Nf}\AgdaSpace{}%
\AgdaGeneralizable{Γ}\AgdaSpace{}%
\AgdaGeneralizable{A}\AgdaSpace{}%
\AgdaSymbol{→}\AgdaSpace{}%
\AgdaDatatype{Sp}\AgdaSpace{}%
\AgdaGeneralizable{Γ}\AgdaSpace{}%
\AgdaGeneralizable{A}\AgdaSpace{}%
\AgdaGeneralizable{B}\AgdaSpace{}%
\AgdaSymbol{→}\AgdaSpace{}%
\AgdaDatatype{Nf}\AgdaSpace{}%
\AgdaGeneralizable{Γ}\AgdaSpace{}%
\AgdaGeneralizable{B}\<%
\\
%
\>[2]\AgdaBound{t}\AgdaSpace{}%
\AgdaOperator{\AgdaFunction{◇}}\AgdaSpace{}%
\AgdaInductiveConstructor{ε}\AgdaSpace{}%
\AgdaSymbol{=}\AgdaSpace{}%
\AgdaBound{t}\<%
\\
%
\>[2]\AgdaBound{t}\AgdaSpace{}%
\AgdaOperator{\AgdaFunction{◇}}\AgdaSpace{}%
\AgdaSymbol{(}\AgdaBound{u}\AgdaSpace{}%
\AgdaOperator{\AgdaInductiveConstructor{,}}\AgdaSpace{}%
\AgdaBound{us}\AgdaSymbol{)}\AgdaSpace{}%
\AgdaSymbol{=}\AgdaSpace{}%
\AgdaFunction{napp}\AgdaSpace{}%
\AgdaBound{t}\AgdaSpace{}%
\AgdaBound{u}\AgdaSpace{}%
\AgdaOperator{\AgdaFunction{◇}}\AgdaSpace{}%
\AgdaBound{us}\<%
\end{code}

\begin{code}%
%
\>[2]\AgdaFunction{napp}\AgdaSpace{}%
\AgdaSymbol{:}\AgdaSpace{}%
\AgdaDatatype{Nf}\AgdaSpace{}%
\AgdaGeneralizable{Γ}\AgdaSpace{}%
\AgdaSymbol{(}\AgdaGeneralizable{A}\AgdaSpace{}%
\AgdaOperator{\AgdaInductiveConstructor{⇒}}\AgdaSpace{}%
\AgdaGeneralizable{B}\AgdaSymbol{)}\AgdaSpace{}%
\AgdaSymbol{→}\AgdaSpace{}%
\AgdaDatatype{Nf}\AgdaSpace{}%
\AgdaGeneralizable{Γ}\AgdaSpace{}%
\AgdaGeneralizable{A}\AgdaSpace{}%
\AgdaSymbol{→}\AgdaSpace{}%
\AgdaDatatype{Nf}\AgdaSpace{}%
\AgdaGeneralizable{Γ}\AgdaSpace{}%
\AgdaGeneralizable{B}\<%
\\
%
\>[2]\AgdaFunction{napp}\AgdaSpace{}%
\AgdaSymbol{(}\AgdaInductiveConstructor{lam}\AgdaSpace{}%
\AgdaBound{t}\AgdaSymbol{)}\AgdaSpace{}%
\AgdaBound{u}\AgdaSpace{}%
\AgdaSymbol{=}\AgdaSpace{}%
\AgdaBound{t}\AgdaSpace{}%
\AgdaOperator{\AgdaFunction{[}}\AgdaSpace{}%
\AgdaInductiveConstructor{vz}\AgdaSpace{}%
\AgdaOperator{\AgdaFunction{:=}}\AgdaSpace{}%
\AgdaBound{u}\AgdaSpace{}%
\AgdaOperator{\AgdaFunction{]}}\<%
\end{code}

TODO

\subsubsection*{Categorical Structures}

\begin{code}[hide]%
\>[0]\AgdaKeyword{mutual}\<%
\end{code}

\begin{code}%
\>[0][@{}l@{\AgdaIndent{1}}]%
\>[2]\AgdaKeyword{data}\AgdaSpace{}%
\AgdaDatatype{NfHom}\AgdaSpace{}%
\AgdaSymbol{:}\AgdaSpace{}%
\AgdaSymbol{(}\AgdaBound{t}\AgdaSpace{}%
\AgdaBound{u}\AgdaSpace{}%
\AgdaSymbol{:}\AgdaSpace{}%
\AgdaDatatype{Nf}\AgdaSpace{}%
\AgdaGeneralizable{Γ}\AgdaSpace{}%
\AgdaGeneralizable{A}\AgdaSymbol{)}\AgdaSpace{}%
\AgdaSymbol{→}\AgdaSpace{}%
\AgdaFunction{Type₁}\AgdaSpace{}%
\AgdaKeyword{where}\<%
\\
\>[2][@{}l@{\AgdaIndent{0}}]%
\>[4]\AgdaInductiveConstructor{lam}\AgdaSpace{}%
\AgdaSymbol{:}\AgdaSpace{}%
\AgdaDatatype{NfHom}\AgdaSpace{}%
\AgdaGeneralizable{t}\AgdaSpace{}%
\AgdaGeneralizable{u}\AgdaSpace{}%
\AgdaSymbol{→}\AgdaSpace{}%
\AgdaDatatype{NfHom}\AgdaSpace{}%
\AgdaSymbol{(}\AgdaInductiveConstructor{lam}\AgdaSpace{}%
\AgdaGeneralizable{t}\AgdaSymbol{)}\AgdaSpace{}%
\AgdaSymbol{(}\AgdaInductiveConstructor{lam}\AgdaSpace{}%
\AgdaGeneralizable{u}\AgdaSymbol{)}\<%
\\
%
\>[4]\AgdaInductiveConstructor{ne}%
\>[8]\AgdaSymbol{:}\AgdaSpace{}%
\AgdaRecord{NeHom}\AgdaSpace{}%
\AgdaGeneralizable{spr}\AgdaSpace{}%
\AgdaGeneralizable{tql}\AgdaSpace{}%
\AgdaSymbol{→}\AgdaSpace{}%
\AgdaDatatype{NfHom}\AgdaSpace{}%
\AgdaSymbol{(}\AgdaInductiveConstructor{ne}\AgdaSpace{}%
\AgdaGeneralizable{spr}\AgdaSymbol{)}\AgdaSpace{}%
\AgdaSymbol{(}\AgdaInductiveConstructor{ne}\AgdaSpace{}%
\AgdaGeneralizable{tql}\AgdaSymbol{)}\<%
\\
%
\\[\AgdaEmptyExtraSkip]%
%
\>[2]\AgdaKeyword{record}\AgdaSpace{}%
\AgdaRecord{NeHom}\AgdaSpace{}%
\AgdaSymbol{\{}\AgdaBound{Γ}\AgdaSymbol{\}}\AgdaSpace{}%
\AgdaSymbol{\{}\AgdaBound{B}\AgdaSymbol{\}}\AgdaSpace{}%
\AgdaSymbol{(}\AgdaBound{spr}\AgdaSpace{}%
\AgdaBound{tql}\AgdaSpace{}%
\AgdaSymbol{:}\AgdaSpace{}%
\AgdaRecord{Ne}\AgdaSpace{}%
\AgdaBound{Γ}\AgdaSpace{}%
\AgdaBound{B}\AgdaSymbol{)}\AgdaSpace{}%
\AgdaSymbol{:}\AgdaSpace{}%
\AgdaFunction{Type₁}\AgdaSpace{}%
\AgdaKeyword{where}\<%
\\
\>[2][@{}l@{\AgdaIndent{0}}]%
\>[4]\AgdaKeyword{constructor}\AgdaSpace{}%
\AgdaOperator{\AgdaInductiveConstructor{\AgdaUnderscore{}◃\AgdaUnderscore{}◃\AgdaUnderscore{}}}\<%
\\
%
\>[4]\AgdaKeyword{inductive}\<%
\\
%
\>[4]\AgdaKeyword{open}\AgdaSpace{}%
\AgdaModule{Ne}\AgdaSpace{}%
\AgdaBound{spr}\<%
\\
%
\>[4]\AgdaKeyword{open}\AgdaSpace{}%
\AgdaModule{Ne}\AgdaSpace{}%
\AgdaBound{tql}\AgdaSpace{}%
\AgdaKeyword{renaming}\AgdaSpace{}%
\AgdaSymbol{(}\AgdaField{S}\AgdaSpace{}%
\AgdaSymbol{to}\AgdaSpace{}%
\AgdaField{T}\AgdaSymbol{;}\AgdaSpace{}%
\AgdaField{P}\AgdaSpace{}%
\AgdaSymbol{to}\AgdaSpace{}%
\AgdaField{Q}\AgdaSymbol{;}\AgdaSpace{}%
\AgdaField{R}\AgdaSpace{}%
\AgdaSymbol{to}\AgdaSpace{}%
\AgdaField{L}\AgdaSymbol{)}\<%
\\
%
\>[4]\AgdaKeyword{field}\<%
\\
\>[4][@{}l@{\AgdaIndent{0}}]%
\>[6]\AgdaField{f}\AgdaSpace{}%
\AgdaSymbol{:}\AgdaSpace{}%
\AgdaFunction{S}\AgdaSpace{}%
\AgdaSymbol{→}\AgdaSpace{}%
\AgdaFunction{T}\<%
\\
%
\>[6]\AgdaField{g}\AgdaSpace{}%
\AgdaSymbol{:}\AgdaSpace{}%
\AgdaSymbol{(}\AgdaBound{x}\AgdaSpace{}%
\AgdaSymbol{:}\AgdaSpace{}%
\AgdaDatatype{Var}\AgdaSpace{}%
\AgdaBound{Γ}\AgdaSpace{}%
\AgdaGeneralizable{A}\AgdaSymbol{)}\AgdaSpace{}%
\AgdaSymbol{(}\AgdaBound{s}\AgdaSpace{}%
\AgdaSymbol{:}\AgdaSpace{}%
\AgdaFunction{S}\AgdaSymbol{)}\AgdaSpace{}%
\AgdaSymbol{→}\AgdaSpace{}%
\AgdaFunction{Q}\AgdaSpace{}%
\AgdaBound{x}\AgdaSpace{}%
\AgdaSymbol{(}\AgdaField{f}\AgdaSpace{}%
\AgdaBound{s}\AgdaSymbol{)}\AgdaSpace{}%
\AgdaSymbol{→}\AgdaSpace{}%
\AgdaFunction{P}\AgdaSpace{}%
\AgdaBound{x}\AgdaSpace{}%
\AgdaBound{s}\<%
\\
%
\>[6]\AgdaField{h}%
\>[1119I]\AgdaSymbol{:}\AgdaSpace{}%
\AgdaSymbol{(}\AgdaBound{x}\AgdaSpace{}%
\AgdaSymbol{:}\AgdaSpace{}%
\AgdaDatatype{Var}\AgdaSpace{}%
\AgdaBound{Γ}\AgdaSpace{}%
\AgdaGeneralizable{A}\AgdaSymbol{)}\AgdaSpace{}%
\AgdaSymbol{(}\AgdaBound{s}\AgdaSpace{}%
\AgdaSymbol{:}\AgdaSpace{}%
\AgdaFunction{S}\AgdaSymbol{)}\AgdaSpace{}%
\AgdaSymbol{(}\AgdaBound{q}\AgdaSpace{}%
\AgdaSymbol{:}\AgdaSpace{}%
\AgdaFunction{Q}\AgdaSpace{}%
\AgdaBound{x}\AgdaSpace{}%
\AgdaSymbol{(}\AgdaField{f}\AgdaSpace{}%
\AgdaBound{s}\AgdaSymbol{))}\<%
\\
\>[.][@{}l@{}]\<[1119I]%
\>[8]\AgdaSymbol{→}\AgdaSpace{}%
\AgdaDatatype{SpHom}\AgdaSpace{}%
\AgdaSymbol{(}\AgdaFunction{R}\AgdaSpace{}%
\AgdaBound{x}\AgdaSpace{}%
\AgdaBound{s}\AgdaSpace{}%
\AgdaSymbol{(}\AgdaField{g}\AgdaSpace{}%
\AgdaBound{x}\AgdaSpace{}%
\AgdaBound{s}\AgdaSpace{}%
\AgdaBound{q}\AgdaSymbol{))}\AgdaSpace{}%
\AgdaSymbol{(}\AgdaFunction{L}\AgdaSpace{}%
\AgdaBound{x}\AgdaSpace{}%
\AgdaSymbol{(}\AgdaField{f}\AgdaSpace{}%
\AgdaBound{s}\AgdaSymbol{)}\AgdaSpace{}%
\AgdaBound{q}\AgdaSymbol{)}\<%
\\
\>[0]\<%
\\
%
\>[2]\AgdaKeyword{data}\AgdaSpace{}%
\AgdaDatatype{SpHom}\AgdaSpace{}%
\AgdaSymbol{:}\AgdaSpace{}%
\AgdaSymbol{(}\AgdaBound{t}\AgdaSpace{}%
\AgdaBound{u}\AgdaSpace{}%
\AgdaSymbol{:}\AgdaSpace{}%
\AgdaDatatype{Sp}\AgdaSpace{}%
\AgdaGeneralizable{Γ}\AgdaSpace{}%
\AgdaGeneralizable{A}\AgdaSpace{}%
\AgdaGeneralizable{B}\AgdaSymbol{)}\AgdaSpace{}%
\AgdaSymbol{→}\AgdaSpace{}%
\AgdaFunction{Type₁}\AgdaSpace{}%
\AgdaKeyword{where}\<%
\\
\>[2][@{}l@{\AgdaIndent{0}}]%
\>[4]\AgdaInductiveConstructor{ε}%
\>[8]\AgdaSymbol{:}\AgdaSpace{}%
\AgdaDatatype{SpHom}\AgdaSpace{}%
\AgdaGeneralizable{ts}\AgdaSpace{}%
\AgdaGeneralizable{ts}\<%
\\
%
\>[4]\AgdaOperator{\AgdaInductiveConstructor{\AgdaUnderscore{},\AgdaUnderscore{}}}\AgdaSpace{}%
\AgdaSymbol{:}\AgdaSpace{}%
\AgdaDatatype{NfHom}\AgdaSpace{}%
\AgdaGeneralizable{t}\AgdaSpace{}%
\AgdaGeneralizable{u}\AgdaSpace{}%
\AgdaSymbol{→}\AgdaSpace{}%
\AgdaDatatype{SpHom}\AgdaSpace{}%
\AgdaGeneralizable{ts}\AgdaSpace{}%
\AgdaGeneralizable{us}\AgdaSpace{}%
\AgdaSymbol{→}\AgdaSpace{}%
\AgdaDatatype{SpHom}\AgdaSpace{}%
\AgdaSymbol{(}\AgdaGeneralizable{t}\AgdaSpace{}%
\AgdaOperator{\AgdaInductiveConstructor{,}}\AgdaSpace{}%
\AgdaGeneralizable{ts}\AgdaSymbol{)}\AgdaSpace{}%
\AgdaSymbol{(}\AgdaGeneralizable{u}\AgdaSpace{}%
\AgdaOperator{\AgdaInductiveConstructor{,}}\AgdaSpace{}%
\AgdaGeneralizable{us}\AgdaSymbol{)}\<%
\end{code}

\subsubsection*{Algebraic Structures}

\begin{code}%
\>[0]\AgdaFunction{onenf}\AgdaSpace{}%
\AgdaSymbol{:}\AgdaSpace{}%
\AgdaDatatype{Nf}\AgdaSpace{}%
\AgdaGeneralizable{Γ}\AgdaSpace{}%
\AgdaGeneralizable{A}\<%
\\
\>[0]\AgdaFunction{onenf}\AgdaSpace{}%
\AgdaSymbol{\{}\AgdaBound{Γ}\AgdaSymbol{\}}\AgdaSpace{}%
\AgdaSymbol{\{}\AgdaInductiveConstructor{*}\AgdaSymbol{\}}\AgdaSpace{}%
\AgdaSymbol{=}\AgdaSpace{}%
\AgdaInductiveConstructor{ne}\AgdaSpace{}%
\AgdaSymbol{(}\AgdaRecord{⊤}\AgdaSpace{}%
\AgdaOperator{\AgdaInductiveConstructor{◃}}\AgdaSpace{}%
\AgdaSymbol{(λ\{}\AgdaSpace{}%
\AgdaBound{x}\AgdaSpace{}%
\AgdaInductiveConstructor{tt}\AgdaSpace{}%
\AgdaSymbol{→}\AgdaSpace{}%
\AgdaFunction{⊥}\AgdaSpace{}%
\AgdaSymbol{\})}\AgdaSpace{}%
\AgdaOperator{\AgdaInductiveConstructor{◃}}\AgdaSpace{}%
\AgdaSymbol{λ\{}\AgdaSpace{}%
\AgdaBound{x}\AgdaSpace{}%
\AgdaInductiveConstructor{tt}\AgdaSpace{}%
\AgdaSymbol{()}\AgdaSpace{}%
\AgdaSymbol{\})}\<%
\\
\>[0]\AgdaFunction{onenf}\AgdaSpace{}%
\AgdaSymbol{\{}\AgdaBound{Γ}\AgdaSymbol{\}}\AgdaSpace{}%
\AgdaSymbol{\{}\AgdaBound{A}\AgdaSpace{}%
\AgdaOperator{\AgdaInductiveConstructor{⇒}}\AgdaSpace{}%
\AgdaBound{B}\AgdaSymbol{\}}\AgdaSpace{}%
\AgdaSymbol{=}\AgdaSpace{}%
\AgdaInductiveConstructor{lam}\AgdaSpace{}%
\AgdaFunction{onenf}\<%
\\
%
\\[\AgdaEmptyExtraSkip]%
\>[0]\AgdaFunction{zeronf}\AgdaSpace{}%
\AgdaSymbol{:}\AgdaSpace{}%
\AgdaDatatype{Nf}\AgdaSpace{}%
\AgdaGeneralizable{Γ}\AgdaSpace{}%
\AgdaGeneralizable{A}\<%
\\
\>[0]\AgdaFunction{zeronf}\AgdaSpace{}%
\AgdaSymbol{\{}\AgdaBound{Γ}\AgdaSymbol{\}}\AgdaSpace{}%
\AgdaSymbol{\{}\AgdaInductiveConstructor{*}\AgdaSymbol{\}}\AgdaSpace{}%
\AgdaSymbol{=}\AgdaSpace{}%
\AgdaInductiveConstructor{ne}\AgdaSpace{}%
\AgdaSymbol{(}\AgdaFunction{⊥}\AgdaSpace{}%
\AgdaOperator{\AgdaInductiveConstructor{◃}}\AgdaSpace{}%
\AgdaSymbol{(λ}\AgdaSpace{}%
\AgdaBound{x}\AgdaSpace{}%
\AgdaSymbol{())}\AgdaSpace{}%
\AgdaOperator{\AgdaInductiveConstructor{◃}}\AgdaSpace{}%
\AgdaSymbol{(λ}\AgdaSpace{}%
\AgdaBound{x}\AgdaSpace{}%
\AgdaSymbol{()))}\<%
\\
\>[0]\AgdaFunction{zeronf}\AgdaSpace{}%
\AgdaSymbol{\{}\AgdaBound{Γ}\AgdaSymbol{\}}\AgdaSpace{}%
\AgdaSymbol{\{}\AgdaBound{A}\AgdaSpace{}%
\AgdaOperator{\AgdaInductiveConstructor{⇒}}\AgdaSpace{}%
\AgdaBound{B}\AgdaSymbol{\}}\AgdaSpace{}%
\AgdaSymbol{=}\AgdaSpace{}%
\AgdaInductiveConstructor{lam}\AgdaSpace{}%
\AgdaFunction{zeronf}\<%
\\
%
\\[\AgdaEmptyExtraSkip]%
\>[0]\AgdaFunction{Πnf}\AgdaSpace{}%
\AgdaSymbol{:}\AgdaSpace{}%
\AgdaSymbol{(}\AgdaBound{I}\AgdaSpace{}%
\AgdaSymbol{:}\AgdaSpace{}%
\AgdaFunction{Type}\AgdaSymbol{)}\AgdaSpace{}%
\AgdaSymbol{→}\AgdaSpace{}%
\AgdaSymbol{(}\AgdaBound{I}\AgdaSpace{}%
\AgdaSymbol{→}\AgdaSpace{}%
\AgdaDatatype{Nf}\AgdaSpace{}%
\AgdaGeneralizable{Γ}\AgdaSpace{}%
\AgdaGeneralizable{A}\AgdaSymbol{)}\AgdaSpace{}%
\AgdaSymbol{→}\AgdaSpace{}%
\AgdaDatatype{Nf}\AgdaSpace{}%
\AgdaGeneralizable{Γ}\AgdaSpace{}%
\AgdaGeneralizable{A}\<%
\\
\>[0]\AgdaFunction{Πnf}\AgdaSpace{}%
\AgdaSymbol{\{}\AgdaBound{Γ}\AgdaSymbol{\}}\AgdaSpace{}%
\AgdaSymbol{\{}\AgdaBound{A}\AgdaSpace{}%
\AgdaOperator{\AgdaInductiveConstructor{⇒}}\AgdaSpace{}%
\AgdaBound{B}\AgdaSymbol{\}}\AgdaSpace{}%
\AgdaBound{I}\AgdaSpace{}%
\AgdaBound{ts}\AgdaSpace{}%
\AgdaSymbol{=}\AgdaSpace{}%
\AgdaInductiveConstructor{lam}\AgdaSpace{}%
\AgdaSymbol{(}\AgdaFunction{Πnf}\AgdaSpace{}%
\AgdaBound{I}\AgdaSpace{}%
\AgdaSymbol{(λ}\AgdaSpace{}%
\AgdaBound{i}\AgdaSpace{}%
\AgdaSymbol{→}\AgdaSpace{}%
\AgdaFunction{app}\AgdaSpace{}%
\AgdaSymbol{(}\AgdaBound{ts}\AgdaSpace{}%
\AgdaBound{i}\AgdaSymbol{)))}\<%
\\
\>[0]\AgdaFunction{Πnf}\AgdaSpace{}%
\AgdaSymbol{\{}\AgdaBound{Γ}\AgdaSymbol{\}}\AgdaSpace{}%
\AgdaSymbol{\{}\AgdaInductiveConstructor{*}\AgdaSymbol{\}}\AgdaSpace{}%
\AgdaBound{I}\AgdaSpace{}%
\AgdaBound{ts}\AgdaSpace{}%
\AgdaSymbol{=}\AgdaSpace{}%
\AgdaInductiveConstructor{ne}\AgdaSpace{}%
\AgdaSymbol{(}\AgdaFunction{S}\AgdaSpace{}%
\AgdaOperator{\AgdaInductiveConstructor{◃}}\AgdaSpace{}%
\AgdaFunction{P}\AgdaSpace{}%
\AgdaOperator{\AgdaInductiveConstructor{◃}}\AgdaSpace{}%
\AgdaFunction{R}\AgdaSymbol{)}\<%
\\
\>[0][@{}l@{\AgdaIndent{0}}]%
\>[2]\AgdaKeyword{where}\<%
\\
%
\>[2]\AgdaFunction{S}\AgdaSpace{}%
\AgdaSymbol{:}\AgdaSpace{}%
\AgdaFunction{Type}\<%
\\
%
\>[2]\AgdaFunction{S}\AgdaSpace{}%
\AgdaSymbol{=}\AgdaSpace{}%
\AgdaSymbol{(}\AgdaBound{i}\AgdaSpace{}%
\AgdaSymbol{:}\AgdaSpace{}%
\AgdaBound{I}\AgdaSymbol{)}\AgdaSpace{}%
\AgdaSymbol{→}\AgdaSpace{}%
\AgdaFunction{en}\AgdaSpace{}%
\AgdaSymbol{(}\AgdaBound{ts}\AgdaSpace{}%
\AgdaBound{i}\AgdaSymbol{)}\AgdaSpace{}%
\AgdaSymbol{.}\AgdaField{Ne.S}\<%
\\
%
\\[\AgdaEmptyExtraSkip]%
%
\>[2]\AgdaFunction{P}\AgdaSpace{}%
\AgdaSymbol{:}\AgdaSpace{}%
\AgdaDatatype{Var}\AgdaSpace{}%
\AgdaBound{Γ}\AgdaSpace{}%
\AgdaGeneralizable{A}\AgdaSpace{}%
\AgdaSymbol{→}\AgdaSpace{}%
\AgdaFunction{S}\AgdaSpace{}%
\AgdaSymbol{→}\AgdaSpace{}%
\AgdaFunction{Type}\<%
\\
%
\>[2]\AgdaFunction{P}\AgdaSpace{}%
\AgdaBound{x}\AgdaSpace{}%
\AgdaBound{f}\AgdaSpace{}%
\AgdaSymbol{=}\AgdaSpace{}%
\AgdaFunction{Σ[}\AgdaSpace{}%
\AgdaBound{i}\AgdaSpace{}%
\AgdaFunction{∈}\AgdaSpace{}%
\AgdaBound{I}\AgdaSpace{}%
\AgdaFunction{]}\AgdaSpace{}%
\AgdaFunction{en}\AgdaSpace{}%
\AgdaSymbol{(}\AgdaBound{ts}\AgdaSpace{}%
\AgdaBound{i}\AgdaSymbol{)}\AgdaSpace{}%
\AgdaSymbol{.}\AgdaField{Ne.P}\AgdaSpace{}%
\AgdaBound{x}\AgdaSpace{}%
\AgdaSymbol{(}\AgdaBound{f}\AgdaSpace{}%
\AgdaBound{i}\AgdaSymbol{)}\<%
\\
%
\\[\AgdaEmptyExtraSkip]%
%
\>[2]\AgdaFunction{R}\AgdaSpace{}%
\AgdaSymbol{:}\AgdaSpace{}%
\AgdaSymbol{(}\AgdaBound{x}\AgdaSpace{}%
\AgdaSymbol{:}\AgdaSpace{}%
\AgdaDatatype{Var}\AgdaSpace{}%
\AgdaBound{Γ}\AgdaSpace{}%
\AgdaGeneralizable{A}\AgdaSymbol{)}\AgdaSpace{}%
\AgdaSymbol{(}\AgdaBound{s}\AgdaSpace{}%
\AgdaSymbol{:}\AgdaSpace{}%
\AgdaFunction{S}\AgdaSymbol{)}\AgdaSpace{}%
\AgdaSymbol{→}\AgdaSpace{}%
\AgdaFunction{P}\AgdaSpace{}%
\AgdaBound{x}\AgdaSpace{}%
\AgdaBound{s}\AgdaSpace{}%
\AgdaSymbol{→}\AgdaSpace{}%
\AgdaDatatype{Sp}\AgdaSpace{}%
\AgdaBound{Γ}\AgdaSpace{}%
\AgdaGeneralizable{A}\AgdaSpace{}%
\AgdaInductiveConstructor{*}\<%
\\
%
\>[2]\AgdaFunction{R}\AgdaSpace{}%
\AgdaBound{x}\AgdaSpace{}%
\AgdaBound{f}\AgdaSpace{}%
\AgdaSymbol{(}\AgdaBound{i}\AgdaSpace{}%
\AgdaOperator{\AgdaInductiveConstructor{,}}\AgdaSpace{}%
\AgdaBound{p}\AgdaSymbol{)}\AgdaSpace{}%
\AgdaSymbol{=}\AgdaSpace{}%
\AgdaFunction{en}\AgdaSpace{}%
\AgdaSymbol{(}\AgdaBound{ts}\AgdaSpace{}%
\AgdaBound{i}\AgdaSymbol{)}\AgdaSpace{}%
\AgdaSymbol{.}\AgdaField{Ne.R}\AgdaSpace{}%
\AgdaBound{x}\AgdaSpace{}%
\AgdaSymbol{(}\AgdaBound{f}\AgdaSpace{}%
\AgdaBound{i}\AgdaSymbol{)}\AgdaSpace{}%
\AgdaBound{p}\<%
\\
%
\\[\AgdaEmptyExtraSkip]%
\>[0]\AgdaFunction{Σnf}\AgdaSpace{}%
\AgdaSymbol{:}\AgdaSpace{}%
\AgdaSymbol{(}\AgdaBound{I}\AgdaSpace{}%
\AgdaSymbol{:}\AgdaSpace{}%
\AgdaFunction{Type}\AgdaSymbol{)}\AgdaSpace{}%
\AgdaSymbol{→}\AgdaSpace{}%
\AgdaSymbol{(}\AgdaBound{I}\AgdaSpace{}%
\AgdaSymbol{→}\AgdaSpace{}%
\AgdaDatatype{Nf}\AgdaSpace{}%
\AgdaGeneralizable{Γ}\AgdaSpace{}%
\AgdaGeneralizable{A}\AgdaSymbol{)}\AgdaSpace{}%
\AgdaSymbol{→}\AgdaSpace{}%
\AgdaDatatype{Nf}\AgdaSpace{}%
\AgdaGeneralizable{Γ}\AgdaSpace{}%
\AgdaGeneralizable{A}\<%
\\
\>[0]\AgdaFunction{Σnf}\AgdaSpace{}%
\AgdaSymbol{\{}\AgdaBound{Γ}\AgdaSymbol{\}}\AgdaSpace{}%
\AgdaSymbol{\{}\AgdaBound{A}\AgdaSpace{}%
\AgdaOperator{\AgdaInductiveConstructor{⇒}}\AgdaSpace{}%
\AgdaBound{B}\AgdaSymbol{\}}\AgdaSpace{}%
\AgdaBound{I}\AgdaSpace{}%
\AgdaBound{ts}\AgdaSpace{}%
\AgdaSymbol{=}\AgdaSpace{}%
\AgdaInductiveConstructor{lam}\AgdaSpace{}%
\AgdaSymbol{(}\AgdaFunction{Σnf}\AgdaSpace{}%
\AgdaBound{I}\AgdaSpace{}%
\AgdaSymbol{(λ}\AgdaSpace{}%
\AgdaBound{i}\AgdaSpace{}%
\AgdaSymbol{→}\AgdaSpace{}%
\AgdaFunction{app}\AgdaSpace{}%
\AgdaSymbol{(}\AgdaBound{ts}\AgdaSpace{}%
\AgdaBound{i}\AgdaSymbol{)))}\<%
\\
\>[0]\AgdaFunction{Σnf}\AgdaSpace{}%
\AgdaSymbol{\{}\AgdaBound{Γ}\AgdaSymbol{\}}\AgdaSpace{}%
\AgdaSymbol{\{}\AgdaInductiveConstructor{*}\AgdaSymbol{\}}\AgdaSpace{}%
\AgdaBound{I}\AgdaSpace{}%
\AgdaBound{ts}\AgdaSpace{}%
\AgdaSymbol{=}\AgdaSpace{}%
\AgdaInductiveConstructor{ne}\AgdaSpace{}%
\AgdaSymbol{(}\AgdaFunction{S}\AgdaSpace{}%
\AgdaOperator{\AgdaInductiveConstructor{◃}}\AgdaSpace{}%
\AgdaFunction{P}\AgdaSpace{}%
\AgdaOperator{\AgdaInductiveConstructor{◃}}\AgdaSpace{}%
\AgdaFunction{R}\AgdaSymbol{)}\<%
\\
\>[0][@{}l@{\AgdaIndent{0}}]%
\>[2]\AgdaKeyword{where}\<%
\\
%
\>[2]\AgdaFunction{S}\AgdaSpace{}%
\AgdaSymbol{:}\AgdaSpace{}%
\AgdaFunction{Type}\<%
\\
%
\>[2]\AgdaFunction{S}\AgdaSpace{}%
\AgdaSymbol{=}\AgdaSpace{}%
\AgdaFunction{Σ[}\AgdaSpace{}%
\AgdaBound{i}\AgdaSpace{}%
\AgdaFunction{∈}\AgdaSpace{}%
\AgdaBound{I}\AgdaSpace{}%
\AgdaFunction{]}\AgdaSpace{}%
\AgdaFunction{en}\AgdaSpace{}%
\AgdaSymbol{(}\AgdaBound{ts}\AgdaSpace{}%
\AgdaBound{i}\AgdaSymbol{)}\AgdaSpace{}%
\AgdaSymbol{.}\AgdaField{Ne.S}\<%
\\
%
\\[\AgdaEmptyExtraSkip]%
%
\>[2]\AgdaFunction{P}\AgdaSpace{}%
\AgdaSymbol{:}\AgdaSpace{}%
\AgdaDatatype{Var}\AgdaSpace{}%
\AgdaBound{Γ}\AgdaSpace{}%
\AgdaGeneralizable{A}\AgdaSpace{}%
\AgdaSymbol{→}\AgdaSpace{}%
\AgdaFunction{S}\AgdaSpace{}%
\AgdaSymbol{→}\AgdaSpace{}%
\AgdaFunction{Type}\<%
\\
%
\>[2]\AgdaFunction{P}\AgdaSpace{}%
\AgdaBound{x}\AgdaSpace{}%
\AgdaSymbol{(}\AgdaBound{i}\AgdaSpace{}%
\AgdaOperator{\AgdaInductiveConstructor{,}}\AgdaSpace{}%
\AgdaBound{s}\AgdaSymbol{)}\AgdaSpace{}%
\AgdaSymbol{=}\AgdaSpace{}%
\AgdaFunction{en}\AgdaSpace{}%
\AgdaSymbol{(}\AgdaBound{ts}\AgdaSpace{}%
\AgdaBound{i}\AgdaSymbol{)}\AgdaSpace{}%
\AgdaSymbol{.}\AgdaField{Ne.P}\AgdaSpace{}%
\AgdaBound{x}\AgdaSpace{}%
\AgdaBound{s}\<%
\\
%
\\[\AgdaEmptyExtraSkip]%
%
\>[2]\AgdaFunction{R}\AgdaSpace{}%
\AgdaSymbol{:}\AgdaSpace{}%
\AgdaSymbol{(}\AgdaBound{x}\AgdaSpace{}%
\AgdaSymbol{:}\AgdaSpace{}%
\AgdaDatatype{Var}\AgdaSpace{}%
\AgdaBound{Γ}\AgdaSpace{}%
\AgdaGeneralizable{A}\AgdaSymbol{)}\AgdaSpace{}%
\AgdaSymbol{(}\AgdaBound{s}\AgdaSpace{}%
\AgdaSymbol{:}\AgdaSpace{}%
\AgdaFunction{S}\AgdaSymbol{)}\AgdaSpace{}%
\AgdaSymbol{→}\AgdaSpace{}%
\AgdaFunction{P}\AgdaSpace{}%
\AgdaBound{x}\AgdaSpace{}%
\AgdaBound{s}\AgdaSpace{}%
\AgdaSymbol{→}\AgdaSpace{}%
\AgdaDatatype{Sp}\AgdaSpace{}%
\AgdaBound{Γ}\AgdaSpace{}%
\AgdaGeneralizable{A}\AgdaSpace{}%
\AgdaInductiveConstructor{*}\<%
\\
%
\>[2]\AgdaFunction{R}\AgdaSpace{}%
\AgdaBound{x}\AgdaSpace{}%
\AgdaSymbol{(}\AgdaBound{i}\AgdaSpace{}%
\AgdaOperator{\AgdaInductiveConstructor{,}}\AgdaSpace{}%
\AgdaBound{s}\AgdaSymbol{)}\AgdaSpace{}%
\AgdaBound{p}\AgdaSpace{}%
\AgdaSymbol{=}\AgdaSpace{}%
\AgdaFunction{en}\AgdaSpace{}%
\AgdaSymbol{(}\AgdaBound{ts}\AgdaSpace{}%
\AgdaBound{i}\AgdaSymbol{)}\AgdaSpace{}%
\AgdaSymbol{.}\AgdaField{Ne.R}\AgdaSpace{}%
\AgdaBound{x}\AgdaSpace{}%
\AgdaBound{s}\AgdaSpace{}%
\AgdaBound{p}\<%
\end{code}

\begin{code}[hide]%
\>[0]\AgdaKeyword{infix}\AgdaSpace{}%
\AgdaNumber{2}\AgdaSpace{}%
\AgdaFunction{Πnf-syntax}\<%
\\
\>[0]\AgdaFunction{Πnf-syntax}\AgdaSpace{}%
\AgdaSymbol{:}\AgdaSpace{}%
\AgdaSymbol{(}\AgdaBound{I}\AgdaSpace{}%
\AgdaSymbol{:}\AgdaSpace{}%
\AgdaFunction{Type}\AgdaSymbol{)}\AgdaSpace{}%
\AgdaSymbol{→}\AgdaSpace{}%
\AgdaSymbol{(}\AgdaBound{I}\AgdaSpace{}%
\AgdaSymbol{→}\AgdaSpace{}%
\AgdaDatatype{Nf}\AgdaSpace{}%
\AgdaGeneralizable{Γ}\AgdaSpace{}%
\AgdaGeneralizable{A}\AgdaSymbol{)}\AgdaSpace{}%
\AgdaSymbol{→}\AgdaSpace{}%
\AgdaDatatype{Nf}\AgdaSpace{}%
\AgdaGeneralizable{Γ}\AgdaSpace{}%
\AgdaGeneralizable{A}\<%
\\
\>[0]\AgdaFunction{Πnf-syntax}\AgdaSpace{}%
\AgdaSymbol{=}\AgdaSpace{}%
\AgdaFunction{Πnf}\<%
\\
\>[0]\AgdaKeyword{syntax}\AgdaSpace{}%
\AgdaFunction{Πnf-syntax}\AgdaSpace{}%
\AgdaBound{A}\AgdaSpace{}%
\AgdaSymbol{(λ}\AgdaSpace{}%
\AgdaBound{x}\AgdaSpace{}%
\AgdaSymbol{→}\AgdaSpace{}%
\AgdaBound{B}\AgdaSymbol{)}\AgdaSpace{}%
\AgdaSymbol{=}\AgdaSpace{}%
\AgdaFunction{Πnf[}\AgdaSpace{}%
\AgdaBound{x}\AgdaSpace{}%
\AgdaFunction{∈}\AgdaSpace{}%
\AgdaBound{A}\AgdaSpace{}%
\AgdaFunction{]}\AgdaSpace{}%
\AgdaBound{B}\<%
\\
%
\\[\AgdaEmptyExtraSkip]%
\>[0]\AgdaKeyword{infix}\AgdaSpace{}%
\AgdaNumber{2}\AgdaSpace{}%
\AgdaFunction{Σnf-syntax}\<%
\\
\>[0]\AgdaFunction{Σnf-syntax}\AgdaSpace{}%
\AgdaSymbol{:}\AgdaSpace{}%
\AgdaSymbol{(}\AgdaBound{I}\AgdaSpace{}%
\AgdaSymbol{:}\AgdaSpace{}%
\AgdaFunction{Type}\AgdaSymbol{)}\AgdaSpace{}%
\AgdaSymbol{→}\AgdaSpace{}%
\AgdaSymbol{(}\AgdaBound{I}\AgdaSpace{}%
\AgdaSymbol{→}\AgdaSpace{}%
\AgdaDatatype{Nf}\AgdaSpace{}%
\AgdaGeneralizable{Γ}\AgdaSpace{}%
\AgdaGeneralizable{A}\AgdaSymbol{)}\AgdaSpace{}%
\AgdaSymbol{→}\AgdaSpace{}%
\AgdaDatatype{Nf}\AgdaSpace{}%
\AgdaGeneralizable{Γ}\AgdaSpace{}%
\AgdaGeneralizable{A}\<%
\\
\>[0]\AgdaFunction{Σnf-syntax}\AgdaSpace{}%
\AgdaSymbol{=}\AgdaSpace{}%
\AgdaFunction{Πnf}\<%
\\
\>[0]\AgdaKeyword{syntax}\AgdaSpace{}%
\AgdaFunction{Σnf-syntax}\AgdaSpace{}%
\AgdaBound{A}\AgdaSpace{}%
\AgdaSymbol{(λ}\AgdaSpace{}%
\AgdaBound{x}\AgdaSpace{}%
\AgdaSymbol{→}\AgdaSpace{}%
\AgdaBound{B}\AgdaSymbol{)}\AgdaSpace{}%
\AgdaSymbol{=}\AgdaSpace{}%
\AgdaFunction{Σnf[}\AgdaSpace{}%
\AgdaBound{x}\AgdaSpace{}%
\AgdaFunction{∈}\AgdaSpace{}%
\AgdaBound{A}\AgdaSpace{}%
\AgdaFunction{]}\AgdaSpace{}%
\AgdaBound{B}\<%
\end{code}

\subsection{Semantics}

We need to first interpret types, contexts and variables:

\begin{code}%
\>[0]\AgdaOperator{\AgdaFunction{⟦\AgdaUnderscore{}⟧t}}\AgdaSpace{}%
\AgdaSymbol{:}\AgdaSpace{}%
\AgdaDatatype{Ty}\AgdaSpace{}%
\AgdaSymbol{→}\AgdaSpace{}%
\AgdaFunction{Type₁}\<%
\\
\>[0]\AgdaOperator{\AgdaFunction{⟦}}\AgdaSpace{}%
\AgdaInductiveConstructor{*}\AgdaSpace{}%
\AgdaOperator{\AgdaFunction{⟧t}}\AgdaSpace{}%
\AgdaSymbol{=}\AgdaSpace{}%
\AgdaFunction{Type}\<%
\\
\>[0]\AgdaOperator{\AgdaFunction{⟦}}\AgdaSpace{}%
\AgdaBound{A}\AgdaSpace{}%
\AgdaOperator{\AgdaInductiveConstructor{⇒}}\AgdaSpace{}%
\AgdaBound{B}\AgdaSpace{}%
\AgdaOperator{\AgdaFunction{⟧t}}\AgdaSpace{}%
\AgdaSymbol{=}\AgdaSpace{}%
\AgdaOperator{\AgdaFunction{⟦}}\AgdaSpace{}%
\AgdaBound{A}\AgdaSpace{}%
\AgdaOperator{\AgdaFunction{⟧t}}\AgdaSpace{}%
\AgdaSymbol{→}\AgdaSpace{}%
\AgdaOperator{\AgdaFunction{⟦}}\AgdaSpace{}%
\AgdaBound{B}\AgdaSpace{}%
\AgdaOperator{\AgdaFunction{⟧t}}\<%
\\
%
\\[\AgdaEmptyExtraSkip]%
\>[0]\AgdaOperator{\AgdaFunction{⟦\AgdaUnderscore{}⟧c}}\AgdaSpace{}%
\AgdaSymbol{:}\AgdaSpace{}%
\AgdaDatatype{Con}\AgdaSpace{}%
\AgdaSymbol{→}\AgdaSpace{}%
\AgdaFunction{Type₁}\<%
\\
\>[0]\AgdaOperator{\AgdaFunction{⟦}}\AgdaSpace{}%
\AgdaInductiveConstructor{∙}\AgdaSpace{}%
\AgdaOperator{\AgdaFunction{⟧c}}\AgdaSpace{}%
\AgdaSymbol{=}\AgdaSpace{}%
\AgdaRecord{Lift}\AgdaSpace{}%
\AgdaSymbol{(}\AgdaPrimitive{lsuc}\AgdaSpace{}%
\AgdaPrimitive{lzero}\AgdaSymbol{)}\AgdaSpace{}%
\AgdaRecord{⊤}\<%
\\
\>[0]\AgdaOperator{\AgdaFunction{⟦}}\AgdaSpace{}%
\AgdaBound{Γ}\AgdaSpace{}%
\AgdaOperator{\AgdaInductiveConstructor{▹}}\AgdaSpace{}%
\AgdaBound{A}\AgdaSpace{}%
\AgdaOperator{\AgdaFunction{⟧c}}\AgdaSpace{}%
\AgdaSymbol{=}\AgdaSpace{}%
\AgdaOperator{\AgdaFunction{⟦}}\AgdaSpace{}%
\AgdaBound{Γ}\AgdaSpace{}%
\AgdaOperator{\AgdaFunction{⟧c}}\AgdaSpace{}%
\AgdaOperator{\AgdaFunction{×}}\AgdaSpace{}%
\AgdaOperator{\AgdaFunction{⟦}}\AgdaSpace{}%
\AgdaBound{A}\AgdaSpace{}%
\AgdaOperator{\AgdaFunction{⟧t}}\<%
\\
%
\\[\AgdaEmptyExtraSkip]%
\>[0]\AgdaOperator{\AgdaFunction{⟦\AgdaUnderscore{}⟧v}}\AgdaSpace{}%
\AgdaSymbol{:}\AgdaSpace{}%
\AgdaDatatype{Var}\AgdaSpace{}%
\AgdaGeneralizable{Γ}\AgdaSpace{}%
\AgdaGeneralizable{A}\AgdaSpace{}%
\AgdaSymbol{→}\AgdaSpace{}%
\AgdaOperator{\AgdaFunction{⟦}}\AgdaSpace{}%
\AgdaGeneralizable{Γ}\AgdaSpace{}%
\AgdaOperator{\AgdaFunction{⟧c}}\AgdaSpace{}%
\AgdaSymbol{→}\AgdaSpace{}%
\AgdaOperator{\AgdaFunction{⟦}}\AgdaSpace{}%
\AgdaGeneralizable{A}\AgdaSpace{}%
\AgdaOperator{\AgdaFunction{⟧t}}\<%
\\
\>[0]\AgdaOperator{\AgdaFunction{⟦}}\AgdaSpace{}%
\AgdaInductiveConstructor{vz}\AgdaSpace{}%
\AgdaOperator{\AgdaFunction{⟧v}}\AgdaSpace{}%
\AgdaSymbol{(}\AgdaBound{as}\AgdaSpace{}%
\AgdaOperator{\AgdaInductiveConstructor{,}}\AgdaSpace{}%
\AgdaBound{a}\AgdaSymbol{)}\AgdaSpace{}%
\AgdaSymbol{=}\AgdaSpace{}%
\AgdaBound{a}\<%
\\
\>[0]\AgdaOperator{\AgdaFunction{⟦}}\AgdaSpace{}%
\AgdaInductiveConstructor{vs}\AgdaSpace{}%
\AgdaBound{x}\AgdaSpace{}%
\AgdaOperator{\AgdaFunction{⟧v}}\AgdaSpace{}%
\AgdaSymbol{(}\AgdaBound{as}\AgdaSpace{}%
\AgdaOperator{\AgdaInductiveConstructor{,}}\AgdaSpace{}%
\AgdaBound{a}\AgdaSymbol{)}\AgdaSpace{}%
\AgdaSymbol{=}\AgdaSpace{}%
\AgdaOperator{\AgdaFunction{⟦}}\AgdaSpace{}%
\AgdaBound{x}\AgdaSpace{}%
\AgdaOperator{\AgdaFunction{⟧v}}\AgdaSpace{}%
\AgdaBound{as}\<%
\end{code}

The semantics of normal forms:

\begin{code}[hide]%
\>[0]\AgdaKeyword{mutual}\<%
\end{code}

\begin{code}%
\>[0][@{}l@{\AgdaIndent{1}}]%
\>[2]\AgdaOperator{\AgdaFunction{⟦\AgdaUnderscore{}⟧nf}}\AgdaSpace{}%
\AgdaSymbol{:}\AgdaSpace{}%
\AgdaDatatype{Nf}\AgdaSpace{}%
\AgdaGeneralizable{Γ}\AgdaSpace{}%
\AgdaGeneralizable{A}\AgdaSpace{}%
\AgdaSymbol{→}\AgdaSpace{}%
\AgdaOperator{\AgdaFunction{⟦}}\AgdaSpace{}%
\AgdaGeneralizable{Γ}\AgdaSpace{}%
\AgdaOperator{\AgdaFunction{⟧c}}\AgdaSpace{}%
\AgdaSymbol{→}\AgdaSpace{}%
\AgdaOperator{\AgdaFunction{⟦}}\AgdaSpace{}%
\AgdaGeneralizable{A}\AgdaSpace{}%
\AgdaOperator{\AgdaFunction{⟧t}}\<%
\\
%
\>[2]\AgdaOperator{\AgdaFunction{⟦}}\AgdaSpace{}%
\AgdaInductiveConstructor{lam}\AgdaSpace{}%
\AgdaBound{t}\AgdaSpace{}%
\AgdaOperator{\AgdaFunction{⟧nf}}\AgdaSpace{}%
\AgdaBound{as}\AgdaSpace{}%
\AgdaBound{a}\AgdaSpace{}%
\AgdaSymbol{=}\AgdaSpace{}%
\AgdaOperator{\AgdaFunction{⟦}}\AgdaSpace{}%
\AgdaBound{t}\AgdaSpace{}%
\AgdaOperator{\AgdaFunction{⟧nf}}\AgdaSpace{}%
\AgdaSymbol{(}\AgdaBound{as}\AgdaSpace{}%
\AgdaOperator{\AgdaInductiveConstructor{,}}\AgdaSpace{}%
\AgdaBound{a}\AgdaSymbol{)}\<%
\\
%
\>[2]\AgdaOperator{\AgdaFunction{⟦}}\AgdaSpace{}%
\AgdaInductiveConstructor{ne}\AgdaSpace{}%
\AgdaBound{spr}\AgdaSpace{}%
\AgdaOperator{\AgdaFunction{⟧nf}}\AgdaSpace{}%
\AgdaBound{as}\AgdaSpace{}%
\AgdaSymbol{=}\AgdaSpace{}%
\AgdaOperator{\AgdaFunction{⟦}}\AgdaSpace{}%
\AgdaBound{spr}\AgdaSpace{}%
\AgdaOperator{\AgdaFunction{⟧ne}}\AgdaSpace{}%
\AgdaBound{as}\<%
\\
%
\\[\AgdaEmptyExtraSkip]%
%
\>[2]\AgdaOperator{\AgdaFunction{⟦\AgdaUnderscore{}⟧ne}}\AgdaSpace{}%
\AgdaSymbol{:}\AgdaSpace{}%
\AgdaRecord{Ne}\AgdaSpace{}%
\AgdaGeneralizable{Γ}\AgdaSpace{}%
\AgdaInductiveConstructor{*}\AgdaSpace{}%
\AgdaSymbol{→}\AgdaSpace{}%
\AgdaOperator{\AgdaFunction{⟦}}\AgdaSpace{}%
\AgdaGeneralizable{Γ}\AgdaSpace{}%
\AgdaOperator{\AgdaFunction{⟧c}}\AgdaSpace{}%
\AgdaSymbol{→}\AgdaSpace{}%
\AgdaFunction{Type}\<%
\\
%
\>[2]\AgdaOperator{\AgdaFunction{⟦\AgdaUnderscore{}⟧ne}}\AgdaSpace{}%
\AgdaSymbol{\{}\AgdaBound{Γ}\AgdaSymbol{\}}\AgdaSpace{}%
\AgdaSymbol{(}\AgdaBound{S}\AgdaSpace{}%
\AgdaOperator{\AgdaInductiveConstructor{◃}}\AgdaSpace{}%
\AgdaBound{P}\AgdaSpace{}%
\AgdaOperator{\AgdaInductiveConstructor{◃}}\AgdaSpace{}%
\AgdaBound{R}\AgdaSymbol{)}\AgdaSpace{}%
\AgdaBound{as}\AgdaSpace{}%
\AgdaSymbol{=}\<%
\\
\>[2][@{}l@{\AgdaIndent{0}}]%
\>[4]\AgdaFunction{Σ[}\AgdaSpace{}%
\AgdaBound{s}\AgdaSpace{}%
\AgdaFunction{∈}\AgdaSpace{}%
\AgdaBound{S}\AgdaSpace{}%
\AgdaFunction{]}\AgdaSpace{}%
\AgdaSymbol{(\{}\AgdaBound{A}\AgdaSpace{}%
\AgdaSymbol{:}\AgdaSpace{}%
\AgdaDatatype{Ty}\AgdaSymbol{\}}\AgdaSpace{}%
\AgdaSymbol{(}\AgdaBound{x}\AgdaSpace{}%
\AgdaSymbol{:}\AgdaSpace{}%
\AgdaDatatype{Var}\AgdaSpace{}%
\AgdaBound{Γ}\AgdaSpace{}%
\AgdaBound{A}\AgdaSymbol{)}\AgdaSpace{}%
\AgdaSymbol{(}\AgdaBound{p}\AgdaSpace{}%
\AgdaSymbol{:}\AgdaSpace{}%
\AgdaBound{P}\AgdaSpace{}%
\AgdaBound{x}\AgdaSpace{}%
\AgdaBound{s}\AgdaSymbol{)}\<%
\\
\>[4][@{}l@{\AgdaIndent{0}}]%
\>[6]\AgdaSymbol{→}\AgdaSpace{}%
\AgdaOperator{\AgdaFunction{⟦}}\AgdaSpace{}%
\AgdaBound{R}\AgdaSpace{}%
\AgdaBound{x}\AgdaSpace{}%
\AgdaBound{s}\AgdaSpace{}%
\AgdaBound{p}\AgdaSpace{}%
\AgdaOperator{\AgdaFunction{⟧sp}}\AgdaSpace{}%
\AgdaBound{as}\AgdaSpace{}%
\AgdaSymbol{(}\AgdaOperator{\AgdaFunction{⟦}}\AgdaSpace{}%
\AgdaBound{x}\AgdaSpace{}%
\AgdaOperator{\AgdaFunction{⟧v}}\AgdaSpace{}%
\AgdaBound{as}\AgdaSymbol{))}\<%
\\
%
\\[\AgdaEmptyExtraSkip]%
%
\>[2]\AgdaOperator{\AgdaFunction{⟦\AgdaUnderscore{}⟧sp}}\AgdaSpace{}%
\AgdaSymbol{:}\AgdaSpace{}%
\AgdaDatatype{Sp}\AgdaSpace{}%
\AgdaGeneralizable{Γ}\AgdaSpace{}%
\AgdaGeneralizable{A}\AgdaSpace{}%
\AgdaGeneralizable{B}\AgdaSpace{}%
\AgdaSymbol{→}\AgdaSpace{}%
\AgdaOperator{\AgdaFunction{⟦}}\AgdaSpace{}%
\AgdaGeneralizable{Γ}\AgdaSpace{}%
\AgdaOperator{\AgdaFunction{⟧c}}\AgdaSpace{}%
\AgdaSymbol{→}\AgdaSpace{}%
\AgdaOperator{\AgdaFunction{⟦}}\AgdaSpace{}%
\AgdaGeneralizable{A}\AgdaSpace{}%
\AgdaOperator{\AgdaFunction{⟧t}}\AgdaSpace{}%
\AgdaSymbol{→}\AgdaSpace{}%
\AgdaOperator{\AgdaFunction{⟦}}\AgdaSpace{}%
\AgdaGeneralizable{B}\AgdaSpace{}%
\AgdaOperator{\AgdaFunction{⟧t}}\<%
\\
%
\>[2]\AgdaOperator{\AgdaFunction{⟦}}\AgdaSpace{}%
\AgdaInductiveConstructor{ε}\AgdaSpace{}%
\AgdaOperator{\AgdaFunction{⟧sp}}\AgdaSpace{}%
\AgdaBound{as}\AgdaSpace{}%
\AgdaBound{a}\AgdaSpace{}%
\AgdaSymbol{=}\AgdaSpace{}%
\AgdaBound{a}\<%
\\
%
\>[2]\AgdaOperator{\AgdaFunction{⟦}}\AgdaSpace{}%
\AgdaBound{t}\AgdaSpace{}%
\AgdaOperator{\AgdaInductiveConstructor{,}}\AgdaSpace{}%
\AgdaBound{ts}\AgdaSpace{}%
\AgdaOperator{\AgdaFunction{⟧sp}}\AgdaSpace{}%
\AgdaBound{as}\AgdaSpace{}%
\AgdaBound{f}\AgdaSpace{}%
\AgdaSymbol{=}\AgdaSpace{}%
\AgdaOperator{\AgdaFunction{⟦}}\AgdaSpace{}%
\AgdaBound{ts}\AgdaSpace{}%
\AgdaOperator{\AgdaFunction{⟧sp}}\AgdaSpace{}%
\AgdaBound{as}\AgdaSpace{}%
\AgdaSymbol{(}\AgdaBound{f}\AgdaSpace{}%
\AgdaSymbol{(}\AgdaOperator{\AgdaFunction{⟦}}\AgdaSpace{}%
\AgdaBound{t}\AgdaSpace{}%
\AgdaOperator{\AgdaFunction{⟧nf}}\AgdaSpace{}%
\AgdaBound{as}\AgdaSymbol{))}\<%
\end{code}

\subsection{Simply Typed Categories with Families}

We show that the higher containers get rise to a SCwF with functions, where \texttt{Ty} is \texttt{Ty}, \texttt{Con} is \texttt{Con}, and \texttt{Tm} is \texttt{Nf}. We need to define the syntax for context substitutions.

\begin{code}%
\>[0]\AgdaKeyword{data}\AgdaSpace{}%
\AgdaDatatype{Nfs}\AgdaSpace{}%
\AgdaSymbol{:}\AgdaSpace{}%
\AgdaDatatype{Con}\AgdaSpace{}%
\AgdaSymbol{→}\AgdaSpace{}%
\AgdaDatatype{Con}\AgdaSpace{}%
\AgdaSymbol{→}\AgdaSpace{}%
\AgdaFunction{Type₁}\AgdaSpace{}%
\AgdaKeyword{where}\<%
\\
\>[0][@{}l@{\AgdaIndent{0}}]%
\>[2]\AgdaInductiveConstructor{ε}%
\>[6]\AgdaSymbol{:}\AgdaSpace{}%
\AgdaDatatype{Nfs}\AgdaSpace{}%
\AgdaGeneralizable{Γ}\AgdaSpace{}%
\AgdaInductiveConstructor{∙}\<%
\\
%
\>[2]\AgdaOperator{\AgdaInductiveConstructor{\AgdaUnderscore{},\AgdaUnderscore{}}}\AgdaSpace{}%
\AgdaSymbol{:}\AgdaSpace{}%
\AgdaDatatype{Nfs}\AgdaSpace{}%
\AgdaGeneralizable{Δ}\AgdaSpace{}%
\AgdaGeneralizable{Γ}\AgdaSpace{}%
\AgdaSymbol{→}\AgdaSpace{}%
\AgdaDatatype{Nf}\AgdaSpace{}%
\AgdaGeneralizable{Δ}\AgdaSpace{}%
\AgdaGeneralizable{A}\AgdaSpace{}%
\AgdaSymbol{→}\AgdaSpace{}%
\AgdaDatatype{Nfs}\AgdaSpace{}%
\AgdaGeneralizable{Δ}\AgdaSpace{}%
\AgdaSymbol{(}\AgdaGeneralizable{Γ}\AgdaSpace{}%
\AgdaOperator{\AgdaInductiveConstructor{▹}}\AgdaSpace{}%
\AgdaGeneralizable{A}\AgdaSymbol{)}\<%
\end{code}

\begin{code}[hide]%
\>[0]\AgdaKeyword{variable}\AgdaSpace{}%
\AgdaGeneralizable{γ}\AgdaSpace{}%
\AgdaGeneralizable{δ}\AgdaSpace{}%
\AgdaGeneralizable{θ}\AgdaSpace{}%
\AgdaSymbol{:}\AgdaSpace{}%
\AgdaDatatype{Nfs}\AgdaSpace{}%
\AgdaGeneralizable{Δ}\AgdaSpace{}%
\AgdaGeneralizable{Γ}\<%
\end{code}

Similar to \texttt{Tms}, the \texttt{Nfs} is a list of normal forms.

\begin{code}[hide]%
\>[0]\AgdaFunction{wkNfs}\AgdaSpace{}%
\AgdaSymbol{:}\AgdaSpace{}%
\AgdaSymbol{(}\AgdaBound{x}\AgdaSpace{}%
\AgdaSymbol{:}\AgdaSpace{}%
\AgdaDatatype{Var}\AgdaSpace{}%
\AgdaGeneralizable{Δ}\AgdaSpace{}%
\AgdaGeneralizable{A}\AgdaSymbol{)}\AgdaSpace{}%
\AgdaSymbol{→}\AgdaSpace{}%
\AgdaDatatype{Nfs}\AgdaSpace{}%
\AgdaSymbol{(}\AgdaGeneralizable{Δ}\AgdaSpace{}%
\AgdaOperator{\AgdaFunction{-}}\AgdaSpace{}%
\AgdaBound{x}\AgdaSymbol{)}\AgdaSpace{}%
\AgdaGeneralizable{Γ}\AgdaSpace{}%
\AgdaSymbol{→}\AgdaSpace{}%
\AgdaDatatype{Nfs}\AgdaSpace{}%
\AgdaGeneralizable{Δ}\AgdaSpace{}%
\AgdaGeneralizable{Γ}\<%
\\
\>[0]\AgdaFunction{wkNfs}\AgdaSpace{}%
\AgdaBound{x}\AgdaSpace{}%
\AgdaInductiveConstructor{ε}\AgdaSpace{}%
\AgdaSymbol{=}\AgdaSpace{}%
\AgdaInductiveConstructor{ε}\<%
\\
\>[0]\AgdaFunction{wkNfs}\AgdaSpace{}%
\AgdaBound{x}\AgdaSpace{}%
\AgdaSymbol{(}\AgdaBound{γ}\AgdaSpace{}%
\AgdaOperator{\AgdaInductiveConstructor{,}}\AgdaSpace{}%
\AgdaBound{t}\AgdaSymbol{)}\AgdaSpace{}%
\AgdaSymbol{=}\AgdaSpace{}%
\AgdaFunction{wkNfs}\AgdaSpace{}%
\AgdaBound{x}\AgdaSpace{}%
\AgdaBound{γ}\AgdaSpace{}%
\AgdaOperator{\AgdaInductiveConstructor{,}}\AgdaSpace{}%
\AgdaFunction{wkNf}\AgdaSpace{}%
\AgdaBound{x}\AgdaSpace{}%
\AgdaBound{t}\<%
\\
%
\\[\AgdaEmptyExtraSkip]%
\>[0]\AgdaOperator{\AgdaFunction{\AgdaUnderscore{}↑}}\AgdaSpace{}%
\AgdaSymbol{:}\AgdaSpace{}%
\AgdaDatatype{Nfs}\AgdaSpace{}%
\AgdaGeneralizable{Δ}\AgdaSpace{}%
\AgdaGeneralizable{Γ}\AgdaSpace{}%
\AgdaSymbol{→}\AgdaSpace{}%
\AgdaDatatype{Nfs}\AgdaSpace{}%
\AgdaSymbol{(}\AgdaGeneralizable{Δ}\AgdaSpace{}%
\AgdaOperator{\AgdaInductiveConstructor{▹}}\AgdaSpace{}%
\AgdaGeneralizable{A}\AgdaSymbol{)}\AgdaSpace{}%
\AgdaSymbol{(}\AgdaGeneralizable{Γ}\AgdaSpace{}%
\AgdaOperator{\AgdaInductiveConstructor{▹}}\AgdaSpace{}%
\AgdaGeneralizable{A}\AgdaSymbol{)}\<%
\\
\>[0]\AgdaBound{γ}\AgdaSpace{}%
\AgdaOperator{\AgdaFunction{↑}}\AgdaSpace{}%
\AgdaSymbol{=}\AgdaSpace{}%
\AgdaFunction{wkNfs}\AgdaSpace{}%
\AgdaInductiveConstructor{vz}\AgdaSpace{}%
\AgdaBound{γ}\AgdaSpace{}%
\AgdaOperator{\AgdaInductiveConstructor{,}}\AgdaSpace{}%
\AgdaFunction{nvar}\AgdaSpace{}%
\AgdaInductiveConstructor{vz}\<%
\\
%
\\[\AgdaEmptyExtraSkip]%
\>[0]\AgdaFunction{id}\AgdaSpace{}%
\AgdaSymbol{:}\AgdaSpace{}%
\AgdaDatatype{Nfs}\AgdaSpace{}%
\AgdaGeneralizable{Γ}\AgdaSpace{}%
\AgdaGeneralizable{Γ}\<%
\\
\>[0]\AgdaFunction{id}\AgdaSpace{}%
\AgdaSymbol{\{}\AgdaInductiveConstructor{∙}\AgdaSymbol{\}}\AgdaSpace{}%
\AgdaSymbol{=}\AgdaSpace{}%
\AgdaInductiveConstructor{ε}\<%
\\
\>[0]\AgdaFunction{id}\AgdaSpace{}%
\AgdaSymbol{\{}\AgdaBound{Γ}\AgdaSpace{}%
\AgdaOperator{\AgdaInductiveConstructor{▹}}\AgdaSpace{}%
\AgdaBound{A}\AgdaSymbol{\}}\AgdaSpace{}%
\AgdaSymbol{=}\AgdaSpace{}%
\AgdaFunction{id}\AgdaSpace{}%
\AgdaOperator{\AgdaFunction{↑}}\<%
\\
%
\\[\AgdaEmptyExtraSkip]%
\>[0]\AgdaFunction{subVar}\AgdaSpace{}%
\AgdaSymbol{:}\AgdaSpace{}%
\AgdaDatatype{Var}\AgdaSpace{}%
\AgdaGeneralizable{Γ}\AgdaSpace{}%
\AgdaGeneralizable{A}\AgdaSpace{}%
\AgdaSymbol{→}\AgdaSpace{}%
\AgdaDatatype{Nfs}\AgdaSpace{}%
\AgdaGeneralizable{Δ}\AgdaSpace{}%
\AgdaGeneralizable{Γ}\AgdaSpace{}%
\AgdaSymbol{→}\AgdaSpace{}%
\AgdaDatatype{Nf}\AgdaSpace{}%
\AgdaGeneralizable{Δ}\AgdaSpace{}%
\AgdaGeneralizable{A}\<%
\\
\>[0]\AgdaFunction{subVar}\AgdaSpace{}%
\AgdaInductiveConstructor{vz}\AgdaSpace{}%
\AgdaSymbol{(}\AgdaBound{γ}\AgdaSpace{}%
\AgdaOperator{\AgdaInductiveConstructor{,}}\AgdaSpace{}%
\AgdaBound{t}\AgdaSymbol{)}\AgdaSpace{}%
\AgdaSymbol{=}\AgdaSpace{}%
\AgdaBound{t}\<%
\\
\>[0]\AgdaFunction{subVar}\AgdaSpace{}%
\AgdaSymbol{(}\AgdaInductiveConstructor{vs}\AgdaSpace{}%
\AgdaBound{x}\AgdaSymbol{)}\AgdaSpace{}%
\AgdaSymbol{(}\AgdaBound{γ}\AgdaSpace{}%
\AgdaOperator{\AgdaInductiveConstructor{,}}\AgdaSpace{}%
\AgdaBound{t}\AgdaSymbol{)}\AgdaSpace{}%
\AgdaSymbol{=}\AgdaSpace{}%
\AgdaFunction{subVar}\AgdaSpace{}%
\AgdaBound{x}\AgdaSpace{}%
\AgdaBound{γ}\<%
\\
%
\\[\AgdaEmptyExtraSkip]%
\>[0]\AgdaFunction{foldSp}\AgdaSpace{}%
\AgdaSymbol{:}\AgdaSpace{}%
\AgdaDatatype{Nf}\AgdaSpace{}%
\AgdaGeneralizable{Γ}\AgdaSpace{}%
\AgdaGeneralizable{A}\AgdaSpace{}%
\AgdaSymbol{→}\AgdaSpace{}%
\AgdaDatatype{Sp}\AgdaSpace{}%
\AgdaGeneralizable{Γ}\AgdaSpace{}%
\AgdaGeneralizable{A}\AgdaSpace{}%
\AgdaGeneralizable{B}\AgdaSpace{}%
\AgdaSymbol{→}\AgdaSpace{}%
\AgdaDatatype{Nf}\AgdaSpace{}%
\AgdaGeneralizable{Γ}\AgdaSpace{}%
\AgdaGeneralizable{B}\<%
\\
\>[0]\AgdaFunction{foldSp}\AgdaSpace{}%
\AgdaBound{t}\AgdaSpace{}%
\AgdaInductiveConstructor{ε}\AgdaSpace{}%
\AgdaSymbol{=}\AgdaSpace{}%
\AgdaBound{t}\<%
\\
\>[0]\AgdaFunction{foldSp}\AgdaSpace{}%
\AgdaBound{t}\AgdaSpace{}%
\AgdaSymbol{(}\AgdaBound{u}\AgdaSpace{}%
\AgdaOperator{\AgdaInductiveConstructor{,}}\AgdaSpace{}%
\AgdaBound{us}\AgdaSymbol{)}\AgdaSpace{}%
\AgdaSymbol{=}\AgdaSpace{}%
\AgdaFunction{foldSp}\AgdaSpace{}%
\AgdaSymbol{(}\AgdaFunction{napp}\AgdaSpace{}%
\AgdaBound{t}\AgdaSpace{}%
\AgdaBound{u}\AgdaSymbol{)}\AgdaSpace{}%
\AgdaBound{us}\<%
\end{code}

\begin{code}%
\>[0]\AgdaKeyword{mutual}\<%
\\
\>[0][@{}l@{\AgdaIndent{0}}]%
\>[2]\AgdaOperator{\AgdaFunction{\AgdaUnderscore{}[\AgdaUnderscore{}]}}\AgdaSpace{}%
\AgdaSymbol{:}\AgdaSpace{}%
\AgdaDatatype{Nf}\AgdaSpace{}%
\AgdaGeneralizable{Γ}\AgdaSpace{}%
\AgdaGeneralizable{A}\AgdaSpace{}%
\AgdaSymbol{→}\AgdaSpace{}%
\AgdaDatatype{Nfs}\AgdaSpace{}%
\AgdaGeneralizable{Δ}\AgdaSpace{}%
\AgdaGeneralizable{Γ}\AgdaSpace{}%
\AgdaSymbol{→}\AgdaSpace{}%
\AgdaDatatype{Nf}\AgdaSpace{}%
\AgdaGeneralizable{Δ}\AgdaSpace{}%
\AgdaGeneralizable{A}\<%
\\
%
\>[2]\AgdaInductiveConstructor{lam}\AgdaSpace{}%
\AgdaBound{t}\AgdaSpace{}%
\AgdaOperator{\AgdaFunction{[}}\AgdaSpace{}%
\AgdaBound{γ}\AgdaSpace{}%
\AgdaOperator{\AgdaFunction{]}}\AgdaSpace{}%
\AgdaSymbol{=}\AgdaSpace{}%
\AgdaInductiveConstructor{lam}\AgdaSpace{}%
\AgdaSymbol{(}\AgdaBound{t}\AgdaSpace{}%
\AgdaOperator{\AgdaFunction{[}}\AgdaSpace{}%
\AgdaBound{γ}\AgdaSpace{}%
\AgdaOperator{\AgdaFunction{↑}}\AgdaSpace{}%
\AgdaOperator{\AgdaFunction{]}}\AgdaSymbol{)}\<%
\\
%
\>[2]\AgdaInductiveConstructor{ne}\AgdaSpace{}%
\AgdaBound{spr}\AgdaSpace{}%
\AgdaOperator{\AgdaFunction{[}}\AgdaSpace{}%
\AgdaBound{γ}\AgdaSpace{}%
\AgdaOperator{\AgdaFunction{]}}\AgdaSpace{}%
\AgdaSymbol{=}\AgdaSpace{}%
\AgdaBound{spr}\AgdaSpace{}%
\AgdaOperator{\AgdaFunction{[}}\AgdaSpace{}%
\AgdaBound{γ}\AgdaSpace{}%
\AgdaOperator{\AgdaFunction{]ne2nf}}\<%
\\
%
\\[\AgdaEmptyExtraSkip]%
%
\>[2]\AgdaOperator{\AgdaFunction{\AgdaUnderscore{}[\AgdaUnderscore{}]ne2nf}}\AgdaSpace{}%
\AgdaSymbol{:}\AgdaSpace{}%
\AgdaRecord{Ne}\AgdaSpace{}%
\AgdaGeneralizable{Γ}\AgdaSpace{}%
\AgdaGeneralizable{B}\AgdaSpace{}%
\AgdaSymbol{→}\AgdaSpace{}%
\AgdaDatatype{Nfs}\AgdaSpace{}%
\AgdaGeneralizable{Δ}\AgdaSpace{}%
\AgdaGeneralizable{Γ}\AgdaSpace{}%
\AgdaSymbol{→}\AgdaSpace{}%
\AgdaDatatype{Nf}\AgdaSpace{}%
\AgdaGeneralizable{Δ}\AgdaSpace{}%
\AgdaGeneralizable{B}\<%
\\
%
\>[2]\AgdaOperator{\AgdaFunction{\AgdaUnderscore{}[\AgdaUnderscore{}]ne2nf}}\AgdaSpace{}%
\AgdaSymbol{\{}\AgdaBound{Γ}\AgdaSymbol{\}}\AgdaSpace{}%
\AgdaSymbol{\{}\AgdaBound{B}\AgdaSymbol{\}}\AgdaSpace{}%
\AgdaSymbol{\{}\AgdaBound{Δ}\AgdaSymbol{\}}\AgdaSpace{}%
\AgdaSymbol{(}\AgdaBound{S}\AgdaSpace{}%
\AgdaOperator{\AgdaInductiveConstructor{◃}}\AgdaSpace{}%
\AgdaBound{P}\AgdaSpace{}%
\AgdaOperator{\AgdaInductiveConstructor{◃}}\AgdaSpace{}%
\AgdaBound{R}\AgdaSymbol{)}\AgdaSpace{}%
\AgdaBound{γ}\AgdaSpace{}%
\AgdaSymbol{=}\<%
\\
\>[2][@{}l@{\AgdaIndent{0}}]%
\>[4]\AgdaFunction{Σnf[}\AgdaSpace{}%
\AgdaBound{s}\AgdaSpace{}%
\AgdaFunction{∈}\AgdaSpace{}%
\AgdaBound{S}\AgdaSpace{}%
\AgdaFunction{]}\AgdaSpace{}%
\AgdaFunction{Πnf[}\AgdaSpace{}%
\AgdaBound{A}\AgdaSpace{}%
\AgdaFunction{∈}\AgdaSpace{}%
\AgdaDatatype{Ty}\AgdaSpace{}%
\AgdaFunction{]}\AgdaSpace{}%
\AgdaFunction{Πnf[}\AgdaSpace{}%
\AgdaBound{x}\AgdaSpace{}%
\AgdaFunction{∈}\AgdaSpace{}%
\AgdaDatatype{Var}\AgdaSpace{}%
\AgdaBound{Γ}\AgdaSpace{}%
\AgdaBound{A}\AgdaSpace{}%
\AgdaFunction{]}\AgdaSpace{}%
\AgdaFunction{Πnf[}\AgdaSpace{}%
\AgdaBound{p}\AgdaSpace{}%
\AgdaFunction{∈}\AgdaSpace{}%
\AgdaBound{P}\AgdaSpace{}%
\AgdaBound{x}\AgdaSpace{}%
\AgdaBound{s}\AgdaSpace{}%
\AgdaFunction{]}\<%
\\
%
\>[4]\AgdaFunction{foldSp}\AgdaSpace{}%
\AgdaSymbol{\{}\AgdaBound{Δ}\AgdaSymbol{\}}\AgdaSpace{}%
\AgdaSymbol{\{}\AgdaBound{A}\AgdaSymbol{\}}\AgdaSpace{}%
\AgdaSymbol{(}\AgdaFunction{subVar}\AgdaSpace{}%
\AgdaBound{x}\AgdaSpace{}%
\AgdaBound{γ}\AgdaSymbol{)}\AgdaSpace{}%
\AgdaSymbol{(}\AgdaBound{R}\AgdaSpace{}%
\AgdaBound{x}\AgdaSpace{}%
\AgdaBound{s}\AgdaSpace{}%
\AgdaBound{p}\AgdaSpace{}%
\AgdaOperator{\AgdaFunction{[}}\AgdaSpace{}%
\AgdaBound{γ}\AgdaSpace{}%
\AgdaOperator{\AgdaFunction{]sp}}\AgdaSymbol{)}\<%
\\
%
\\[\AgdaEmptyExtraSkip]%
%
\>[2]\AgdaOperator{\AgdaFunction{\AgdaUnderscore{}[\AgdaUnderscore{}]sp}}\AgdaSpace{}%
\AgdaSymbol{:}\AgdaSpace{}%
\AgdaDatatype{Sp}\AgdaSpace{}%
\AgdaGeneralizable{Γ}\AgdaSpace{}%
\AgdaGeneralizable{A}\AgdaSpace{}%
\AgdaGeneralizable{B}\AgdaSpace{}%
\AgdaSymbol{→}\AgdaSpace{}%
\AgdaDatatype{Nfs}\AgdaSpace{}%
\AgdaGeneralizable{Δ}\AgdaSpace{}%
\AgdaGeneralizable{Γ}\AgdaSpace{}%
\AgdaSymbol{→}\AgdaSpace{}%
\AgdaDatatype{Sp}\AgdaSpace{}%
\AgdaGeneralizable{Δ}\AgdaSpace{}%
\AgdaGeneralizable{A}\AgdaSpace{}%
\AgdaGeneralizable{B}\<%
\\
%
\>[2]\AgdaInductiveConstructor{ε}\AgdaSpace{}%
\AgdaOperator{\AgdaFunction{[}}\AgdaSpace{}%
\AgdaBound{γ}\AgdaSpace{}%
\AgdaOperator{\AgdaFunction{]sp}}\AgdaSpace{}%
\AgdaSymbol{=}\AgdaSpace{}%
\AgdaInductiveConstructor{ε}\<%
\\
%
\>[2]\AgdaSymbol{(}\AgdaBound{t}\AgdaSpace{}%
\AgdaOperator{\AgdaInductiveConstructor{,}}\AgdaSpace{}%
\AgdaBound{ts}\AgdaSymbol{)}\AgdaSpace{}%
\AgdaOperator{\AgdaFunction{[}}\AgdaSpace{}%
\AgdaBound{γ}\AgdaSpace{}%
\AgdaOperator{\AgdaFunction{]sp}}\AgdaSpace{}%
\AgdaSymbol{=}\AgdaSpace{}%
\AgdaSymbol{(}\AgdaBound{t}\AgdaSpace{}%
\AgdaOperator{\AgdaFunction{[}}\AgdaSpace{}%
\AgdaBound{γ}\AgdaSpace{}%
\AgdaOperator{\AgdaFunction{]}}\AgdaSymbol{)}\AgdaSpace{}%
\AgdaOperator{\AgdaInductiveConstructor{,}}\AgdaSpace{}%
\AgdaSymbol{(}\AgdaBound{ts}\AgdaSpace{}%
\AgdaOperator{\AgdaFunction{[}}\AgdaSpace{}%
\AgdaBound{γ}\AgdaSpace{}%
\AgdaOperator{\AgdaFunction{]sp}}\AgdaSymbol{)}\<%
\end{code}
\chapter{Conclusions}

This is conclusions.

\chapter{Future Work Plan}

\begin{code}[hide]%
\>[0]\AgdaSymbol{\{-\#}\AgdaSpace{}%
\AgdaKeyword{OPTIONS}\AgdaSpace{}%
\AgdaPragma{--allow-unsolved-metas}\AgdaSpace{}%
\AgdaSymbol{\#-\}}\<%
\\
\>[0]\AgdaKeyword{module}\AgdaSpace{}%
\AgdaModule{future}\AgdaSpace{}%
\AgdaKeyword{where}\<%
\\
%
\\[\AgdaEmptyExtraSkip]%
\>[0]\AgdaKeyword{open}\AgdaSpace{}%
\AgdaKeyword{import}\AgdaSpace{}%
\AgdaModule{Relation.Binary.PropositionalEquality}\<%
\\
\>[0]\AgdaKeyword{open}\AgdaSpace{}%
\AgdaKeyword{import}\AgdaSpace{}%
\AgdaModule{outcomes}\<%
\\
%
\\[\AgdaEmptyExtraSkip]%
\>[0]\AgdaKeyword{infix}\AgdaSpace{}%
\AgdaNumber{2}\AgdaSpace{}%
\AgdaFunction{Πtm-syntax}\<%
\\
\>[0]\AgdaFunction{Πtm-syntax}\AgdaSpace{}%
\AgdaSymbol{:}\AgdaSpace{}%
\AgdaSymbol{(}\AgdaBound{I}\AgdaSpace{}%
\AgdaSymbol{:}\AgdaSpace{}%
\AgdaFunction{Type}\AgdaSymbol{)}\AgdaSpace{}%
\AgdaSymbol{→}\AgdaSpace{}%
\AgdaSymbol{(}\AgdaBound{I}\AgdaSpace{}%
\AgdaSymbol{→}\AgdaSpace{}%
\AgdaDatatype{Tm}\AgdaSpace{}%
\AgdaGeneralizable{Γ}\AgdaSpace{}%
\AgdaGeneralizable{A}\AgdaSymbol{)}\AgdaSpace{}%
\AgdaSymbol{→}\AgdaSpace{}%
\AgdaDatatype{Tm}\AgdaSpace{}%
\AgdaGeneralizable{Γ}\AgdaSpace{}%
\AgdaGeneralizable{A}\<%
\\
\>[0]\AgdaFunction{Πtm-syntax}\AgdaSpace{}%
\AgdaSymbol{=}\AgdaSpace{}%
\AgdaInductiveConstructor{Πtm}\<%
\\
\>[0]\AgdaKeyword{syntax}\AgdaSpace{}%
\AgdaFunction{Πtm-syntax}\AgdaSpace{}%
\AgdaBound{A}\AgdaSpace{}%
\AgdaSymbol{(λ}\AgdaSpace{}%
\AgdaBound{x}\AgdaSpace{}%
\AgdaSymbol{→}\AgdaSpace{}%
\AgdaBound{B}\AgdaSymbol{)}\AgdaSpace{}%
\AgdaSymbol{=}\AgdaSpace{}%
\AgdaFunction{Πtm[}\AgdaSpace{}%
\AgdaBound{x}\AgdaSpace{}%
\AgdaFunction{∈}\AgdaSpace{}%
\AgdaBound{A}\AgdaSpace{}%
\AgdaFunction{]}\AgdaSpace{}%
\AgdaBound{B}\<%
\\
%
\\[\AgdaEmptyExtraSkip]%
\>[0]\AgdaKeyword{infix}\AgdaSpace{}%
\AgdaNumber{2}\AgdaSpace{}%
\AgdaFunction{Σtm-syntax}\<%
\\
\>[0]\AgdaFunction{Σtm-syntax}\AgdaSpace{}%
\AgdaSymbol{:}\AgdaSpace{}%
\AgdaSymbol{(}\AgdaBound{I}\AgdaSpace{}%
\AgdaSymbol{:}\AgdaSpace{}%
\AgdaFunction{Type}\AgdaSymbol{)}\AgdaSpace{}%
\AgdaSymbol{→}\AgdaSpace{}%
\AgdaSymbol{(}\AgdaBound{I}\AgdaSpace{}%
\AgdaSymbol{→}\AgdaSpace{}%
\AgdaDatatype{Tm}\AgdaSpace{}%
\AgdaGeneralizable{Γ}\AgdaSpace{}%
\AgdaGeneralizable{A}\AgdaSymbol{)}\AgdaSpace{}%
\AgdaSymbol{→}\AgdaSpace{}%
\AgdaDatatype{Tm}\AgdaSpace{}%
\AgdaGeneralizable{Γ}\AgdaSpace{}%
\AgdaGeneralizable{A}\<%
\\
\>[0]\AgdaFunction{Σtm-syntax}\AgdaSpace{}%
\AgdaSymbol{=}\AgdaSpace{}%
\AgdaInductiveConstructor{Σtm}\<%
\\
\>[0]\AgdaKeyword{syntax}\AgdaSpace{}%
\AgdaFunction{Σtm-syntax}\AgdaSpace{}%
\AgdaBound{A}\AgdaSpace{}%
\AgdaSymbol{(λ}\AgdaSpace{}%
\AgdaBound{x}\AgdaSpace{}%
\AgdaSymbol{→}\AgdaSpace{}%
\AgdaBound{B}\AgdaSymbol{)}\AgdaSpace{}%
\AgdaSymbol{=}\AgdaSpace{}%
\AgdaFunction{Σtm[}\AgdaSpace{}%
\AgdaBound{x}\AgdaSpace{}%
\AgdaFunction{∈}\AgdaSpace{}%
\AgdaBound{A}\AgdaSpace{}%
\AgdaFunction{]}\AgdaSpace{}%
\AgdaBound{B}\<%
\end{code}

\subsubsection*{Full Formalization}

There are many things left to do in our current research. The primary object at this stage is to finish the constructions and proofs as we promised before. As a summary, we wish to:

\begin{itemize}
  \item{construct the functorial meaning of higher containers}
  \item{fully interpret higher containers to higher functors}
  \item{define the semantics of morphisms of higher containers}
  \item{prove the universal properties of categorical objects}
  \item{finish the proofs of higher containers as a model of simply typed \lambda-calculus}
\end{itemize}

Moreover, current formalization is carried out in plain Agda. We wish to strictly replicate the whole formalization with all h-level in Cubical Agda.

\subsubsection*{Soundness and Completeness}

Another objective is to prove the completeness and soundness properties of the normalizer $nf$. To do that, we first define an embedding from $Nf(\Gamma,A)$ to $Tm(\Gamma,A)$:

\begin{code}[hide]%
\>[0]\AgdaKeyword{mutual}\<%
\end{code}

\subsubsection*{Embedding}

\begin{code}%
\>[0][@{}l@{\AgdaIndent{1}}]%
\>[2]\AgdaFunction{embNf}\AgdaSpace{}%
\AgdaSymbol{:}\AgdaSpace{}%
\AgdaDatatype{Nf}\AgdaSpace{}%
\AgdaGeneralizable{Γ}\AgdaSpace{}%
\AgdaGeneralizable{A}\AgdaSpace{}%
\AgdaSymbol{→}\AgdaSpace{}%
\AgdaDatatype{Tm}\AgdaSpace{}%
\AgdaGeneralizable{Γ}\AgdaSpace{}%
\AgdaGeneralizable{A}\<%
\\
%
\>[2]\AgdaFunction{embNf}\AgdaSpace{}%
\AgdaSymbol{(}\AgdaInductiveConstructor{lam}\AgdaSpace{}%
\AgdaBound{t}\AgdaSymbol{)}\AgdaSpace{}%
\AgdaSymbol{=}\AgdaSpace{}%
\AgdaInductiveConstructor{lam}\AgdaSpace{}%
\AgdaSymbol{(}\AgdaFunction{embNf}\AgdaSpace{}%
\AgdaBound{t}\AgdaSymbol{)}\<%
\\
%
\>[2]\AgdaFunction{embNf}\AgdaSpace{}%
\AgdaSymbol{(}\AgdaInductiveConstructor{ne}\AgdaSpace{}%
\AgdaBound{spr}\AgdaSymbol{)}\AgdaSpace{}%
\AgdaSymbol{=}\AgdaSpace{}%
\AgdaFunction{embNe}\AgdaSpace{}%
\AgdaBound{spr}\<%
\\
%
\\[\AgdaEmptyExtraSkip]%
%
\>[2]\AgdaFunction{embNe}\AgdaSpace{}%
\AgdaSymbol{:}\AgdaSpace{}%
\AgdaRecord{Ne}\AgdaSpace{}%
\AgdaGeneralizable{Γ}\AgdaSpace{}%
\AgdaGeneralizable{A}\AgdaSpace{}%
\AgdaSymbol{→}\AgdaSpace{}%
\AgdaDatatype{Tm}\AgdaSpace{}%
\AgdaGeneralizable{Γ}\AgdaSpace{}%
\AgdaGeneralizable{A}\<%
\\
%
\>[2]\AgdaFunction{embNe}\AgdaSpace{}%
\AgdaSymbol{(}\AgdaBound{S}\AgdaSpace{}%
\AgdaOperator{\AgdaInductiveConstructor{◃}}\AgdaSpace{}%
\AgdaBound{P}\AgdaSpace{}%
\AgdaOperator{\AgdaInductiveConstructor{◃}}\AgdaSpace{}%
\AgdaBound{R}\AgdaSymbol{)}\AgdaSpace{}%
\AgdaSymbol{=}\AgdaSpace{}%
\AgdaFunction{Σtm[}\AgdaSpace{}%
\AgdaBound{s}\AgdaSpace{}%
\AgdaFunction{∈}\AgdaSpace{}%
\AgdaBound{S}\AgdaSpace{}%
\AgdaFunction{]}\<%
\\
\>[2][@{}l@{\AgdaIndent{0}}]%
\>[4]\AgdaFunction{Πtm[}\AgdaSpace{}%
\AgdaBound{A}\AgdaSpace{}%
\AgdaFunction{∈}\AgdaSpace{}%
\AgdaDatatype{Ty}\AgdaSpace{}%
\AgdaFunction{]}\AgdaSpace{}%
\AgdaFunction{Πtm[}\AgdaSpace{}%
\AgdaBound{x}\AgdaSpace{}%
\AgdaFunction{∈}\AgdaSpace{}%
\AgdaDatatype{Var}\AgdaSpace{}%
\AgdaSymbol{\AgdaUnderscore{}}\AgdaSpace{}%
\AgdaBound{A}\AgdaSpace{}%
\AgdaFunction{]}\AgdaSpace{}%
\AgdaFunction{Πtm[}\AgdaSpace{}%
\AgdaBound{p}\AgdaSpace{}%
\AgdaFunction{∈}\AgdaSpace{}%
\AgdaBound{P}\AgdaSpace{}%
\AgdaBound{x}\AgdaSpace{}%
\AgdaBound{s}\AgdaSpace{}%
\AgdaFunction{]}\<%
\\
%
\>[4]\AgdaFunction{embSp}\AgdaSpace{}%
\AgdaSymbol{(}\AgdaBound{R}\AgdaSpace{}%
\AgdaBound{x}\AgdaSpace{}%
\AgdaBound{s}\AgdaSpace{}%
\AgdaBound{p}\AgdaSymbol{)}\AgdaSpace{}%
\AgdaSymbol{(}\AgdaInductiveConstructor{var}\AgdaSpace{}%
\AgdaBound{x}\AgdaSymbol{)}\<%
\\
%
\\[\AgdaEmptyExtraSkip]%
%
\>[2]\AgdaFunction{embSp}\AgdaSpace{}%
\AgdaSymbol{:}\AgdaSpace{}%
\AgdaDatatype{Sp}\AgdaSpace{}%
\AgdaGeneralizable{Γ}\AgdaSpace{}%
\AgdaGeneralizable{A}\AgdaSpace{}%
\AgdaGeneralizable{B}\AgdaSpace{}%
\AgdaSymbol{→}\AgdaSpace{}%
\AgdaDatatype{Tm}\AgdaSpace{}%
\AgdaGeneralizable{Γ}\AgdaSpace{}%
\AgdaGeneralizable{A}\AgdaSpace{}%
\AgdaSymbol{→}\AgdaSpace{}%
\AgdaDatatype{Tm}\AgdaSpace{}%
\AgdaGeneralizable{Γ}\AgdaSpace{}%
\AgdaGeneralizable{B}\<%
\\
%
\>[2]\AgdaFunction{embSp}\AgdaSpace{}%
\AgdaInductiveConstructor{ε}\AgdaSpace{}%
\AgdaBound{u}\AgdaSpace{}%
\AgdaSymbol{=}\AgdaSpace{}%
\AgdaBound{u}\<%
\\
%
\>[2]\AgdaFunction{embSp}\AgdaSpace{}%
\AgdaSymbol{(}\AgdaBound{t}\AgdaSpace{}%
\AgdaOperator{\AgdaInductiveConstructor{,}}\AgdaSpace{}%
\AgdaBound{ts}\AgdaSymbol{)}\AgdaSpace{}%
\AgdaBound{u}\AgdaSpace{}%
\AgdaSymbol{=}\AgdaSpace{}%
\AgdaFunction{embSp}\AgdaSpace{}%
\AgdaBound{ts}\AgdaSpace{}%
\AgdaSymbol{(}\AgdaBound{u}\AgdaSpace{}%
\AgdaOperator{\AgdaInductiveConstructor{\$}}\AgdaSpace{}%
\AgdaFunction{embNf}\AgdaSpace{}%
\AgdaBound{t}\AgdaSymbol{)}\<%
\end{code}

\begin{code}[hide]%
\>[0]\AgdaKeyword{data}\AgdaSpace{}%
\AgdaOperator{\AgdaDatatype{\AgdaUnderscore{}βη≡\AgdaUnderscore{}}}\AgdaSpace{}%
\AgdaSymbol{:}\AgdaSpace{}%
\AgdaDatatype{Tm}\AgdaSpace{}%
\AgdaGeneralizable{Γ}\AgdaSpace{}%
\AgdaGeneralizable{A}\AgdaSpace{}%
\AgdaSymbol{→}\AgdaSpace{}%
\AgdaDatatype{Tm}\AgdaSpace{}%
\AgdaGeneralizable{Γ}\AgdaSpace{}%
\AgdaGeneralizable{A}\AgdaSpace{}%
\AgdaSymbol{→}\AgdaSpace{}%
\AgdaFunction{Type}\AgdaSpace{}%
\AgdaKeyword{where}\<%
\end{code}

Here, the completeness means the embedding of any normalized term is \beta\eta-equivalent to itself:

\begin{code}%
\>[0]\AgdaFunction{completeness}\AgdaSpace{}%
\AgdaSymbol{:}\AgdaSpace{}%
\AgdaSymbol{(}\AgdaBound{t}\AgdaSpace{}%
\AgdaSymbol{:}\AgdaSpace{}%
\AgdaDatatype{Tm}\AgdaSpace{}%
\AgdaGeneralizable{Γ}\AgdaSpace{}%
\AgdaGeneralizable{A}\AgdaSymbol{)}\AgdaSpace{}%
\AgdaSymbol{→}\AgdaSpace{}%
\AgdaFunction{embNf}\AgdaSpace{}%
\AgdaSymbol{(}\AgdaFunction{nf}\AgdaSpace{}%
\AgdaBound{t}\AgdaSymbol{)}\AgdaSpace{}%
\AgdaOperator{\AgdaDatatype{βη≡}}\AgdaSpace{}%
\AgdaBound{t}\<%
\end{code}

where the \beta\eta-equivalence should be a relation of \lambda-terms which reflects the convertibility under \beta-equivalence and \eta-equivalence. We need to characterize this relation as well.

The soundness says, if two terms are convertible, then they should reduce to the same canonical normal forms:

\begin{code}%
\>[0]\AgdaFunction{soundness}\AgdaSpace{}%
\AgdaSymbol{:}\AgdaSpace{}%
\AgdaSymbol{(}\AgdaBound{t}\AgdaSpace{}%
\AgdaBound{u}\AgdaSpace{}%
\AgdaSymbol{:}\AgdaSpace{}%
\AgdaDatatype{Tm}\AgdaSpace{}%
\AgdaGeneralizable{Γ}\AgdaSpace{}%
\AgdaGeneralizable{A}\AgdaSymbol{)}\AgdaSpace{}%
\AgdaSymbol{→}\AgdaSpace{}%
\AgdaBound{t}\AgdaSpace{}%
\AgdaOperator{\AgdaDatatype{βη≡}}\AgdaSpace{}%
\AgdaBound{u}\AgdaSpace{}%
\AgdaSymbol{→}\AgdaSpace{}%
\AgdaFunction{nf}\AgdaSpace{}%
\AgdaBound{t}\AgdaSpace{}%
\AgdaOperator{\AgdaDatatype{≡}}\AgdaSpace{}%
\AgdaFunction{nf}\AgdaSpace{}%
\AgdaBound{u}\<%
\end{code}

\begin{code}[hide]%
\>[0]\AgdaFunction{completeness}\AgdaSpace{}%
\AgdaSymbol{=}\AgdaSpace{}%
\AgdaHole{\{!\ \ \ !\}}\<%
\\
\>[0]\AgdaFunction{soundness}\AgdaSpace{}%
\AgdaSymbol{=}\AgdaSpace{}%
\AgdaHole{\{!\ \ \ !\}}\<%
\end{code}

Combining two results together, we wish to show that the ``convertibility of terms is decidable''.
\chapter{Appendix}

\section{Agda Code - Higher-Order Containers Examples}

\begin{code}[hide]%
\>[0]\AgdaKeyword{open}\AgdaSpace{}%
\AgdaKeyword{import}\AgdaSpace{}%
\AgdaModule{Level}\AgdaSpace{}%
\AgdaKeyword{renaming}\AgdaSpace{}%
\AgdaSymbol{(}\AgdaPrimitive{zero}\AgdaSpace{}%
\AgdaSymbol{to}\AgdaSpace{}%
\AgdaPrimitive{lzero}\AgdaSymbol{;}\AgdaSpace{}%
\AgdaPrimitive{suc}\AgdaSpace{}%
\AgdaSymbol{to}\AgdaSpace{}%
\AgdaPrimitive{lsuc}\AgdaSymbol{)}\<%
\\
\>[0]\AgdaKeyword{open}\AgdaSpace{}%
\AgdaKeyword{import}\AgdaSpace{}%
\AgdaModule{Data.Empty}\<%
\\
\>[0]\AgdaKeyword{open}\AgdaSpace{}%
\AgdaKeyword{import}\AgdaSpace{}%
\AgdaModule{Data.Unit}\<%
\\
\>[0]\AgdaKeyword{open}\AgdaSpace{}%
\AgdaKeyword{import}\AgdaSpace{}%
\AgdaModule{Data.Sum}\<%
\\
\>[0]\AgdaKeyword{open}\AgdaSpace{}%
\AgdaKeyword{import}\AgdaSpace{}%
\AgdaModule{Data.Product}\<%
\\
\>[0]\AgdaKeyword{open}\AgdaSpace{}%
\AgdaKeyword{import}\AgdaSpace{}%
\AgdaModule{Function.Base}\AgdaSpace{}%
\AgdaKeyword{hiding}\AgdaSpace{}%
\AgdaSymbol{(}\AgdaOperator{\AgdaFunction{\AgdaUnderscore{}\$\AgdaUnderscore{}}}\AgdaSymbol{)}\<%
\\
\>[0]\AgdaKeyword{open}\AgdaSpace{}%
\AgdaKeyword{import}\AgdaSpace{}%
\AgdaModule{Relation.Binary.PropositionalEquality}\AgdaSpace{}%
\AgdaKeyword{hiding}\AgdaSpace{}%
\AgdaSymbol{(}\AgdaOperator{\AgdaInductiveConstructor{[\AgdaUnderscore{}]}}\AgdaSymbol{)}\<%
\\
%
\\[\AgdaEmptyExtraSkip]%
\>[0]\AgdaKeyword{open}\AgdaSpace{}%
\AgdaKeyword{import}\AgdaSpace{}%
\AgdaModule{outcomes}\<%
\end{code}

\subsubsection*{Morphisms}\label{Hom}

\begin{code}%
\>[0]\AgdaFunction{idNfHom}\AgdaSpace{}%
\AgdaSymbol{:}\AgdaSpace{}%
\AgdaDatatype{NfHom}\AgdaSpace{}%
\AgdaGeneralizable{t}\AgdaSpace{}%
\AgdaGeneralizable{t}\<%
\\
\>[0]\AgdaFunction{idNfHom}\AgdaSpace{}%
\AgdaSymbol{\{}\AgdaArgument{t}\AgdaSpace{}%
\AgdaSymbol{=}\AgdaSpace{}%
\AgdaInductiveConstructor{ne}\AgdaSpace{}%
\AgdaBound{spr}\AgdaSymbol{\}}\AgdaSpace{}%
\AgdaSymbol{=}\AgdaSpace{}%
\AgdaInductiveConstructor{ne}\AgdaSpace{}%
\AgdaSymbol{(}\AgdaFunction{id}\AgdaSpace{}%
\AgdaOperator{\AgdaInductiveConstructor{◃}}\AgdaSpace{}%
\AgdaSymbol{(λ}\AgdaSpace{}%
\AgdaBound{x}\AgdaSpace{}%
\AgdaBound{s}\AgdaSpace{}%
\AgdaSymbol{→}\AgdaSpace{}%
\AgdaFunction{id}\AgdaSymbol{)}\AgdaSpace{}%
\AgdaOperator{\AgdaInductiveConstructor{◃}}\AgdaSpace{}%
\AgdaSymbol{λ}\AgdaSpace{}%
\AgdaBound{x}\AgdaSpace{}%
\AgdaBound{s}\AgdaSpace{}%
\AgdaBound{q}\AgdaSpace{}%
\AgdaSymbol{→}\AgdaSpace{}%
\AgdaInductiveConstructor{ε}\AgdaSymbol{)}\<%
\\
\>[0]\AgdaFunction{idNfHom}\AgdaSpace{}%
\AgdaSymbol{\{}\AgdaArgument{t}\AgdaSpace{}%
\AgdaSymbol{=}\AgdaSpace{}%
\AgdaInductiveConstructor{lam}\AgdaSpace{}%
\AgdaBound{t}\AgdaSymbol{\}}\AgdaSpace{}%
\AgdaSymbol{=}\AgdaSpace{}%
\AgdaInductiveConstructor{lam}\AgdaSpace{}%
\AgdaSymbol{(}\AgdaFunction{idNfHom}\AgdaSpace{}%
\AgdaSymbol{\{}\AgdaArgument{t}\AgdaSpace{}%
\AgdaSymbol{=}\AgdaSpace{}%
\AgdaBound{t}\AgdaSymbol{\})}\<%
\\
%
\\[\AgdaEmptyExtraSkip]%
\>[0]\AgdaOperator{\AgdaFunction{\AgdaUnderscore{}∘nfHom\AgdaUnderscore{}}}\AgdaSpace{}%
\AgdaSymbol{:}\AgdaSpace{}%
\AgdaDatatype{NfHom}\AgdaSpace{}%
\AgdaGeneralizable{u}\AgdaSpace{}%
\AgdaGeneralizable{w}\AgdaSpace{}%
\AgdaSymbol{→}\AgdaSpace{}%
\AgdaDatatype{NfHom}\AgdaSpace{}%
\AgdaGeneralizable{t}\AgdaSpace{}%
\AgdaGeneralizable{u}\AgdaSpace{}%
\AgdaSymbol{→}\AgdaSpace{}%
\AgdaDatatype{NfHom}\AgdaSpace{}%
\AgdaGeneralizable{t}\AgdaSpace{}%
\AgdaGeneralizable{w}\<%
\\
\>[0]\AgdaOperator{\AgdaFunction{\AgdaUnderscore{}∘spHom\AgdaUnderscore{}}}\AgdaSpace{}%
\AgdaSymbol{:}\AgdaSpace{}%
\AgdaDatatype{SpHom}\AgdaSpace{}%
\AgdaGeneralizable{us}\AgdaSpace{}%
\AgdaGeneralizable{ws}\AgdaSpace{}%
\AgdaSymbol{→}\AgdaSpace{}%
\AgdaDatatype{SpHom}\AgdaSpace{}%
\AgdaGeneralizable{ts}\AgdaSpace{}%
\AgdaGeneralizable{us}\AgdaSpace{}%
\AgdaSymbol{→}\AgdaSpace{}%
\AgdaDatatype{SpHom}\AgdaSpace{}%
\AgdaGeneralizable{ts}\AgdaSpace{}%
\AgdaGeneralizable{ws}\<%
\\
%
\\[\AgdaEmptyExtraSkip]%
\>[0]\AgdaInductiveConstructor{lam}\AgdaSpace{}%
\AgdaBound{f}\AgdaSpace{}%
\AgdaOperator{\AgdaFunction{∘nfHom}}\AgdaSpace{}%
\AgdaInductiveConstructor{lam}\AgdaSpace{}%
\AgdaBound{g}\AgdaSpace{}%
\AgdaSymbol{=}\AgdaSpace{}%
\AgdaInductiveConstructor{lam}\AgdaSpace{}%
\AgdaSymbol{(}\AgdaBound{f}\AgdaSpace{}%
\AgdaOperator{\AgdaFunction{∘nfHom}}\AgdaSpace{}%
\AgdaBound{g}\AgdaSymbol{)}\<%
\\
\>[0]\AgdaInductiveConstructor{ne}\AgdaSpace{}%
\AgdaSymbol{(}\AgdaBound{f}\AgdaSpace{}%
\AgdaOperator{\AgdaInductiveConstructor{◃}}\AgdaSpace{}%
\AgdaBound{g}\AgdaSpace{}%
\AgdaOperator{\AgdaInductiveConstructor{◃}}\AgdaSpace{}%
\AgdaBound{h}\AgdaSymbol{)}\AgdaSpace{}%
\AgdaOperator{\AgdaFunction{∘nfHom}}\AgdaSpace{}%
\AgdaInductiveConstructor{ne}\AgdaSpace{}%
\AgdaSymbol{(}\AgdaBound{f'}\AgdaSpace{}%
\AgdaOperator{\AgdaInductiveConstructor{◃}}\AgdaSpace{}%
\AgdaBound{g'}\AgdaSpace{}%
\AgdaOperator{\AgdaInductiveConstructor{◃}}\AgdaSpace{}%
\AgdaBound{h'}\AgdaSymbol{)}\AgdaSpace{}%
\AgdaSymbol{=}\AgdaSpace{}%
\AgdaInductiveConstructor{ne}\AgdaSpace{}%
\AgdaSymbol{(}\<%
\\
\>[0][@{}l@{\AgdaIndent{0}}]%
\>[2]\AgdaSymbol{(}\AgdaBound{f}\AgdaSpace{}%
\AgdaOperator{\AgdaFunction{∘}}\AgdaSpace{}%
\AgdaBound{f'}\AgdaSymbol{)}\<%
\\
%
\>[2]\AgdaOperator{\AgdaInductiveConstructor{◃}}\AgdaSpace{}%
\AgdaSymbol{(λ}\AgdaSpace{}%
\AgdaBound{x}\AgdaSpace{}%
\AgdaBound{s}\AgdaSpace{}%
\AgdaSymbol{→}\AgdaSpace{}%
\AgdaBound{g'}\AgdaSpace{}%
\AgdaBound{x}\AgdaSpace{}%
\AgdaBound{s}\AgdaSpace{}%
\AgdaOperator{\AgdaFunction{∘}}\AgdaSpace{}%
\AgdaBound{g}\AgdaSpace{}%
\AgdaBound{x}\AgdaSpace{}%
\AgdaSymbol{(}\AgdaBound{f'}\AgdaSpace{}%
\AgdaBound{s}\AgdaSymbol{))}\<%
\\
%
\>[2]\AgdaOperator{\AgdaInductiveConstructor{◃}}\AgdaSpace{}%
\AgdaSymbol{λ}\AgdaSpace{}%
\AgdaBound{x}\AgdaSpace{}%
\AgdaBound{s}\AgdaSpace{}%
\AgdaBound{q}\AgdaSpace{}%
\AgdaSymbol{→}\AgdaSpace{}%
\AgdaSymbol{(}\AgdaBound{h}\AgdaSpace{}%
\AgdaBound{x}\AgdaSpace{}%
\AgdaSymbol{(}\AgdaBound{f'}\AgdaSpace{}%
\AgdaBound{s}\AgdaSymbol{)}\AgdaSpace{}%
\AgdaBound{q}\AgdaSymbol{)}\AgdaSpace{}%
\AgdaOperator{\AgdaFunction{∘spHom}}\AgdaSpace{}%
\AgdaBound{h'}\AgdaSpace{}%
\AgdaBound{x}\AgdaSpace{}%
\AgdaBound{s}\AgdaSpace{}%
\AgdaSymbol{(}\AgdaBound{g}\AgdaSpace{}%
\AgdaBound{x}\AgdaSpace{}%
\AgdaSymbol{(}\AgdaBound{f'}\AgdaSpace{}%
\AgdaBound{s}\AgdaSymbol{)}\AgdaSpace{}%
\AgdaBound{q}\AgdaSymbol{)}\<%
\\
%
\>[2]\AgdaSymbol{)}\<%
\\
%
\\[\AgdaEmptyExtraSkip]%
\>[0]\AgdaInductiveConstructor{ε}\AgdaSpace{}%
\AgdaOperator{\AgdaFunction{∘spHom}}\AgdaSpace{}%
\AgdaInductiveConstructor{ε}\AgdaSpace{}%
\AgdaSymbol{=}\AgdaSpace{}%
\AgdaInductiveConstructor{ε}\<%
\\
\>[0]\AgdaInductiveConstructor{ε}\AgdaSpace{}%
\AgdaOperator{\AgdaFunction{∘spHom}}\AgdaSpace{}%
\AgdaSymbol{(}\AgdaBound{g}\AgdaSpace{}%
\AgdaOperator{\AgdaInductiveConstructor{,}}\AgdaSpace{}%
\AgdaBound{gs}\AgdaSymbol{)}\AgdaSpace{}%
\AgdaSymbol{=}\AgdaSpace{}%
\AgdaBound{g}\AgdaSpace{}%
\AgdaOperator{\AgdaInductiveConstructor{,}}\AgdaSpace{}%
\AgdaBound{gs}\<%
\\
\>[0]\AgdaSymbol{(}\AgdaBound{f}\AgdaSpace{}%
\AgdaOperator{\AgdaInductiveConstructor{,}}\AgdaSpace{}%
\AgdaBound{fs}\AgdaSymbol{)}\AgdaSpace{}%
\AgdaOperator{\AgdaFunction{∘spHom}}\AgdaSpace{}%
\AgdaInductiveConstructor{ε}\AgdaSpace{}%
\AgdaSymbol{=}\AgdaSpace{}%
\AgdaBound{f}\AgdaSpace{}%
\AgdaOperator{\AgdaInductiveConstructor{,}}\AgdaSpace{}%
\AgdaBound{fs}\<%
\\
\>[0]\AgdaSymbol{(}\AgdaBound{f}\AgdaSpace{}%
\AgdaOperator{\AgdaInductiveConstructor{,}}\AgdaSpace{}%
\AgdaBound{fs}\AgdaSymbol{)}\AgdaSpace{}%
\AgdaOperator{\AgdaFunction{∘spHom}}\AgdaSpace{}%
\AgdaSymbol{(}\AgdaBound{g}\AgdaSpace{}%
\AgdaOperator{\AgdaInductiveConstructor{,}}\AgdaSpace{}%
\AgdaBound{gs}\AgdaSymbol{)}\AgdaSpace{}%
\AgdaSymbol{=}\AgdaSpace{}%
\AgdaSymbol{(}\AgdaBound{f}\AgdaSpace{}%
\AgdaOperator{\AgdaFunction{∘nfHom}}\AgdaSpace{}%
\AgdaBound{g}\AgdaSymbol{)}\AgdaSpace{}%
\AgdaOperator{\AgdaInductiveConstructor{,}}\AgdaSpace{}%
\AgdaSymbol{(}\AgdaBound{fs}\AgdaSpace{}%
\AgdaOperator{\AgdaFunction{∘spHom}}\AgdaSpace{}%
\AgdaBound{gs}\AgdaSymbol{)}\<%
\\
%
\\[\AgdaEmptyExtraSkip]%
\>[0]\AgdaFunction{⊤nf}\AgdaSpace{}%
\AgdaSymbol{:}\AgdaSpace{}%
\AgdaDatatype{Nf}\AgdaSpace{}%
\AgdaGeneralizable{Γ}\AgdaSpace{}%
\AgdaGeneralizable{A}\<%
\\
\>[0]\AgdaFunction{⊤nf}\AgdaSpace{}%
\AgdaSymbol{\{}\AgdaBound{Γ}\AgdaSymbol{\}}\AgdaSpace{}%
\AgdaSymbol{\{}\AgdaInductiveConstructor{*}\AgdaSymbol{\}}\AgdaSpace{}%
\AgdaSymbol{=}\AgdaSpace{}%
\AgdaInductiveConstructor{ne}\AgdaSpace{}%
\AgdaSymbol{(}\AgdaRecord{⊤}\AgdaSpace{}%
\AgdaOperator{\AgdaInductiveConstructor{◃}}\AgdaSpace{}%
\AgdaSymbol{(λ\{}\AgdaSpace{}%
\AgdaBound{x}\AgdaSpace{}%
\AgdaInductiveConstructor{tt}\AgdaSpace{}%
\AgdaSymbol{→}\AgdaSpace{}%
\AgdaFunction{⊥}\AgdaSpace{}%
\AgdaSymbol{\})}\AgdaSpace{}%
\AgdaOperator{\AgdaInductiveConstructor{◃}}\AgdaSpace{}%
\AgdaSymbol{λ\{}\AgdaSpace{}%
\AgdaBound{x}\AgdaSpace{}%
\AgdaInductiveConstructor{tt}\AgdaSpace{}%
\AgdaSymbol{()}\AgdaSpace{}%
\AgdaSymbol{\})}\<%
\\
\>[0]\AgdaFunction{⊤nf}\AgdaSpace{}%
\AgdaSymbol{\{}\AgdaBound{Γ}\AgdaSymbol{\}}\AgdaSpace{}%
\AgdaSymbol{\{}\AgdaBound{A}\AgdaSpace{}%
\AgdaOperator{\AgdaInductiveConstructor{⇒}}\AgdaSpace{}%
\AgdaBound{B}\AgdaSymbol{\}}\AgdaSpace{}%
\AgdaSymbol{=}\AgdaSpace{}%
\AgdaInductiveConstructor{lam}\AgdaSpace{}%
\AgdaFunction{⊤nf}\<%
\\
%
\\[\AgdaEmptyExtraSkip]%
\>[0]\AgdaFunction{⊥nf}\AgdaSpace{}%
\AgdaSymbol{:}\AgdaSpace{}%
\AgdaDatatype{Nf}\AgdaSpace{}%
\AgdaGeneralizable{Γ}\AgdaSpace{}%
\AgdaGeneralizable{A}\<%
\\
\>[0]\AgdaFunction{⊥nf}\AgdaSpace{}%
\AgdaSymbol{\{}\AgdaBound{Γ}\AgdaSymbol{\}}\AgdaSpace{}%
\AgdaSymbol{\{}\AgdaInductiveConstructor{*}\AgdaSymbol{\}}\AgdaSpace{}%
\AgdaSymbol{=}\AgdaSpace{}%
\AgdaInductiveConstructor{ne}\AgdaSpace{}%
\AgdaSymbol{(}\AgdaFunction{⊥}\AgdaSpace{}%
\AgdaOperator{\AgdaInductiveConstructor{◃}}\AgdaSpace{}%
\AgdaSymbol{(λ}\AgdaSpace{}%
\AgdaBound{x}\AgdaSpace{}%
\AgdaSymbol{())}\AgdaSpace{}%
\AgdaOperator{\AgdaInductiveConstructor{◃}}\AgdaSpace{}%
\AgdaSymbol{(λ}\AgdaSpace{}%
\AgdaBound{x}\AgdaSpace{}%
\AgdaSymbol{()))}\<%
\\
\>[0]\AgdaFunction{⊥nf}\AgdaSpace{}%
\AgdaSymbol{\{}\AgdaBound{Γ}\AgdaSymbol{\}}\AgdaSpace{}%
\AgdaSymbol{\{}\AgdaBound{A}\AgdaSpace{}%
\AgdaOperator{\AgdaInductiveConstructor{⇒}}\AgdaSpace{}%
\AgdaBound{B}\AgdaSymbol{\}}\AgdaSpace{}%
\AgdaSymbol{=}\AgdaSpace{}%
\AgdaInductiveConstructor{lam}\AgdaSpace{}%
\AgdaFunction{⊥nf}\<%
\\
%
\\[\AgdaEmptyExtraSkip]%
\>[0]\AgdaOperator{\AgdaFunction{\AgdaUnderscore{}×nf\AgdaUnderscore{}}}\AgdaSpace{}%
\AgdaSymbol{:}\AgdaSpace{}%
\AgdaDatatype{Nf}\AgdaSpace{}%
\AgdaGeneralizable{Γ}\AgdaSpace{}%
\AgdaGeneralizable{A}\AgdaSpace{}%
\AgdaSymbol{→}\AgdaSpace{}%
\AgdaDatatype{Nf}\AgdaSpace{}%
\AgdaGeneralizable{Γ}\AgdaSpace{}%
\AgdaGeneralizable{A}\AgdaSpace{}%
\AgdaSymbol{→}\AgdaSpace{}%
\AgdaDatatype{Nf}\AgdaSpace{}%
\AgdaGeneralizable{Γ}\AgdaSpace{}%
\AgdaGeneralizable{A}\<%
\\
\>[0]\AgdaInductiveConstructor{lam}\AgdaSpace{}%
\AgdaBound{t}\AgdaSpace{}%
\AgdaOperator{\AgdaFunction{×nf}}\AgdaSpace{}%
\AgdaInductiveConstructor{lam}\AgdaSpace{}%
\AgdaBound{u}\AgdaSpace{}%
\AgdaSymbol{=}\AgdaSpace{}%
\AgdaInductiveConstructor{lam}\AgdaSpace{}%
\AgdaSymbol{(}\AgdaBound{t}\AgdaSpace{}%
\AgdaOperator{\AgdaFunction{×nf}}\AgdaSpace{}%
\AgdaBound{u}\AgdaSymbol{)}\<%
\\
\>[0]\AgdaOperator{\AgdaFunction{\AgdaUnderscore{}×nf\AgdaUnderscore{}}}\AgdaSpace{}%
\AgdaSymbol{\{}\AgdaBound{Γ}\AgdaSymbol{\}}\AgdaSpace{}%
\AgdaSymbol{\{}\AgdaBound{B}\AgdaSymbol{\}}\AgdaSpace{}%
\AgdaSymbol{(}\AgdaInductiveConstructor{ne}\AgdaSpace{}%
\AgdaSymbol{(}\AgdaBound{S}\AgdaSpace{}%
\AgdaOperator{\AgdaInductiveConstructor{◃}}\AgdaSpace{}%
\AgdaBound{P}\AgdaSpace{}%
\AgdaOperator{\AgdaInductiveConstructor{◃}}\AgdaSpace{}%
\AgdaBound{R}\AgdaSymbol{))}\AgdaSpace{}%
\AgdaSymbol{(}\AgdaInductiveConstructor{ne}\AgdaSpace{}%
\AgdaSymbol{(}\AgdaBound{T}\AgdaSpace{}%
\AgdaOperator{\AgdaInductiveConstructor{◃}}\AgdaSpace{}%
\AgdaBound{Q}\AgdaSpace{}%
\AgdaOperator{\AgdaInductiveConstructor{◃}}\AgdaSpace{}%
\AgdaBound{L}\AgdaSymbol{))}\AgdaSpace{}%
\AgdaSymbol{=}\AgdaSpace{}%
\AgdaInductiveConstructor{ne}\AgdaSpace{}%
\AgdaSymbol{(}\AgdaFunction{S'}\AgdaSpace{}%
\AgdaOperator{\AgdaInductiveConstructor{◃}}\AgdaSpace{}%
\AgdaFunction{P'}\AgdaSpace{}%
\AgdaOperator{\AgdaInductiveConstructor{◃}}\AgdaSpace{}%
\AgdaFunction{R'}\AgdaSymbol{)}\<%
\\
\>[0][@{}l@{\AgdaIndent{0}}]%
\>[2]\AgdaKeyword{where}\<%
\\
%
\>[2]\AgdaFunction{S'}\AgdaSpace{}%
\AgdaSymbol{:}\AgdaSpace{}%
\AgdaPrimitive{Set}\<%
\\
%
\>[2]\AgdaFunction{S'}\AgdaSpace{}%
\AgdaSymbol{=}\AgdaSpace{}%
\AgdaBound{S}\AgdaSpace{}%
\AgdaOperator{\AgdaFunction{×}}\AgdaSpace{}%
\AgdaBound{T}\<%
\\
%
\\[\AgdaEmptyExtraSkip]%
%
\>[2]\AgdaFunction{P'}\AgdaSpace{}%
\AgdaSymbol{:}\AgdaSpace{}%
\AgdaDatatype{Var}\AgdaSpace{}%
\AgdaBound{Γ}\AgdaSpace{}%
\AgdaGeneralizable{A}\AgdaSpace{}%
\AgdaSymbol{→}\AgdaSpace{}%
\AgdaFunction{S'}\AgdaSpace{}%
\AgdaSymbol{→}\AgdaSpace{}%
\AgdaPrimitive{Set}\<%
\\
%
\>[2]\AgdaFunction{P'}\AgdaSpace{}%
\AgdaBound{x}\AgdaSpace{}%
\AgdaSymbol{(}\AgdaBound{s}\AgdaSpace{}%
\AgdaOperator{\AgdaInductiveConstructor{,}}\AgdaSpace{}%
\AgdaBound{t}\AgdaSymbol{)}\AgdaSpace{}%
\AgdaSymbol{=}\AgdaSpace{}%
\AgdaBound{P}\AgdaSpace{}%
\AgdaBound{x}\AgdaSpace{}%
\AgdaBound{s}\AgdaSpace{}%
\AgdaOperator{\AgdaDatatype{⊎}}\AgdaSpace{}%
\AgdaBound{Q}\AgdaSpace{}%
\AgdaBound{x}\AgdaSpace{}%
\AgdaBound{t}\<%
\\
%
\\[\AgdaEmptyExtraSkip]%
%
\>[2]\AgdaFunction{R'}\AgdaSpace{}%
\AgdaSymbol{:}\AgdaSpace{}%
\AgdaSymbol{(}\AgdaBound{x}\AgdaSpace{}%
\AgdaSymbol{:}\AgdaSpace{}%
\AgdaDatatype{Var}\AgdaSpace{}%
\AgdaBound{Γ}\AgdaSpace{}%
\AgdaGeneralizable{A}\AgdaSymbol{)}\AgdaSpace{}%
\AgdaSymbol{(}\AgdaBound{s}\AgdaSpace{}%
\AgdaSymbol{:}\AgdaSpace{}%
\AgdaFunction{S'}\AgdaSymbol{)}\AgdaSpace{}%
\AgdaSymbol{→}\AgdaSpace{}%
\AgdaFunction{P'}\AgdaSpace{}%
\AgdaBound{x}\AgdaSpace{}%
\AgdaBound{s}\AgdaSpace{}%
\AgdaSymbol{→}\AgdaSpace{}%
\AgdaDatatype{Sp}\AgdaSpace{}%
\AgdaBound{Γ}\AgdaSpace{}%
\AgdaGeneralizable{A}\AgdaSpace{}%
\AgdaBound{B}\<%
\\
%
\>[2]\AgdaFunction{R'}\AgdaSpace{}%
\AgdaBound{x}\AgdaSpace{}%
\AgdaSymbol{(}\AgdaBound{s}\AgdaSpace{}%
\AgdaOperator{\AgdaInductiveConstructor{,}}\AgdaSpace{}%
\AgdaBound{t}\AgdaSymbol{)}\AgdaSpace{}%
\AgdaSymbol{(}\AgdaInductiveConstructor{inj₁}\AgdaSpace{}%
\AgdaBound{p}\AgdaSymbol{)}\AgdaSpace{}%
\AgdaSymbol{=}\AgdaSpace{}%
\AgdaBound{R}\AgdaSpace{}%
\AgdaBound{x}\AgdaSpace{}%
\AgdaBound{s}\AgdaSpace{}%
\AgdaBound{p}\<%
\\
%
\>[2]\AgdaFunction{R'}\AgdaSpace{}%
\AgdaBound{x}\AgdaSpace{}%
\AgdaSymbol{(}\AgdaBound{s}\AgdaSpace{}%
\AgdaOperator{\AgdaInductiveConstructor{,}}\AgdaSpace{}%
\AgdaBound{t}\AgdaSymbol{)}\AgdaSpace{}%
\AgdaSymbol{(}\AgdaInductiveConstructor{inj₂}\AgdaSpace{}%
\AgdaBound{q}\AgdaSymbol{)}\AgdaSpace{}%
\AgdaSymbol{=}\AgdaSpace{}%
\AgdaBound{L}\AgdaSpace{}%
\AgdaBound{x}\AgdaSpace{}%
\AgdaBound{t}\AgdaSpace{}%
\AgdaBound{q}\<%
\\
%
\\[\AgdaEmptyExtraSkip]%
\>[0]\AgdaOperator{\AgdaFunction{\AgdaUnderscore{}⊎nf\AgdaUnderscore{}}}\AgdaSpace{}%
\AgdaSymbol{:}\AgdaSpace{}%
\AgdaDatatype{Nf}\AgdaSpace{}%
\AgdaGeneralizable{Γ}\AgdaSpace{}%
\AgdaGeneralizable{A}\AgdaSpace{}%
\AgdaSymbol{→}\AgdaSpace{}%
\AgdaDatatype{Nf}\AgdaSpace{}%
\AgdaGeneralizable{Γ}\AgdaSpace{}%
\AgdaGeneralizable{A}\AgdaSpace{}%
\AgdaSymbol{→}\AgdaSpace{}%
\AgdaDatatype{Nf}\AgdaSpace{}%
\AgdaGeneralizable{Γ}\AgdaSpace{}%
\AgdaGeneralizable{A}\<%
\\
\>[0]\AgdaInductiveConstructor{lam}\AgdaSpace{}%
\AgdaBound{t}\AgdaSpace{}%
\AgdaOperator{\AgdaFunction{⊎nf}}\AgdaSpace{}%
\AgdaInductiveConstructor{lam}\AgdaSpace{}%
\AgdaBound{u}\AgdaSpace{}%
\AgdaSymbol{=}\AgdaSpace{}%
\AgdaInductiveConstructor{lam}\AgdaSpace{}%
\AgdaSymbol{(}\AgdaBound{t}\AgdaSpace{}%
\AgdaOperator{\AgdaFunction{⊎nf}}\AgdaSpace{}%
\AgdaBound{u}\AgdaSymbol{)}\<%
\\
\>[0]\AgdaOperator{\AgdaFunction{\AgdaUnderscore{}⊎nf\AgdaUnderscore{}}}\AgdaSpace{}%
\AgdaSymbol{\{}\AgdaBound{Γ}\AgdaSymbol{\}}\AgdaSpace{}%
\AgdaSymbol{\{}\AgdaBound{B}\AgdaSymbol{\}}\AgdaSpace{}%
\AgdaSymbol{(}\AgdaInductiveConstructor{ne}\AgdaSpace{}%
\AgdaSymbol{(}\AgdaBound{S}\AgdaSpace{}%
\AgdaOperator{\AgdaInductiveConstructor{◃}}\AgdaSpace{}%
\AgdaBound{P}\AgdaSpace{}%
\AgdaOperator{\AgdaInductiveConstructor{◃}}\AgdaSpace{}%
\AgdaBound{R}\AgdaSymbol{))}\AgdaSpace{}%
\AgdaSymbol{(}\AgdaInductiveConstructor{ne}\AgdaSpace{}%
\AgdaSymbol{(}\AgdaBound{T}\AgdaSpace{}%
\AgdaOperator{\AgdaInductiveConstructor{◃}}\AgdaSpace{}%
\AgdaBound{Q}\AgdaSpace{}%
\AgdaOperator{\AgdaInductiveConstructor{◃}}\AgdaSpace{}%
\AgdaBound{L}\AgdaSymbol{))}\AgdaSpace{}%
\AgdaSymbol{=}\AgdaSpace{}%
\AgdaInductiveConstructor{ne}\AgdaSpace{}%
\AgdaSymbol{(}\AgdaFunction{S'}\AgdaSpace{}%
\AgdaOperator{\AgdaInductiveConstructor{◃}}\AgdaSpace{}%
\AgdaFunction{P'}\AgdaSpace{}%
\AgdaOperator{\AgdaInductiveConstructor{◃}}\AgdaSpace{}%
\AgdaFunction{R'}\AgdaSymbol{)}\<%
\\
\>[0][@{}l@{\AgdaIndent{0}}]%
\>[2]\AgdaKeyword{where}\<%
\\
%
\>[2]\AgdaFunction{S'}\AgdaSpace{}%
\AgdaSymbol{:}\AgdaSpace{}%
\AgdaPrimitive{Set}\<%
\\
%
\>[2]\AgdaFunction{S'}\AgdaSpace{}%
\AgdaSymbol{=}\AgdaSpace{}%
\AgdaBound{S}\AgdaSpace{}%
\AgdaOperator{\AgdaDatatype{⊎}}\AgdaSpace{}%
\AgdaBound{T}\<%
\\
%
\\[\AgdaEmptyExtraSkip]%
%
\>[2]\AgdaFunction{P'}\AgdaSpace{}%
\AgdaSymbol{:}\AgdaSpace{}%
\AgdaDatatype{Var}\AgdaSpace{}%
\AgdaBound{Γ}\AgdaSpace{}%
\AgdaGeneralizable{A}\AgdaSpace{}%
\AgdaSymbol{→}\AgdaSpace{}%
\AgdaFunction{S'}\AgdaSpace{}%
\AgdaSymbol{→}\AgdaSpace{}%
\AgdaPrimitive{Set}\<%
\\
%
\>[2]\AgdaFunction{P'}\AgdaSpace{}%
\AgdaBound{x}\AgdaSpace{}%
\AgdaSymbol{(}\AgdaInductiveConstructor{inj₁}\AgdaSpace{}%
\AgdaBound{s}\AgdaSymbol{)}\AgdaSpace{}%
\AgdaSymbol{=}\AgdaSpace{}%
\AgdaBound{P}\AgdaSpace{}%
\AgdaBound{x}\AgdaSpace{}%
\AgdaBound{s}\<%
\\
%
\>[2]\AgdaFunction{P'}\AgdaSpace{}%
\AgdaBound{x}\AgdaSpace{}%
\AgdaSymbol{(}\AgdaInductiveConstructor{inj₂}\AgdaSpace{}%
\AgdaBound{t}\AgdaSymbol{)}\AgdaSpace{}%
\AgdaSymbol{=}\AgdaSpace{}%
\AgdaBound{Q}\AgdaSpace{}%
\AgdaBound{x}\AgdaSpace{}%
\AgdaBound{t}\<%
\\
%
\\[\AgdaEmptyExtraSkip]%
%
\>[2]\AgdaFunction{R'}\AgdaSpace{}%
\AgdaSymbol{:}\AgdaSpace{}%
\AgdaSymbol{(}\AgdaBound{x}\AgdaSpace{}%
\AgdaSymbol{:}\AgdaSpace{}%
\AgdaDatatype{Var}\AgdaSpace{}%
\AgdaBound{Γ}\AgdaSpace{}%
\AgdaGeneralizable{A}\AgdaSymbol{)}\AgdaSpace{}%
\AgdaSymbol{(}\AgdaBound{s}\AgdaSpace{}%
\AgdaSymbol{:}\AgdaSpace{}%
\AgdaFunction{S'}\AgdaSymbol{)}\AgdaSpace{}%
\AgdaSymbol{→}\AgdaSpace{}%
\AgdaFunction{P'}\AgdaSpace{}%
\AgdaBound{x}\AgdaSpace{}%
\AgdaBound{s}\AgdaSpace{}%
\AgdaSymbol{→}\AgdaSpace{}%
\AgdaDatatype{Sp}\AgdaSpace{}%
\AgdaBound{Γ}\AgdaSpace{}%
\AgdaGeneralizable{A}\AgdaSpace{}%
\AgdaBound{B}\<%
\\
%
\>[2]\AgdaFunction{R'}\AgdaSpace{}%
\AgdaBound{x}\AgdaSpace{}%
\AgdaSymbol{(}\AgdaInductiveConstructor{inj₁}\AgdaSpace{}%
\AgdaBound{s}\AgdaSymbol{)}\AgdaSpace{}%
\AgdaBound{p}\AgdaSpace{}%
\AgdaSymbol{=}\AgdaSpace{}%
\AgdaBound{R}\AgdaSpace{}%
\AgdaBound{x}\AgdaSpace{}%
\AgdaBound{s}\AgdaSpace{}%
\AgdaBound{p}\<%
\\
%
\>[2]\AgdaFunction{R'}\AgdaSpace{}%
\AgdaBound{x}\AgdaSpace{}%
\AgdaSymbol{(}\AgdaInductiveConstructor{inj₂}\AgdaSpace{}%
\AgdaBound{t}\AgdaSymbol{)}\AgdaSpace{}%
\AgdaBound{q}\AgdaSpace{}%
\AgdaSymbol{=}\AgdaSpace{}%
\AgdaBound{L}\AgdaSpace{}%
\AgdaBound{x}\AgdaSpace{}%
\AgdaBound{t}\AgdaSpace{}%
\AgdaBound{q}\<%
\\
%
\\[\AgdaEmptyExtraSkip]%
\>[0]\AgdaFunction{!nf}\AgdaSpace{}%
\AgdaSymbol{:}\AgdaSpace{}%
\AgdaSymbol{(}\AgdaBound{t}\AgdaSpace{}%
\AgdaSymbol{:}\AgdaSpace{}%
\AgdaDatatype{Nf}\AgdaSpace{}%
\AgdaGeneralizable{Γ}\AgdaSpace{}%
\AgdaGeneralizable{A}\AgdaSymbol{)}\AgdaSpace{}%
\AgdaSymbol{→}\AgdaSpace{}%
\AgdaDatatype{NfHom}\AgdaSpace{}%
\AgdaBound{t}\AgdaSpace{}%
\AgdaFunction{⊤nf}\<%
\\
\>[0]\AgdaFunction{!nf}\AgdaSpace{}%
\AgdaSymbol{(}\AgdaInductiveConstructor{lam}\AgdaSpace{}%
\AgdaBound{t}\AgdaSymbol{)}\AgdaSpace{}%
\AgdaSymbol{=}\AgdaSpace{}%
\AgdaInductiveConstructor{lam}\AgdaSpace{}%
\AgdaSymbol{(}\AgdaFunction{!nf}\AgdaSpace{}%
\AgdaBound{t}\AgdaSymbol{)}\<%
\\
\>[0]\AgdaFunction{!nf}\AgdaSpace{}%
\AgdaSymbol{(}\AgdaInductiveConstructor{ne}\AgdaSpace{}%
\AgdaSymbol{(}\AgdaBound{S}\AgdaSpace{}%
\AgdaOperator{\AgdaInductiveConstructor{◃}}\AgdaSpace{}%
\AgdaBound{P}\AgdaSpace{}%
\AgdaOperator{\AgdaInductiveConstructor{◃}}\AgdaSpace{}%
\AgdaBound{R}\AgdaSymbol{))}\AgdaSpace{}%
\AgdaSymbol{=}\AgdaSpace{}%
\AgdaInductiveConstructor{ne}\AgdaSpace{}%
\AgdaSymbol{((λ}\AgdaSpace{}%
\AgdaBound{\AgdaUnderscore{}}\AgdaSpace{}%
\AgdaSymbol{→}\AgdaSpace{}%
\AgdaInductiveConstructor{tt}\AgdaSymbol{)}\AgdaSpace{}%
\AgdaOperator{\AgdaInductiveConstructor{◃}}\AgdaSpace{}%
\AgdaSymbol{(λ}\AgdaSpace{}%
\AgdaBound{x}\AgdaSpace{}%
\AgdaBound{s}\AgdaSpace{}%
\AgdaSymbol{())}\AgdaSpace{}%
\AgdaOperator{\AgdaInductiveConstructor{◃}}\AgdaSpace{}%
\AgdaSymbol{λ}\AgdaSpace{}%
\AgdaBound{x}\AgdaSpace{}%
\AgdaBound{s}\AgdaSpace{}%
\AgdaSymbol{())}\<%
\\
%
\\[\AgdaEmptyExtraSkip]%
\>[0]\AgdaFunction{¿nf}\AgdaSpace{}%
\AgdaSymbol{:}\AgdaSpace{}%
\AgdaSymbol{(}\AgdaBound{t}\AgdaSpace{}%
\AgdaSymbol{:}\AgdaSpace{}%
\AgdaDatatype{Nf}\AgdaSpace{}%
\AgdaGeneralizable{Γ}\AgdaSpace{}%
\AgdaGeneralizable{A}\AgdaSymbol{)}\AgdaSpace{}%
\AgdaSymbol{→}\AgdaSpace{}%
\AgdaDatatype{NfHom}\AgdaSpace{}%
\AgdaFunction{⊥nf}\AgdaSpace{}%
\AgdaBound{t}\<%
\\
\>[0]\AgdaFunction{¿nf}\AgdaSpace{}%
\AgdaSymbol{(}\AgdaInductiveConstructor{lam}\AgdaSpace{}%
\AgdaBound{t}\AgdaSymbol{)}\AgdaSpace{}%
\AgdaSymbol{=}\AgdaSpace{}%
\AgdaInductiveConstructor{lam}\AgdaSpace{}%
\AgdaSymbol{(}\AgdaFunction{¿nf}\AgdaSpace{}%
\AgdaBound{t}\AgdaSymbol{)}\<%
\\
\>[0]\AgdaFunction{¿nf}\AgdaSpace{}%
\AgdaSymbol{(}\AgdaInductiveConstructor{ne}\AgdaSpace{}%
\AgdaSymbol{(}\AgdaBound{S}\AgdaSpace{}%
\AgdaOperator{\AgdaInductiveConstructor{◃}}\AgdaSpace{}%
\AgdaBound{P}\AgdaSpace{}%
\AgdaOperator{\AgdaInductiveConstructor{◃}}\AgdaSpace{}%
\AgdaBound{R}\AgdaSymbol{))}\AgdaSpace{}%
\AgdaSymbol{=}\AgdaSpace{}%
\AgdaInductiveConstructor{ne}\AgdaSpace{}%
\AgdaSymbol{((λ}\AgdaSpace{}%
\AgdaSymbol{())}\AgdaSpace{}%
\AgdaOperator{\AgdaInductiveConstructor{◃}}\AgdaSpace{}%
\AgdaSymbol{(λ}\AgdaSpace{}%
\AgdaBound{x}\AgdaSpace{}%
\AgdaSymbol{())}\AgdaSpace{}%
\AgdaOperator{\AgdaInductiveConstructor{◃}}\AgdaSpace{}%
\AgdaSymbol{λ}\AgdaSpace{}%
\AgdaBound{x}\AgdaSpace{}%
\AgdaSymbol{())}\<%
\\
%
\\[\AgdaEmptyExtraSkip]%
\>[0]\AgdaFunction{π₁nf}\AgdaSpace{}%
\AgdaSymbol{:}\AgdaSpace{}%
\AgdaSymbol{(}\AgdaBound{t}\AgdaSpace{}%
\AgdaBound{u}\AgdaSpace{}%
\AgdaSymbol{:}\AgdaSpace{}%
\AgdaDatatype{Nf}\AgdaSpace{}%
\AgdaGeneralizable{Γ}\AgdaSpace{}%
\AgdaGeneralizable{A}\AgdaSymbol{)}\AgdaSpace{}%
\AgdaSymbol{→}\AgdaSpace{}%
\AgdaDatatype{NfHom}\AgdaSpace{}%
\AgdaSymbol{(}\AgdaBound{t}\AgdaSpace{}%
\AgdaOperator{\AgdaFunction{×nf}}\AgdaSpace{}%
\AgdaBound{u}\AgdaSymbol{)}\AgdaSpace{}%
\AgdaBound{t}\<%
\\
\>[0]\AgdaFunction{π₁nf}\AgdaSpace{}%
\AgdaSymbol{(}\AgdaInductiveConstructor{lam}\AgdaSpace{}%
\AgdaBound{t}\AgdaSymbol{)}\AgdaSpace{}%
\AgdaSymbol{(}\AgdaInductiveConstructor{lam}\AgdaSpace{}%
\AgdaBound{u}\AgdaSymbol{)}\AgdaSpace{}%
\AgdaSymbol{=}\AgdaSpace{}%
\AgdaInductiveConstructor{lam}\AgdaSpace{}%
\AgdaSymbol{(}\AgdaFunction{π₁nf}\AgdaSpace{}%
\AgdaBound{t}\AgdaSpace{}%
\AgdaBound{u}\AgdaSymbol{)}\<%
\\
\>[0]\AgdaFunction{π₁nf}\AgdaSpace{}%
\AgdaSymbol{\{}\AgdaBound{Γ}\AgdaSymbol{\}}\AgdaSpace{}%
\AgdaSymbol{\{}\AgdaBound{B}\AgdaSymbol{\}}\AgdaSpace{}%
\AgdaSymbol{(}\AgdaInductiveConstructor{ne}\AgdaSpace{}%
\AgdaSymbol{(}\AgdaBound{S}\AgdaSpace{}%
\AgdaOperator{\AgdaInductiveConstructor{◃}}\AgdaSpace{}%
\AgdaBound{P}\AgdaSpace{}%
\AgdaOperator{\AgdaInductiveConstructor{◃}}\AgdaSpace{}%
\AgdaBound{R}\AgdaSymbol{))}\AgdaSpace{}%
\AgdaSymbol{(}\AgdaInductiveConstructor{ne}\AgdaSpace{}%
\AgdaSymbol{(}\AgdaBound{T}\AgdaSpace{}%
\AgdaOperator{\AgdaInductiveConstructor{◃}}\AgdaSpace{}%
\AgdaBound{Q}\AgdaSpace{}%
\AgdaOperator{\AgdaInductiveConstructor{◃}}\AgdaSpace{}%
\AgdaBound{L}\AgdaSymbol{))}\AgdaSpace{}%
\AgdaSymbol{=}\AgdaSpace{}%
\AgdaInductiveConstructor{ne}\AgdaSpace{}%
\AgdaSymbol{(}\AgdaFunction{f}\AgdaSpace{}%
\AgdaOperator{\AgdaInductiveConstructor{◃}}\AgdaSpace{}%
\AgdaFunction{g}\AgdaSpace{}%
\AgdaOperator{\AgdaInductiveConstructor{◃}}\AgdaSpace{}%
\AgdaFunction{h}\AgdaSymbol{)}\<%
\\
\>[0][@{}l@{\AgdaIndent{0}}]%
\>[2]\AgdaKeyword{where}\<%
\\
%
\>[2]\AgdaFunction{f}\AgdaSpace{}%
\AgdaSymbol{:}\AgdaSpace{}%
\AgdaBound{S}\AgdaSpace{}%
\AgdaOperator{\AgdaFunction{×}}\AgdaSpace{}%
\AgdaBound{T}\AgdaSpace{}%
\AgdaSymbol{→}\AgdaSpace{}%
\AgdaBound{S}\<%
\\
%
\>[2]\AgdaFunction{f}\AgdaSpace{}%
\AgdaSymbol{(}\AgdaBound{s}\AgdaSpace{}%
\AgdaOperator{\AgdaInductiveConstructor{,}}\AgdaSpace{}%
\AgdaBound{t}\AgdaSymbol{)}\AgdaSpace{}%
\AgdaSymbol{=}\AgdaSpace{}%
\AgdaBound{s}\<%
\\
%
\\[\AgdaEmptyExtraSkip]%
%
\>[2]\AgdaFunction{g}\AgdaSpace{}%
\AgdaSymbol{:}\AgdaSpace{}%
\AgdaSymbol{(}\AgdaBound{x}\AgdaSpace{}%
\AgdaSymbol{:}\AgdaSpace{}%
\AgdaDatatype{Var}\AgdaSpace{}%
\AgdaBound{Γ}\AgdaSpace{}%
\AgdaGeneralizable{A}\AgdaSymbol{)}\AgdaSpace{}%
\AgdaSymbol{(}\AgdaBound{st}\AgdaSpace{}%
\AgdaSymbol{:}\AgdaSpace{}%
\AgdaBound{S}\AgdaSpace{}%
\AgdaOperator{\AgdaFunction{×}}\AgdaSpace{}%
\AgdaBound{T}\AgdaSymbol{)}\AgdaSpace{}%
\AgdaSymbol{→}\AgdaSpace{}%
\AgdaBound{P}\AgdaSpace{}%
\AgdaBound{x}\AgdaSpace{}%
\AgdaSymbol{(}\AgdaFunction{f}\AgdaSpace{}%
\AgdaBound{st}\AgdaSymbol{)}\AgdaSpace{}%
\AgdaSymbol{→}\AgdaSpace{}%
\AgdaBound{P}\AgdaSpace{}%
\AgdaBound{x}\AgdaSpace{}%
\AgdaSymbol{(}\AgdaBound{st}\AgdaSpace{}%
\AgdaSymbol{.}\AgdaField{proj₁}\AgdaSymbol{)}\AgdaSpace{}%
\AgdaOperator{\AgdaDatatype{⊎}}\AgdaSpace{}%
\AgdaBound{Q}\AgdaSpace{}%
\AgdaBound{x}\AgdaSpace{}%
\AgdaSymbol{(}\AgdaBound{st}\AgdaSpace{}%
\AgdaSymbol{.}\AgdaField{proj₂}\AgdaSymbol{)}\<%
\\
%
\>[2]\AgdaFunction{g}\AgdaSpace{}%
\AgdaBound{x}\AgdaSpace{}%
\AgdaSymbol{(}\AgdaBound{s}\AgdaSpace{}%
\AgdaOperator{\AgdaInductiveConstructor{,}}\AgdaSpace{}%
\AgdaBound{t}\AgdaSymbol{)}\AgdaSpace{}%
\AgdaBound{p}\AgdaSpace{}%
\AgdaSymbol{=}\AgdaSpace{}%
\AgdaInductiveConstructor{inj₁}\AgdaSpace{}%
\AgdaBound{p}\<%
\\
%
\\[\AgdaEmptyExtraSkip]%
%
\>[2]\AgdaFunction{h}\AgdaSpace{}%
\AgdaSymbol{:}\AgdaSpace{}%
\AgdaSymbol{(}\AgdaBound{x}\AgdaSpace{}%
\AgdaSymbol{:}\AgdaSpace{}%
\AgdaDatatype{Var}\AgdaSpace{}%
\AgdaBound{Γ}\AgdaSpace{}%
\AgdaGeneralizable{A}\AgdaSymbol{)}\AgdaSpace{}%
\AgdaSymbol{(}\AgdaBound{st}\AgdaSpace{}%
\AgdaSymbol{:}\AgdaSpace{}%
\AgdaBound{S}\AgdaSpace{}%
\AgdaOperator{\AgdaFunction{×}}\AgdaSpace{}%
\AgdaBound{T}\AgdaSymbol{)}\AgdaSpace{}%
\AgdaSymbol{(}\AgdaBound{q}\AgdaSpace{}%
\AgdaSymbol{:}\AgdaSpace{}%
\AgdaBound{P}\AgdaSpace{}%
\AgdaBound{x}\AgdaSpace{}%
\AgdaSymbol{(}\AgdaFunction{f}\AgdaSpace{}%
\AgdaBound{st}\AgdaSymbol{))}\AgdaSpace{}%
\AgdaSymbol{→}\AgdaSpace{}%
\AgdaDatatype{SpHom}\AgdaSpace{}%
\AgdaSymbol{(}\AgdaBound{R}\AgdaSpace{}%
\AgdaBound{x}\AgdaSpace{}%
\AgdaSymbol{(}\AgdaFunction{f}\AgdaSpace{}%
\AgdaBound{st}\AgdaSymbol{)}\AgdaSpace{}%
\AgdaBound{q}\AgdaSymbol{)}\AgdaSpace{}%
\AgdaSymbol{(}\AgdaBound{R}\AgdaSpace{}%
\AgdaBound{x}\AgdaSpace{}%
\AgdaSymbol{(}\AgdaFunction{f}\AgdaSpace{}%
\AgdaBound{st}\AgdaSymbol{)}\AgdaSpace{}%
\AgdaBound{q}\AgdaSymbol{)}\<%
\\
%
\>[2]\AgdaFunction{h}\AgdaSpace{}%
\AgdaBound{x}\AgdaSpace{}%
\AgdaSymbol{(}\AgdaBound{s}\AgdaSpace{}%
\AgdaOperator{\AgdaInductiveConstructor{,}}\AgdaSpace{}%
\AgdaBound{t}\AgdaSymbol{)}\AgdaSpace{}%
\AgdaBound{q}\AgdaSpace{}%
\AgdaSymbol{=}\AgdaSpace{}%
\AgdaInductiveConstructor{ε}\<%
\\
%
\\[\AgdaEmptyExtraSkip]%
\>[0]\AgdaFunction{i₁nf}\AgdaSpace{}%
\AgdaSymbol{:}\AgdaSpace{}%
\AgdaSymbol{(}\AgdaBound{t}\AgdaSpace{}%
\AgdaBound{u}\AgdaSpace{}%
\AgdaSymbol{:}\AgdaSpace{}%
\AgdaDatatype{Nf}\AgdaSpace{}%
\AgdaGeneralizable{Γ}\AgdaSpace{}%
\AgdaGeneralizable{A}\AgdaSymbol{)}\AgdaSpace{}%
\AgdaSymbol{→}\AgdaSpace{}%
\AgdaDatatype{NfHom}\AgdaSpace{}%
\AgdaBound{t}\AgdaSpace{}%
\AgdaSymbol{(}\AgdaBound{t}\AgdaSpace{}%
\AgdaOperator{\AgdaFunction{⊎nf}}\AgdaSpace{}%
\AgdaBound{u}\AgdaSymbol{)}\<%
\\
\>[0]\AgdaFunction{i₁nf}\AgdaSpace{}%
\AgdaSymbol{(}\AgdaInductiveConstructor{lam}\AgdaSpace{}%
\AgdaBound{t}\AgdaSymbol{)}\AgdaSpace{}%
\AgdaSymbol{(}\AgdaInductiveConstructor{lam}\AgdaSpace{}%
\AgdaBound{u}\AgdaSymbol{)}\AgdaSpace{}%
\AgdaSymbol{=}\AgdaSpace{}%
\AgdaInductiveConstructor{lam}\AgdaSpace{}%
\AgdaSymbol{(}\AgdaFunction{i₁nf}\AgdaSpace{}%
\AgdaBound{t}\AgdaSpace{}%
\AgdaBound{u}\AgdaSymbol{)}\<%
\\
\>[0]\AgdaFunction{i₁nf}\AgdaSpace{}%
\AgdaSymbol{\{}\AgdaBound{Γ}\AgdaSymbol{\}}\AgdaSpace{}%
\AgdaSymbol{(}\AgdaInductiveConstructor{ne}\AgdaSpace{}%
\AgdaSymbol{(}\AgdaBound{S}\AgdaSpace{}%
\AgdaOperator{\AgdaInductiveConstructor{◃}}\AgdaSpace{}%
\AgdaBound{P}\AgdaSpace{}%
\AgdaOperator{\AgdaInductiveConstructor{◃}}\AgdaSpace{}%
\AgdaBound{R}\AgdaSymbol{))}\AgdaSpace{}%
\AgdaSymbol{(}\AgdaInductiveConstructor{ne}\AgdaSpace{}%
\AgdaSymbol{(}\AgdaBound{T}\AgdaSpace{}%
\AgdaOperator{\AgdaInductiveConstructor{◃}}\AgdaSpace{}%
\AgdaBound{Q}\AgdaSpace{}%
\AgdaOperator{\AgdaInductiveConstructor{◃}}\AgdaSpace{}%
\AgdaBound{L}\AgdaSymbol{))}\AgdaSpace{}%
\AgdaSymbol{=}\AgdaSpace{}%
\AgdaInductiveConstructor{ne}\AgdaSpace{}%
\AgdaSymbol{(}\AgdaFunction{f}\AgdaSpace{}%
\AgdaOperator{\AgdaInductiveConstructor{◃}}\AgdaSpace{}%
\AgdaFunction{g}\AgdaSpace{}%
\AgdaOperator{\AgdaInductiveConstructor{◃}}\AgdaSpace{}%
\AgdaFunction{h}\AgdaSymbol{)}\<%
\\
\>[0][@{}l@{\AgdaIndent{0}}]%
\>[2]\AgdaKeyword{where}\<%
\\
%
\>[2]\AgdaFunction{f}\AgdaSpace{}%
\AgdaSymbol{:}\AgdaSpace{}%
\AgdaBound{S}\AgdaSpace{}%
\AgdaSymbol{→}\AgdaSpace{}%
\AgdaBound{S}\AgdaSpace{}%
\AgdaOperator{\AgdaDatatype{⊎}}\AgdaSpace{}%
\AgdaBound{T}\<%
\\
%
\>[2]\AgdaFunction{f}\AgdaSpace{}%
\AgdaBound{s}\AgdaSpace{}%
\AgdaSymbol{=}\AgdaSpace{}%
\AgdaInductiveConstructor{inj₁}\AgdaSpace{}%
\AgdaBound{s}\<%
\\
%
\\[\AgdaEmptyExtraSkip]%
%
\>[2]\AgdaFunction{g}\AgdaSpace{}%
\AgdaSymbol{:}\AgdaSpace{}%
\AgdaSymbol{(}\AgdaBound{x}\AgdaSpace{}%
\AgdaSymbol{:}\AgdaSpace{}%
\AgdaDatatype{Var}\AgdaSpace{}%
\AgdaBound{Γ}\AgdaSpace{}%
\AgdaGeneralizable{A}\AgdaSymbol{)}\AgdaSpace{}%
\AgdaSymbol{(}\AgdaBound{s}\AgdaSpace{}%
\AgdaSymbol{:}\AgdaSpace{}%
\AgdaBound{S}\AgdaSymbol{)}\AgdaSpace{}%
\AgdaSymbol{→}\AgdaSpace{}%
\AgdaBound{P}\AgdaSpace{}%
\AgdaBound{x}\AgdaSpace{}%
\AgdaBound{s}\AgdaSpace{}%
\AgdaSymbol{→}\AgdaSpace{}%
\AgdaBound{P}\AgdaSpace{}%
\AgdaBound{x}\AgdaSpace{}%
\AgdaBound{s}\<%
\\
%
\>[2]\AgdaFunction{g}\AgdaSpace{}%
\AgdaBound{x}\AgdaSpace{}%
\AgdaBound{s}\AgdaSpace{}%
\AgdaBound{p}\AgdaSpace{}%
\AgdaSymbol{=}\AgdaSpace{}%
\AgdaBound{p}\<%
\\
%
\\[\AgdaEmptyExtraSkip]%
%
\>[2]\AgdaFunction{h}\AgdaSpace{}%
\AgdaSymbol{:}\AgdaSpace{}%
\AgdaSymbol{(}\AgdaBound{x}\AgdaSpace{}%
\AgdaSymbol{:}\AgdaSpace{}%
\AgdaDatatype{Var}\AgdaSpace{}%
\AgdaBound{Γ}\AgdaSpace{}%
\AgdaGeneralizable{A}\AgdaSymbol{)}\AgdaSpace{}%
\AgdaSymbol{(}\AgdaBound{s}\AgdaSpace{}%
\AgdaSymbol{:}\AgdaSpace{}%
\AgdaBound{S}\AgdaSymbol{)}\AgdaSpace{}%
\AgdaSymbol{(}\AgdaBound{q}\AgdaSpace{}%
\AgdaSymbol{:}\AgdaSpace{}%
\AgdaBound{P}\AgdaSpace{}%
\AgdaBound{x}\AgdaSpace{}%
\AgdaBound{s}\AgdaSymbol{)}\AgdaSpace{}%
\AgdaSymbol{→}\AgdaSpace{}%
\AgdaDatatype{SpHom}\AgdaSpace{}%
\AgdaSymbol{(}\AgdaBound{R}\AgdaSpace{}%
\AgdaBound{x}\AgdaSpace{}%
\AgdaBound{s}\AgdaSpace{}%
\AgdaSymbol{(}\AgdaFunction{g}\AgdaSpace{}%
\AgdaBound{x}\AgdaSpace{}%
\AgdaBound{s}\AgdaSpace{}%
\AgdaBound{q}\AgdaSymbol{))}\AgdaSpace{}%
\AgdaSymbol{(}\AgdaBound{R}\AgdaSpace{}%
\AgdaBound{x}\AgdaSpace{}%
\AgdaBound{s}\AgdaSpace{}%
\AgdaBound{q}\AgdaSymbol{)}\<%
\\
%
\>[2]\AgdaFunction{h}\AgdaSpace{}%
\AgdaBound{x}\AgdaSpace{}%
\AgdaBound{s}\AgdaSpace{}%
\AgdaBound{q}\AgdaSpace{}%
\AgdaSymbol{=}\AgdaSpace{}%
\AgdaInductiveConstructor{ε}\<%
\\
%
\\[\AgdaEmptyExtraSkip]%
\>[0]\AgdaOperator{\AgdaFunction{<\AgdaUnderscore{},\AgdaUnderscore{}>nf}}\AgdaSpace{}%
\AgdaSymbol{:}\AgdaSpace{}%
\AgdaDatatype{NfHom}\AgdaSpace{}%
\AgdaGeneralizable{t}\AgdaSpace{}%
\AgdaGeneralizable{u}\AgdaSpace{}%
\AgdaSymbol{→}\AgdaSpace{}%
\AgdaDatatype{NfHom}\AgdaSpace{}%
\AgdaGeneralizable{t}\AgdaSpace{}%
\AgdaGeneralizable{w}\AgdaSpace{}%
\AgdaSymbol{→}\AgdaSpace{}%
\AgdaDatatype{NfHom}\AgdaSpace{}%
\AgdaGeneralizable{t}\AgdaSpace{}%
\AgdaSymbol{(}\AgdaGeneralizable{u}\AgdaSpace{}%
\AgdaOperator{\AgdaFunction{×nf}}\AgdaSpace{}%
\AgdaGeneralizable{w}\AgdaSymbol{)}\<%
\\
\>[0]\AgdaOperator{\AgdaFunction{<}}\AgdaSpace{}%
\AgdaInductiveConstructor{lam}\AgdaSpace{}%
\AgdaBound{tu}\AgdaSpace{}%
\AgdaOperator{\AgdaFunction{,}}\AgdaSpace{}%
\AgdaInductiveConstructor{lam}\AgdaSpace{}%
\AgdaBound{tv}\AgdaSpace{}%
\AgdaOperator{\AgdaFunction{>nf}}\AgdaSpace{}%
\AgdaSymbol{=}\AgdaSpace{}%
\AgdaInductiveConstructor{lam}\AgdaSpace{}%
\AgdaOperator{\AgdaFunction{<}}\AgdaSpace{}%
\AgdaBound{tu}\AgdaSpace{}%
\AgdaOperator{\AgdaFunction{,}}\AgdaSpace{}%
\AgdaBound{tv}\AgdaSpace{}%
\AgdaOperator{\AgdaFunction{>nf}}\<%
\\
\>[0]\AgdaOperator{\AgdaFunction{<\AgdaUnderscore{},\AgdaUnderscore{}>nf}}\AgdaSpace{}%
\AgdaSymbol{\{}\AgdaBound{Γ}\AgdaSymbol{\}}\AgdaSpace{}%
\AgdaSymbol{\{}\AgdaBound{B}\AgdaSymbol{\}}\AgdaSpace{}%
\AgdaSymbol{(}\AgdaInductiveConstructor{ne}\AgdaSpace{}%
\AgdaSymbol{(}\AgdaBound{f₁}\AgdaSpace{}%
\AgdaOperator{\AgdaInductiveConstructor{◃}}\AgdaSpace{}%
\AgdaBound{g₁}\AgdaSpace{}%
\AgdaOperator{\AgdaInductiveConstructor{◃}}\AgdaSpace{}%
\AgdaBound{h₁}\AgdaSymbol{))}\AgdaSpace{}%
\AgdaSymbol{(}\AgdaInductiveConstructor{ne}\AgdaSpace{}%
\AgdaSymbol{(}\AgdaBound{f₂}\AgdaSpace{}%
\AgdaOperator{\AgdaInductiveConstructor{◃}}\AgdaSpace{}%
\AgdaBound{g₂}\AgdaSpace{}%
\AgdaOperator{\AgdaInductiveConstructor{◃}}\AgdaSpace{}%
\AgdaBound{h₂}\AgdaSymbol{))}\AgdaSpace{}%
\AgdaSymbol{=}\AgdaSpace{}%
\AgdaInductiveConstructor{ne}\AgdaSpace{}%
\AgdaSymbol{(}\AgdaFunction{ff}\AgdaSpace{}%
\AgdaOperator{\AgdaInductiveConstructor{◃}}\AgdaSpace{}%
\AgdaFunction{gg}\AgdaSpace{}%
\AgdaOperator{\AgdaInductiveConstructor{◃}}\AgdaSpace{}%
\AgdaFunction{hh}\AgdaSymbol{)}\<%
\\
\>[0][@{}l@{\AgdaIndent{0}}]%
\>[2]\AgdaKeyword{where}\<%
\\
%
\>[2]\AgdaFunction{ff}\AgdaSpace{}%
\AgdaSymbol{:}\AgdaSpace{}%
\AgdaSymbol{\AgdaUnderscore{}}\<%
\\
%
\>[2]\AgdaFunction{ff}\AgdaSpace{}%
\AgdaSymbol{=}\AgdaSpace{}%
\AgdaOperator{\AgdaFunction{<}}\AgdaSpace{}%
\AgdaBound{f₁}\AgdaSpace{}%
\AgdaOperator{\AgdaFunction{,}}\AgdaSpace{}%
\AgdaBound{f₂}\AgdaSpace{}%
\AgdaOperator{\AgdaFunction{>}}\<%
\\
%
\\[\AgdaEmptyExtraSkip]%
%
\>[2]\AgdaFunction{gg}\AgdaSpace{}%
\AgdaSymbol{:}\AgdaSpace{}%
\AgdaSymbol{(}\AgdaBound{x}\AgdaSpace{}%
\AgdaSymbol{:}\AgdaSpace{}%
\AgdaDatatype{Var}\AgdaSpace{}%
\AgdaBound{Γ}\AgdaSpace{}%
\AgdaGeneralizable{A}\AgdaSymbol{)}\AgdaSpace{}%
\AgdaSymbol{(}\AgdaBound{s}\AgdaSpace{}%
\AgdaSymbol{:}\AgdaSpace{}%
\AgdaSymbol{\AgdaUnderscore{})}\AgdaSpace{}%
\AgdaSymbol{→}\AgdaSpace{}%
\AgdaSymbol{\AgdaUnderscore{}}\<%
\\
%
\>[2]\AgdaFunction{gg}\AgdaSpace{}%
\AgdaBound{x}\AgdaSpace{}%
\AgdaBound{s}\AgdaSpace{}%
\AgdaSymbol{(}\AgdaInductiveConstructor{inj₁}\AgdaSpace{}%
\AgdaBound{q₁}\AgdaSymbol{)}\AgdaSpace{}%
\AgdaSymbol{=}\AgdaSpace{}%
\AgdaBound{g₁}\AgdaSpace{}%
\AgdaBound{x}\AgdaSpace{}%
\AgdaBound{s}\AgdaSpace{}%
\AgdaBound{q₁}\<%
\\
%
\>[2]\AgdaFunction{gg}\AgdaSpace{}%
\AgdaBound{x}\AgdaSpace{}%
\AgdaBound{s}\AgdaSpace{}%
\AgdaSymbol{(}\AgdaInductiveConstructor{inj₂}\AgdaSpace{}%
\AgdaBound{q₂}\AgdaSymbol{)}\AgdaSpace{}%
\AgdaSymbol{=}\AgdaSpace{}%
\AgdaBound{g₂}\AgdaSpace{}%
\AgdaBound{x}\AgdaSpace{}%
\AgdaBound{s}\AgdaSpace{}%
\AgdaBound{q₂}\<%
\\
%
\\[\AgdaEmptyExtraSkip]%
%
\>[2]\AgdaFunction{hh}\AgdaSpace{}%
\AgdaSymbol{:}\AgdaSpace{}%
\AgdaSymbol{(}\AgdaBound{x}\AgdaSpace{}%
\AgdaSymbol{:}\AgdaSpace{}%
\AgdaDatatype{Var}\AgdaSpace{}%
\AgdaBound{Γ}\AgdaSpace{}%
\AgdaGeneralizable{A}\AgdaSymbol{)}\AgdaSpace{}%
\AgdaSymbol{(}\AgdaBound{s}\AgdaSpace{}%
\AgdaSymbol{:}\AgdaSpace{}%
\AgdaSymbol{\AgdaUnderscore{})}\AgdaSpace{}%
\AgdaSymbol{(}\AgdaBound{q}\AgdaSpace{}%
\AgdaSymbol{:}\AgdaSpace{}%
\AgdaSymbol{\AgdaUnderscore{})}\AgdaSpace{}%
\AgdaSymbol{→}\AgdaSpace{}%
\AgdaSymbol{\AgdaUnderscore{}}\<%
\\
%
\>[2]\AgdaFunction{hh}\AgdaSpace{}%
\AgdaBound{x}\AgdaSpace{}%
\AgdaBound{s}\AgdaSpace{}%
\AgdaSymbol{(}\AgdaInductiveConstructor{inj₁}\AgdaSpace{}%
\AgdaBound{q₁}\AgdaSymbol{)}\AgdaSpace{}%
\AgdaSymbol{=}\AgdaSpace{}%
\AgdaBound{h₁}\AgdaSpace{}%
\AgdaBound{x}\AgdaSpace{}%
\AgdaBound{s}\AgdaSpace{}%
\AgdaBound{q₁}\<%
\\
%
\>[2]\AgdaFunction{hh}\AgdaSpace{}%
\AgdaBound{x}\AgdaSpace{}%
\AgdaBound{s}\AgdaSpace{}%
\AgdaSymbol{(}\AgdaInductiveConstructor{inj₂}\AgdaSpace{}%
\AgdaBound{q₂}\AgdaSymbol{)}\AgdaSpace{}%
\AgdaSymbol{=}\AgdaSpace{}%
\AgdaBound{h₂}\AgdaSpace{}%
\AgdaBound{x}\AgdaSpace{}%
\AgdaBound{s}\AgdaSpace{}%
\AgdaBound{q₂}\<%
\\
%
\\[\AgdaEmptyExtraSkip]%
\>[0]\AgdaOperator{\AgdaFunction{[\AgdaUnderscore{},\AgdaUnderscore{}]nf}}\AgdaSpace{}%
\AgdaSymbol{:}\AgdaSpace{}%
\AgdaDatatype{NfHom}\AgdaSpace{}%
\AgdaGeneralizable{t}\AgdaSpace{}%
\AgdaGeneralizable{w}\AgdaSpace{}%
\AgdaSymbol{→}\AgdaSpace{}%
\AgdaDatatype{NfHom}\AgdaSpace{}%
\AgdaGeneralizable{u}\AgdaSpace{}%
\AgdaGeneralizable{w}\AgdaSpace{}%
\AgdaSymbol{→}\AgdaSpace{}%
\AgdaDatatype{NfHom}\AgdaSpace{}%
\AgdaSymbol{(}\AgdaGeneralizable{t}\AgdaSpace{}%
\AgdaOperator{\AgdaFunction{⊎nf}}\AgdaSpace{}%
\AgdaGeneralizable{u}\AgdaSymbol{)}\AgdaSpace{}%
\AgdaGeneralizable{w}\<%
\\
\>[0]\AgdaOperator{\AgdaFunction{[}}\AgdaSpace{}%
\AgdaInductiveConstructor{lam}\AgdaSpace{}%
\AgdaBound{tv}\AgdaSpace{}%
\AgdaOperator{\AgdaFunction{,}}\AgdaSpace{}%
\AgdaInductiveConstructor{lam}\AgdaSpace{}%
\AgdaBound{uv}\AgdaSpace{}%
\AgdaOperator{\AgdaFunction{]nf}}\AgdaSpace{}%
\AgdaSymbol{=}\AgdaSpace{}%
\AgdaInductiveConstructor{lam}\AgdaSpace{}%
\AgdaSymbol{(}\AgdaOperator{\AgdaFunction{[}}\AgdaSpace{}%
\AgdaBound{tv}\AgdaSpace{}%
\AgdaOperator{\AgdaFunction{,}}\AgdaSpace{}%
\AgdaBound{uv}\AgdaSpace{}%
\AgdaOperator{\AgdaFunction{]nf}}\AgdaSymbol{)}\<%
\\
\>[0]\AgdaOperator{\AgdaFunction{[\AgdaUnderscore{},\AgdaUnderscore{}]nf}}\AgdaSpace{}%
\AgdaSymbol{\{}\AgdaBound{Γ}\AgdaSymbol{\}}\AgdaSpace{}%
\AgdaSymbol{\{}\AgdaBound{B}\AgdaSymbol{\}}\AgdaSpace{}%
\AgdaSymbol{(}\AgdaInductiveConstructor{ne}\AgdaSpace{}%
\AgdaSymbol{(}\AgdaBound{f₁}\AgdaSpace{}%
\AgdaOperator{\AgdaInductiveConstructor{◃}}\AgdaSpace{}%
\AgdaBound{g₁}\AgdaSpace{}%
\AgdaOperator{\AgdaInductiveConstructor{◃}}\AgdaSpace{}%
\AgdaBound{h₁}\AgdaSymbol{))}\AgdaSpace{}%
\AgdaSymbol{(}\AgdaInductiveConstructor{ne}\AgdaSpace{}%
\AgdaSymbol{(}\AgdaBound{f₂}\AgdaSpace{}%
\AgdaOperator{\AgdaInductiveConstructor{◃}}\AgdaSpace{}%
\AgdaBound{g₂}\AgdaSpace{}%
\AgdaOperator{\AgdaInductiveConstructor{◃}}\AgdaSpace{}%
\AgdaBound{h₂}\AgdaSymbol{))}\AgdaSpace{}%
\AgdaSymbol{=}\AgdaSpace{}%
\AgdaInductiveConstructor{ne}\AgdaSpace{}%
\AgdaSymbol{(}\AgdaFunction{ff}\AgdaSpace{}%
\AgdaOperator{\AgdaInductiveConstructor{◃}}\AgdaSpace{}%
\AgdaFunction{gg}\AgdaSpace{}%
\AgdaOperator{\AgdaInductiveConstructor{◃}}\AgdaSpace{}%
\AgdaFunction{hh}\AgdaSymbol{)}\<%
\\
\>[0][@{}l@{\AgdaIndent{0}}]%
\>[2]\AgdaKeyword{where}\<%
\\
%
\>[2]\AgdaFunction{ff}\AgdaSpace{}%
\AgdaSymbol{:}\AgdaSpace{}%
\AgdaSymbol{\AgdaUnderscore{}}\<%
\\
%
\>[2]\AgdaFunction{ff}\AgdaSpace{}%
\AgdaSymbol{=}\AgdaSpace{}%
\AgdaOperator{\AgdaFunction{[}}\AgdaSpace{}%
\AgdaBound{f₁}\AgdaSpace{}%
\AgdaOperator{\AgdaFunction{,}}\AgdaSpace{}%
\AgdaBound{f₂}\AgdaSpace{}%
\AgdaOperator{\AgdaFunction{]}}\<%
\\
%
\\[\AgdaEmptyExtraSkip]%
%
\>[2]\AgdaFunction{gg}\AgdaSpace{}%
\AgdaSymbol{:}\AgdaSpace{}%
\AgdaSymbol{(}\AgdaBound{x}\AgdaSpace{}%
\AgdaSymbol{:}\AgdaSpace{}%
\AgdaDatatype{Var}\AgdaSpace{}%
\AgdaBound{Γ}\AgdaSpace{}%
\AgdaGeneralizable{A}\AgdaSymbol{)}\AgdaSpace{}%
\AgdaSymbol{(}\AgdaBound{s}\AgdaSpace{}%
\AgdaSymbol{:}\AgdaSpace{}%
\AgdaSymbol{\AgdaUnderscore{})}\AgdaSpace{}%
\AgdaSymbol{→}\AgdaSpace{}%
\AgdaSymbol{\AgdaUnderscore{}}\<%
\\
%
\>[2]\AgdaFunction{gg}\AgdaSpace{}%
\AgdaBound{x}\AgdaSpace{}%
\AgdaSymbol{(}\AgdaInductiveConstructor{inj₁}\AgdaSpace{}%
\AgdaBound{s₁}\AgdaSymbol{)}\AgdaSpace{}%
\AgdaBound{q₁}\AgdaSpace{}%
\AgdaSymbol{=}\AgdaSpace{}%
\AgdaBound{g₁}\AgdaSpace{}%
\AgdaBound{x}\AgdaSpace{}%
\AgdaBound{s₁}\AgdaSpace{}%
\AgdaBound{q₁}\<%
\\
%
\>[2]\AgdaFunction{gg}\AgdaSpace{}%
\AgdaBound{x}\AgdaSpace{}%
\AgdaSymbol{(}\AgdaInductiveConstructor{inj₂}\AgdaSpace{}%
\AgdaBound{s₂}\AgdaSymbol{)}\AgdaSpace{}%
\AgdaBound{q₂}\AgdaSpace{}%
\AgdaSymbol{=}\AgdaSpace{}%
\AgdaBound{g₂}\AgdaSpace{}%
\AgdaBound{x}\AgdaSpace{}%
\AgdaBound{s₂}\AgdaSpace{}%
\AgdaBound{q₂}\<%
\\
%
\\[\AgdaEmptyExtraSkip]%
%
\>[2]\AgdaFunction{hh}\AgdaSpace{}%
\AgdaSymbol{:}\AgdaSpace{}%
\AgdaSymbol{(}\AgdaBound{x}\AgdaSpace{}%
\AgdaSymbol{:}\AgdaSpace{}%
\AgdaDatatype{Var}\AgdaSpace{}%
\AgdaBound{Γ}\AgdaSpace{}%
\AgdaGeneralizable{A}\AgdaSymbol{)}\AgdaSpace{}%
\AgdaSymbol{(}\AgdaBound{s}\AgdaSpace{}%
\AgdaSymbol{:}\AgdaSpace{}%
\AgdaSymbol{\AgdaUnderscore{})}\AgdaSpace{}%
\AgdaSymbol{(}\AgdaBound{q}\AgdaSpace{}%
\AgdaSymbol{:}\AgdaSpace{}%
\AgdaSymbol{\AgdaUnderscore{})}\AgdaSpace{}%
\AgdaSymbol{→}\AgdaSpace{}%
\AgdaSymbol{\AgdaUnderscore{}}\<%
\\
%
\>[2]\AgdaFunction{hh}\AgdaSpace{}%
\AgdaBound{x}\AgdaSpace{}%
\AgdaSymbol{(}\AgdaInductiveConstructor{inj₁}\AgdaSpace{}%
\AgdaBound{s₁}\AgdaSymbol{)}\AgdaSpace{}%
\AgdaBound{q₁}\AgdaSpace{}%
\AgdaSymbol{=}\AgdaSpace{}%
\AgdaBound{h₁}\AgdaSpace{}%
\AgdaBound{x}\AgdaSpace{}%
\AgdaBound{s₁}\AgdaSpace{}%
\AgdaBound{q₁}\<%
\\
%
\>[2]\AgdaFunction{hh}\AgdaSpace{}%
\AgdaBound{x}\AgdaSpace{}%
\AgdaSymbol{(}\AgdaInductiveConstructor{inj₂}\AgdaSpace{}%
\AgdaBound{s₂}\AgdaSymbol{)}\AgdaSpace{}%
\AgdaBound{q₂}\AgdaSpace{}%
\AgdaSymbol{=}\AgdaSpace{}%
\AgdaBound{h₂}\AgdaSpace{}%
\AgdaBound{x}\AgdaSpace{}%
\AgdaBound{s₂}\AgdaSpace{}%
\AgdaBound{q₂}\<%
\end{code}

\subsubsection*{SCwF}\label{SCwF}

\begin{code}%
\>[0]\AgdaFunction{idNfs}\AgdaSpace{}%
\AgdaSymbol{:}\AgdaSpace{}%
\AgdaDatatype{Nfs}\AgdaSpace{}%
\AgdaGeneralizable{Γ}\AgdaSpace{}%
\AgdaGeneralizable{Γ}\<%
\\
\>[0]\AgdaFunction{idNfs}\AgdaSpace{}%
\AgdaSymbol{\{}\AgdaInductiveConstructor{∙}\AgdaSymbol{\}}\AgdaSpace{}%
\AgdaSymbol{=}\AgdaSpace{}%
\AgdaInductiveConstructor{ε}\<%
\\
\>[0]\AgdaFunction{idNfs}\AgdaSpace{}%
\AgdaSymbol{\{}\AgdaBound{Γ}\AgdaSpace{}%
\AgdaOperator{\AgdaInductiveConstructor{▹}}\AgdaSpace{}%
\AgdaBound{A}\AgdaSymbol{\}}\AgdaSpace{}%
\AgdaSymbol{=}\AgdaSpace{}%
\AgdaFunction{idNfs}\AgdaSpace{}%
\AgdaOperator{\AgdaFunction{↑}}\<%
\\
%
\\[\AgdaEmptyExtraSkip]%
\>[0]\AgdaOperator{\AgdaFunction{\AgdaUnderscore{}∘nfs\AgdaUnderscore{}}}\AgdaSpace{}%
\AgdaSymbol{:}\AgdaSpace{}%
\AgdaDatatype{Nfs}\AgdaSpace{}%
\AgdaGeneralizable{Δ}\AgdaSpace{}%
\AgdaGeneralizable{Γ}\AgdaSpace{}%
\AgdaSymbol{→}\AgdaSpace{}%
\AgdaDatatype{Nfs}\AgdaSpace{}%
\AgdaGeneralizable{Θ}\AgdaSpace{}%
\AgdaGeneralizable{Δ}\AgdaSpace{}%
\AgdaSymbol{→}\AgdaSpace{}%
\AgdaDatatype{Nfs}\AgdaSpace{}%
\AgdaGeneralizable{Θ}\AgdaSpace{}%
\AgdaGeneralizable{Γ}\<%
\\
\>[0]\AgdaInductiveConstructor{ε}\AgdaSpace{}%
\AgdaOperator{\AgdaFunction{∘nfs}}\AgdaSpace{}%
\AgdaBound{γ}\AgdaSpace{}%
\AgdaSymbol{=}\AgdaSpace{}%
\AgdaInductiveConstructor{ε}\<%
\\
\>[0]\AgdaSymbol{(}\AgdaBound{δ}\AgdaSpace{}%
\AgdaOperator{\AgdaInductiveConstructor{,}}\AgdaSpace{}%
\AgdaBound{t}\AgdaSymbol{)}\AgdaSpace{}%
\AgdaOperator{\AgdaFunction{∘nfs}}\AgdaSpace{}%
\AgdaBound{γ}\AgdaSpace{}%
\AgdaSymbol{=}\AgdaSpace{}%
\AgdaSymbol{(}\AgdaBound{δ}\AgdaSpace{}%
\AgdaOperator{\AgdaFunction{∘nfs}}\AgdaSpace{}%
\AgdaBound{γ}\AgdaSymbol{)}\AgdaSpace{}%
\AgdaOperator{\AgdaInductiveConstructor{,}}\AgdaSpace{}%
\AgdaSymbol{(}\AgdaBound{t}\AgdaSpace{}%
\AgdaOperator{\AgdaFunction{[}}\AgdaSpace{}%
\AgdaBound{γ}\AgdaSpace{}%
\AgdaOperator{\AgdaFunction{]nf}}\AgdaSymbol{)}\<%
\\
%
\\[\AgdaEmptyExtraSkip]%
\>[0]\AgdaFunction{π₁}\AgdaSpace{}%
\AgdaSymbol{:}\AgdaSpace{}%
\AgdaDatatype{Nfs}\AgdaSpace{}%
\AgdaGeneralizable{Δ}\AgdaSpace{}%
\AgdaSymbol{(}\AgdaGeneralizable{Γ}\AgdaSpace{}%
\AgdaOperator{\AgdaInductiveConstructor{▹}}\AgdaSpace{}%
\AgdaGeneralizable{A}\AgdaSymbol{)}\AgdaSpace{}%
\AgdaSymbol{→}\AgdaSpace{}%
\AgdaDatatype{Nfs}\AgdaSpace{}%
\AgdaGeneralizable{Δ}\AgdaSpace{}%
\AgdaGeneralizable{Γ}\<%
\\
\>[0]\AgdaFunction{π₁}\AgdaSpace{}%
\AgdaSymbol{(}\AgdaBound{γ}\AgdaSpace{}%
\AgdaOperator{\AgdaInductiveConstructor{,}}\AgdaSpace{}%
\AgdaBound{t}\AgdaSymbol{)}\AgdaSpace{}%
\AgdaSymbol{=}\AgdaSpace{}%
\AgdaBound{γ}\<%
\\
%
\\[\AgdaEmptyExtraSkip]%
\>[0]\AgdaFunction{π₂}\AgdaSpace{}%
\AgdaSymbol{:}\AgdaSpace{}%
\AgdaDatatype{Nfs}\AgdaSpace{}%
\AgdaGeneralizable{Δ}\AgdaSpace{}%
\AgdaSymbol{(}\AgdaGeneralizable{Γ}\AgdaSpace{}%
\AgdaOperator{\AgdaInductiveConstructor{▹}}\AgdaSpace{}%
\AgdaGeneralizable{A}\AgdaSymbol{)}\AgdaSpace{}%
\AgdaSymbol{→}\AgdaSpace{}%
\AgdaDatatype{Nf}\AgdaSpace{}%
\AgdaGeneralizable{Δ}\AgdaSpace{}%
\AgdaGeneralizable{A}\<%
\\
\>[0]\AgdaFunction{π₂}\AgdaSpace{}%
\AgdaSymbol{(}\AgdaBound{γ}\AgdaSpace{}%
\AgdaOperator{\AgdaInductiveConstructor{,}}\AgdaSpace{}%
\AgdaBound{t}\AgdaSymbol{)}\AgdaSpace{}%
\AgdaSymbol{=}\AgdaSpace{}%
\AgdaBound{t}\<%
\\
%
\\[\AgdaEmptyExtraSkip]%
\>[0]\AgdaFunction{wk}\AgdaSpace{}%
\AgdaSymbol{:}\AgdaSpace{}%
\AgdaDatatype{Nfs}\AgdaSpace{}%
\AgdaSymbol{(}\AgdaGeneralizable{Γ}\AgdaSpace{}%
\AgdaOperator{\AgdaInductiveConstructor{▹}}\AgdaSpace{}%
\AgdaGeneralizable{A}\AgdaSymbol{)}\AgdaSpace{}%
\AgdaGeneralizable{Γ}\<%
\\
\>[0]\AgdaFunction{wk}\AgdaSpace{}%
\AgdaSymbol{=}\AgdaSpace{}%
\AgdaFunction{π₁}\AgdaSpace{}%
\AgdaFunction{idNfs}\<%
\\
%
\\[\AgdaEmptyExtraSkip]%
\>[0]\AgdaFunction{nvz}\AgdaSpace{}%
\AgdaSymbol{:}\AgdaSpace{}%
\AgdaDatatype{Nf}\AgdaSpace{}%
\AgdaSymbol{(}\AgdaGeneralizable{Γ}\AgdaSpace{}%
\AgdaOperator{\AgdaInductiveConstructor{▹}}\AgdaSpace{}%
\AgdaGeneralizable{A}\AgdaSymbol{)}\AgdaSpace{}%
\AgdaGeneralizable{A}\<%
\\
\>[0]\AgdaFunction{nvz}\AgdaSpace{}%
\AgdaSymbol{=}\AgdaSpace{}%
\AgdaFunction{π₂}\AgdaSpace{}%
\AgdaFunction{idNfs}\<%
\\
%
\\[\AgdaEmptyExtraSkip]%
\>[0]\AgdaFunction{nvs}\AgdaSpace{}%
\AgdaSymbol{:}\AgdaSpace{}%
\AgdaDatatype{Nf}\AgdaSpace{}%
\AgdaGeneralizable{Γ}\AgdaSpace{}%
\AgdaGeneralizable{A}\AgdaSpace{}%
\AgdaSymbol{→}\AgdaSpace{}%
\AgdaDatatype{Nf}\AgdaSpace{}%
\AgdaSymbol{(}\AgdaGeneralizable{Γ}\AgdaSpace{}%
\AgdaOperator{\AgdaInductiveConstructor{▹}}\AgdaSpace{}%
\AgdaGeneralizable{B}\AgdaSymbol{)}\AgdaSpace{}%
\AgdaGeneralizable{A}\<%
\\
\>[0]\AgdaFunction{nvs}\AgdaSpace{}%
\AgdaBound{t}\AgdaSpace{}%
\AgdaSymbol{=}\AgdaSpace{}%
\AgdaBound{t}\AgdaSpace{}%
\AgdaOperator{\AgdaFunction{[}}\AgdaSpace{}%
\AgdaFunction{wk}\AgdaSpace{}%
\AgdaOperator{\AgdaFunction{]nf}}\<%
\\
%
\\[\AgdaEmptyExtraSkip]%
\>[0]\AgdaOperator{\AgdaFunction{<\AgdaUnderscore{}>}}\AgdaSpace{}%
\AgdaSymbol{:}\AgdaSpace{}%
\AgdaDatatype{Nf}\AgdaSpace{}%
\AgdaGeneralizable{Γ}\AgdaSpace{}%
\AgdaGeneralizable{A}\AgdaSpace{}%
\AgdaSymbol{→}\AgdaSpace{}%
\AgdaDatatype{Nfs}\AgdaSpace{}%
\AgdaGeneralizable{Γ}\AgdaSpace{}%
\AgdaSymbol{(}\AgdaGeneralizable{Γ}\AgdaSpace{}%
\AgdaOperator{\AgdaInductiveConstructor{▹}}\AgdaSpace{}%
\AgdaGeneralizable{A}\AgdaSymbol{)}\<%
\\
\>[0]\AgdaOperator{\AgdaFunction{<}}\AgdaSpace{}%
\AgdaBound{t}\AgdaSpace{}%
\AgdaOperator{\AgdaFunction{>}}\AgdaSpace{}%
\AgdaSymbol{=}\AgdaSpace{}%
\AgdaFunction{idNfs}\AgdaSpace{}%
\AgdaOperator{\AgdaInductiveConstructor{,}}\AgdaSpace{}%
\AgdaBound{t}\<%
\\
%
\\[\AgdaEmptyExtraSkip]%
\>[0]\AgdaFunction{π₁β}\AgdaSpace{}%
\AgdaSymbol{:}\AgdaSpace{}%
\AgdaFunction{π₁}\AgdaSpace{}%
\AgdaSymbol{(}\AgdaGeneralizable{γ}\AgdaSpace{}%
\AgdaOperator{\AgdaInductiveConstructor{,}}\AgdaSpace{}%
\AgdaGeneralizable{t}\AgdaSymbol{)}\AgdaSpace{}%
\AgdaOperator{\AgdaDatatype{≡}}\AgdaSpace{}%
\AgdaGeneralizable{γ}\<%
\\
\>[0]\AgdaFunction{π₁β}\AgdaSpace{}%
\AgdaSymbol{=}\AgdaSpace{}%
\AgdaInductiveConstructor{refl}\<%
\\
%
\\[\AgdaEmptyExtraSkip]%
\>[0]\AgdaFunction{π₂β}\AgdaSpace{}%
\AgdaSymbol{:}\AgdaSpace{}%
\AgdaFunction{π₂}\AgdaSpace{}%
\AgdaSymbol{(}\AgdaGeneralizable{γ}\AgdaSpace{}%
\AgdaOperator{\AgdaInductiveConstructor{,}}\AgdaSpace{}%
\AgdaGeneralizable{t}\AgdaSymbol{)}\AgdaSpace{}%
\AgdaOperator{\AgdaDatatype{≡}}\AgdaSpace{}%
\AgdaGeneralizable{t}\<%
\\
\>[0]\AgdaFunction{π₂β}\AgdaSpace{}%
\AgdaSymbol{=}\AgdaSpace{}%
\AgdaInductiveConstructor{refl}\<%
\\
%
\\[\AgdaEmptyExtraSkip]%
\>[0]\AgdaFunction{πη}\AgdaSpace{}%
\AgdaSymbol{:}\AgdaSpace{}%
\AgdaSymbol{(}\AgdaFunction{π₁}\AgdaSpace{}%
\AgdaGeneralizable{γ}\AgdaSpace{}%
\AgdaOperator{\AgdaInductiveConstructor{,}}\AgdaSpace{}%
\AgdaFunction{π₂}\AgdaSpace{}%
\AgdaGeneralizable{γ}\AgdaSymbol{)}\AgdaSpace{}%
\AgdaOperator{\AgdaDatatype{≡}}\AgdaSpace{}%
\AgdaGeneralizable{γ}\<%
\\
\>[0]\AgdaFunction{πη}\AgdaSpace{}%
\AgdaSymbol{\{}\AgdaArgument{γ}\AgdaSpace{}%
\AgdaSymbol{=}\AgdaSpace{}%
\AgdaBound{γ}\AgdaSpace{}%
\AgdaOperator{\AgdaInductiveConstructor{,}}\AgdaSpace{}%
\AgdaBound{t}\AgdaSymbol{\}}\AgdaSpace{}%
\AgdaSymbol{=}\AgdaSpace{}%
\AgdaInductiveConstructor{refl}\<%
\\
%
\\[\AgdaEmptyExtraSkip]%
\>[0]\AgdaFunction{,∘}\AgdaSpace{}%
\AgdaSymbol{:}\AgdaSpace{}%
\AgdaSymbol{(}\AgdaGeneralizable{γ}\AgdaSpace{}%
\AgdaOperator{\AgdaInductiveConstructor{,}}\AgdaSpace{}%
\AgdaGeneralizable{t}\AgdaSymbol{)}\AgdaSpace{}%
\AgdaOperator{\AgdaFunction{∘nfs}}\AgdaSpace{}%
\AgdaGeneralizable{δ}\AgdaSpace{}%
\AgdaOperator{\AgdaDatatype{≡}}\AgdaSpace{}%
\AgdaSymbol{(}\AgdaGeneralizable{γ}\AgdaSpace{}%
\AgdaOperator{\AgdaFunction{∘nfs}}\AgdaSpace{}%
\AgdaGeneralizable{δ}\AgdaSpace{}%
\AgdaOperator{\AgdaInductiveConstructor{,}}\AgdaSpace{}%
\AgdaGeneralizable{t}\AgdaSpace{}%
\AgdaOperator{\AgdaFunction{[}}\AgdaSpace{}%
\AgdaGeneralizable{δ}\AgdaSpace{}%
\AgdaOperator{\AgdaFunction{]nf}}\AgdaSymbol{)}\<%
\\
\>[0]\AgdaFunction{,∘}\AgdaSpace{}%
\AgdaSymbol{=}\AgdaSpace{}%
\AgdaInductiveConstructor{refl}\<%
\end{code}

\subsubsection*{Example - $Htm$}\label{Htm}

\begin{code}%
\>[0]\AgdaFunction{Htm}\AgdaSpace{}%
\AgdaSymbol{:}\AgdaSpace{}%
\AgdaDatatype{Tm}\AgdaSpace{}%
\AgdaInductiveConstructor{∙}\AgdaSpace{}%
\AgdaSymbol{((}\AgdaInductiveConstructor{*}\AgdaSpace{}%
\AgdaOperator{\AgdaInductiveConstructor{⇒}}\AgdaSpace{}%
\AgdaInductiveConstructor{*}\AgdaSymbol{)}\AgdaSpace{}%
\AgdaOperator{\AgdaInductiveConstructor{⇒}}\AgdaSpace{}%
\AgdaInductiveConstructor{*}\AgdaSpace{}%
\AgdaOperator{\AgdaInductiveConstructor{⇒}}\AgdaSpace{}%
\AgdaInductiveConstructor{*}\AgdaSymbol{)}\<%
\\
\>[0]\AgdaFunction{Htm}\AgdaSpace{}%
\AgdaSymbol{=}\AgdaSpace{}%
\AgdaInductiveConstructor{lam}\AgdaSpace{}%
\AgdaSymbol{(}\AgdaInductiveConstructor{lam}\AgdaSpace{}%
\AgdaSymbol{(}\AgdaInductiveConstructor{Πtm}\AgdaSpace{}%
\AgdaSymbol{(}\AgdaRecord{⊤}\AgdaSpace{}%
\AgdaOperator{\AgdaDatatype{⊎}}\AgdaSpace{}%
\AgdaRecord{⊤}\AgdaSymbol{)}\AgdaSpace{}%
\AgdaSymbol{(λ}\<%
\\
\>[0][@{}l@{\AgdaIndent{0}}]%
\>[2]\AgdaSymbol{\{}\AgdaSpace{}%
\AgdaSymbol{(}\AgdaInductiveConstructor{inj₁}\AgdaSpace{}%
\AgdaInductiveConstructor{tt}\AgdaSymbol{)}\AgdaSpace{}%
\AgdaSymbol{→}\AgdaSpace{}%
\AgdaInductiveConstructor{var}\AgdaSpace{}%
\AgdaInductiveConstructor{vz}\<%
\\
%
\>[2]\AgdaSymbol{;}\AgdaSpace{}%
\AgdaSymbol{(}\AgdaInductiveConstructor{inj₂}\AgdaSpace{}%
\AgdaInductiveConstructor{tt}\AgdaSymbol{)}\AgdaSpace{}%
\AgdaSymbol{→}\AgdaSpace{}%
\AgdaInductiveConstructor{var}\AgdaSpace{}%
\AgdaSymbol{(}\AgdaInductiveConstructor{vs}\AgdaSpace{}%
\AgdaInductiveConstructor{vz}\AgdaSymbol{)}\AgdaSpace{}%
\AgdaOperator{\AgdaInductiveConstructor{\$}}\AgdaSpace{}%
\AgdaSymbol{((}\AgdaInductiveConstructor{var}\AgdaSpace{}%
\AgdaSymbol{(}\AgdaInductiveConstructor{vs}\AgdaSpace{}%
\AgdaInductiveConstructor{vz}\AgdaSymbol{))}\AgdaSpace{}%
\AgdaOperator{\AgdaInductiveConstructor{\$}}\AgdaSpace{}%
\AgdaInductiveConstructor{var}\AgdaSpace{}%
\AgdaInductiveConstructor{vz}\AgdaSymbol{)}\<%
\\
%
\>[2]\AgdaSymbol{\})))}\<%
\end{code}

\subsubsection*{Example - $Hnf$}\label{Hnf}

\begin{code}%
\>[0]\AgdaFunction{Hnf}\AgdaSpace{}%
\AgdaSymbol{:}\AgdaSpace{}%
\AgdaDatatype{Nf}\AgdaSpace{}%
\AgdaInductiveConstructor{∙}\AgdaSpace{}%
\AgdaSymbol{((}\AgdaInductiveConstructor{*}\AgdaSpace{}%
\AgdaOperator{\AgdaInductiveConstructor{⇒}}\AgdaSpace{}%
\AgdaInductiveConstructor{*}\AgdaSymbol{)}\AgdaSpace{}%
\AgdaOperator{\AgdaInductiveConstructor{⇒}}\AgdaSpace{}%
\AgdaInductiveConstructor{*}\AgdaSpace{}%
\AgdaOperator{\AgdaInductiveConstructor{⇒}}\AgdaSpace{}%
\AgdaInductiveConstructor{*}\AgdaSymbol{)}\<%
\\
\>[0]\AgdaFunction{Hnf}\AgdaSpace{}%
\AgdaSymbol{=}\AgdaSpace{}%
\AgdaInductiveConstructor{lam}\AgdaSpace{}%
\AgdaSymbol{(}\AgdaInductiveConstructor{lam}\AgdaSpace{}%
\AgdaSymbol{(}\AgdaInductiveConstructor{ne}\AgdaSpace{}%
\AgdaSymbol{(}\AgdaFunction{S}\AgdaSpace{}%
\AgdaOperator{\AgdaInductiveConstructor{◃}}\AgdaSpace{}%
\AgdaFunction{P}\AgdaSpace{}%
\AgdaOperator{\AgdaInductiveConstructor{◃}}\AgdaSpace{}%
\AgdaFunction{R}\AgdaSymbol{)))}\<%
\\
\>[0][@{}l@{\AgdaIndent{0}}]%
\>[2]\AgdaKeyword{where}\<%
\\
%
\>[2]\AgdaFunction{Γ₀}\AgdaSpace{}%
\AgdaSymbol{:}\AgdaSpace{}%
\AgdaDatatype{Con}\<%
\\
%
\>[2]\AgdaFunction{Γ₀}\AgdaSpace{}%
\AgdaSymbol{=}\AgdaSpace{}%
\AgdaSymbol{(}\AgdaInductiveConstructor{∙}\AgdaSpace{}%
\AgdaOperator{\AgdaInductiveConstructor{▹}}\AgdaSpace{}%
\AgdaInductiveConstructor{*}\AgdaSpace{}%
\AgdaOperator{\AgdaInductiveConstructor{⇒}}\AgdaSpace{}%
\AgdaInductiveConstructor{*}\AgdaSpace{}%
\AgdaOperator{\AgdaInductiveConstructor{▹}}\AgdaSpace{}%
\AgdaInductiveConstructor{*}\AgdaSymbol{)}\<%
\\
%
\\[\AgdaEmptyExtraSkip]%
%
\>[2]\AgdaFunction{S}\AgdaSpace{}%
\AgdaSymbol{:}\AgdaSpace{}%
\AgdaFunction{Type}\<%
\\
%
\>[2]\AgdaFunction{S}\AgdaSpace{}%
\AgdaSymbol{=}\AgdaSpace{}%
\AgdaRecord{⊤}\AgdaSpace{}%
\AgdaOperator{\AgdaDatatype{⊎}}\AgdaSpace{}%
\AgdaRecord{⊤}\<%
\\
%
\\[\AgdaEmptyExtraSkip]%
%
\>[2]\AgdaFunction{P}\AgdaSpace{}%
\AgdaSymbol{:}\AgdaSpace{}%
\AgdaDatatype{Var}\AgdaSpace{}%
\AgdaFunction{Γ₀}\AgdaSpace{}%
\AgdaGeneralizable{A}\AgdaSpace{}%
\AgdaSymbol{→}\AgdaSpace{}%
\AgdaFunction{S}\AgdaSpace{}%
\AgdaSymbol{→}\AgdaSpace{}%
\AgdaFunction{Type}\<%
\\
%
\>[2]\AgdaFunction{P}\AgdaSpace{}%
\AgdaInductiveConstructor{vz}\AgdaSpace{}%
\AgdaSymbol{(}\AgdaInductiveConstructor{inj₁}\AgdaSpace{}%
\AgdaInductiveConstructor{tt}\AgdaSymbol{)}\AgdaSpace{}%
\AgdaSymbol{=}\AgdaSpace{}%
\AgdaRecord{⊤}\<%
\\
%
\>[2]\AgdaFunction{P}\AgdaSpace{}%
\AgdaInductiveConstructor{vz}\AgdaSpace{}%
\AgdaSymbol{(}\AgdaInductiveConstructor{inj₂}\AgdaSpace{}%
\AgdaInductiveConstructor{tt}\AgdaSymbol{)}\AgdaSpace{}%
\AgdaSymbol{=}\AgdaSpace{}%
\AgdaFunction{⊥}\<%
\\
%
\>[2]\AgdaFunction{P}\AgdaSpace{}%
\AgdaSymbol{(}\AgdaInductiveConstructor{vs}\AgdaSpace{}%
\AgdaInductiveConstructor{vz}\AgdaSymbol{)}\AgdaSpace{}%
\AgdaSymbol{(}\AgdaInductiveConstructor{inj₁}\AgdaSpace{}%
\AgdaInductiveConstructor{tt}\AgdaSymbol{)}\AgdaSpace{}%
\AgdaSymbol{=}\AgdaSpace{}%
\AgdaFunction{⊥}\<%
\\
%
\>[2]\AgdaFunction{P}\AgdaSpace{}%
\AgdaSymbol{(}\AgdaInductiveConstructor{vs}\AgdaSpace{}%
\AgdaInductiveConstructor{vz}\AgdaSymbol{)}\AgdaSpace{}%
\AgdaSymbol{(}\AgdaInductiveConstructor{inj₂}\AgdaSpace{}%
\AgdaInductiveConstructor{tt}\AgdaSymbol{)}\AgdaSpace{}%
\AgdaSymbol{=}\AgdaSpace{}%
\AgdaRecord{⊤}\<%
\\
%
\\[\AgdaEmptyExtraSkip]%
%
\>[2]\AgdaFunction{R}\AgdaSpace{}%
\AgdaSymbol{:}\AgdaSpace{}%
\AgdaSymbol{(}\AgdaBound{x}\AgdaSpace{}%
\AgdaSymbol{:}\AgdaSpace{}%
\AgdaDatatype{Var}\AgdaSpace{}%
\AgdaFunction{Γ₀}\AgdaSpace{}%
\AgdaGeneralizable{A}\AgdaSymbol{)}\AgdaSpace{}%
\AgdaSymbol{(}\AgdaBound{s}\AgdaSpace{}%
\AgdaSymbol{:}\AgdaSpace{}%
\AgdaFunction{S}\AgdaSymbol{)}\AgdaSpace{}%
\AgdaSymbol{→}\AgdaSpace{}%
\AgdaFunction{P}\AgdaSpace{}%
\AgdaBound{x}\AgdaSpace{}%
\AgdaBound{s}\AgdaSpace{}%
\AgdaSymbol{→}\AgdaSpace{}%
\AgdaDatatype{Sp}\AgdaSpace{}%
\AgdaFunction{Γ₀}\AgdaSpace{}%
\AgdaGeneralizable{A}\AgdaSpace{}%
\AgdaInductiveConstructor{*}\<%
\\
%
\>[2]\AgdaFunction{R}\AgdaSpace{}%
\AgdaInductiveConstructor{vz}\AgdaSpace{}%
\AgdaSymbol{(}\AgdaInductiveConstructor{inj₁}\AgdaSpace{}%
\AgdaInductiveConstructor{tt}\AgdaSymbol{)}\AgdaSpace{}%
\AgdaInductiveConstructor{tt}\AgdaSpace{}%
\AgdaSymbol{=}\AgdaSpace{}%
\AgdaInductiveConstructor{ε}\<%
\\
%
\>[2]\AgdaFunction{R}%
\>[1276I]\AgdaSymbol{(}\AgdaInductiveConstructor{vs}\AgdaSpace{}%
\AgdaInductiveConstructor{vz}\AgdaSymbol{)}\AgdaSpace{}%
\AgdaSymbol{(}\AgdaInductiveConstructor{inj₂}\AgdaSpace{}%
\AgdaInductiveConstructor{tt}\AgdaSymbol{)}\AgdaSpace{}%
\AgdaInductiveConstructor{tt}\AgdaSpace{}%
\AgdaSymbol{=}\AgdaSpace{}%
\AgdaFunction{FX}\AgdaSpace{}%
\AgdaOperator{\AgdaInductiveConstructor{,}}\AgdaSpace{}%
\AgdaInductiveConstructor{ε}\<%
\\
\>[.][@{}l@{}]\<[1276I]%
\>[4]\AgdaKeyword{where}\<%
\\
%
\>[4]\AgdaFunction{FX}\AgdaSpace{}%
\AgdaSymbol{:}\AgdaSpace{}%
\AgdaDatatype{Nf}\AgdaSpace{}%
\AgdaFunction{Γ₀}\AgdaSpace{}%
\AgdaInductiveConstructor{*}\<%
\\
%
\>[4]\AgdaFunction{FX}\AgdaSpace{}%
\AgdaSymbol{=}\AgdaSpace{}%
\AgdaInductiveConstructor{ne}\AgdaSpace{}%
\AgdaSymbol{(}\AgdaFunction{FX-S}\AgdaSpace{}%
\AgdaOperator{\AgdaInductiveConstructor{◃}}\AgdaSpace{}%
\AgdaFunction{FX-P}\AgdaSpace{}%
\AgdaOperator{\AgdaInductiveConstructor{◃}}\AgdaSpace{}%
\AgdaFunction{FX-R}\AgdaSymbol{)}\<%
\\
\>[4][@{}l@{\AgdaIndent{0}}]%
\>[6]\AgdaKeyword{where}\<%
\\
%
\>[6]\AgdaFunction{FX-S}\AgdaSpace{}%
\AgdaSymbol{:}\AgdaSpace{}%
\AgdaFunction{Type}\<%
\\
%
\>[6]\AgdaFunction{FX-S}\AgdaSpace{}%
\AgdaSymbol{=}\AgdaSpace{}%
\AgdaRecord{⊤}\<%
\\
%
\\[\AgdaEmptyExtraSkip]%
%
\>[6]\AgdaFunction{FX-P}\AgdaSpace{}%
\AgdaSymbol{:}\AgdaSpace{}%
\AgdaDatatype{Var}\AgdaSpace{}%
\AgdaFunction{Γ₀}\AgdaSpace{}%
\AgdaGeneralizable{A}\AgdaSpace{}%
\AgdaSymbol{→}\AgdaSpace{}%
\AgdaFunction{FX-S}\AgdaSpace{}%
\AgdaSymbol{→}\AgdaSpace{}%
\AgdaFunction{Type}\<%
\\
%
\>[6]\AgdaFunction{FX-P}\AgdaSpace{}%
\AgdaInductiveConstructor{vz}\AgdaSpace{}%
\AgdaInductiveConstructor{tt}\AgdaSpace{}%
\AgdaSymbol{=}\AgdaSpace{}%
\AgdaFunction{⊥}\<%
\\
%
\>[6]\AgdaFunction{FX-P}\AgdaSpace{}%
\AgdaSymbol{(}\AgdaInductiveConstructor{vs}\AgdaSpace{}%
\AgdaInductiveConstructor{vz}\AgdaSymbol{)}\AgdaSpace{}%
\AgdaInductiveConstructor{tt}\AgdaSpace{}%
\AgdaSymbol{=}\AgdaSpace{}%
\AgdaRecord{⊤}\<%
\\
%
\\[\AgdaEmptyExtraSkip]%
%
\>[6]\AgdaFunction{FX-R}\AgdaSpace{}%
\AgdaSymbol{:}\AgdaSpace{}%
\AgdaSymbol{(}\AgdaBound{x}\AgdaSpace{}%
\AgdaSymbol{:}\AgdaSpace{}%
\AgdaDatatype{Var}\AgdaSpace{}%
\AgdaFunction{Γ₀}\AgdaSpace{}%
\AgdaGeneralizable{A}\AgdaSymbol{)}\AgdaSpace{}%
\AgdaSymbol{(}\AgdaBound{s}\AgdaSpace{}%
\AgdaSymbol{:}\AgdaSpace{}%
\AgdaFunction{FX-S}\AgdaSymbol{)}\AgdaSpace{}%
\AgdaSymbol{→}\AgdaSpace{}%
\AgdaFunction{FX-P}\AgdaSpace{}%
\AgdaBound{x}\AgdaSpace{}%
\AgdaBound{s}\AgdaSpace{}%
\AgdaSymbol{→}\AgdaSpace{}%
\AgdaDatatype{Sp}\AgdaSpace{}%
\AgdaFunction{Γ₀}\AgdaSpace{}%
\AgdaGeneralizable{A}\AgdaSpace{}%
\AgdaInductiveConstructor{*}\<%
\\
%
\>[6]\AgdaFunction{FX-R}\AgdaSpace{}%
\AgdaSymbol{(}\AgdaInductiveConstructor{vs}\AgdaSpace{}%
\AgdaInductiveConstructor{vz}\AgdaSymbol{)}\AgdaSpace{}%
\AgdaInductiveConstructor{tt}\AgdaSpace{}%
\AgdaInductiveConstructor{tt}\AgdaSpace{}%
\AgdaSymbol{=}\AgdaSpace{}%
\AgdaFunction{X}\AgdaSpace{}%
\AgdaOperator{\AgdaInductiveConstructor{,}}\AgdaSpace{}%
\AgdaInductiveConstructor{ε}\<%
\\
\>[6][@{}l@{\AgdaIndent{0}}]%
\>[8]\AgdaKeyword{where}\<%
\\
%
\>[8]\AgdaFunction{X}\AgdaSpace{}%
\AgdaSymbol{:}\AgdaSpace{}%
\AgdaDatatype{Nf}\AgdaSpace{}%
\AgdaFunction{Γ₀}\AgdaSpace{}%
\AgdaInductiveConstructor{*}\<%
\\
%
\>[8]\AgdaFunction{X}%
\>[1347I]\AgdaSymbol{=}\AgdaSpace{}%
\AgdaInductiveConstructor{ne}\AgdaSpace{}%
\AgdaSymbol{(}\AgdaFunction{X-S}\AgdaSpace{}%
\AgdaOperator{\AgdaInductiveConstructor{◃}}\AgdaSpace{}%
\AgdaFunction{X-P}\AgdaSpace{}%
\AgdaOperator{\AgdaInductiveConstructor{◃}}\AgdaSpace{}%
\AgdaFunction{X-R}\AgdaSymbol{)}\<%
\\
\>[.][@{}l@{}]\<[1347I]%
\>[10]\AgdaKeyword{where}\<%
\\
%
\>[10]\AgdaFunction{X-S}\AgdaSpace{}%
\AgdaSymbol{:}\AgdaSpace{}%
\AgdaFunction{Type}\<%
\\
%
\>[10]\AgdaFunction{X-S}\AgdaSpace{}%
\AgdaSymbol{=}\AgdaSpace{}%
\AgdaRecord{⊤}\<%
\\
%
\\[\AgdaEmptyExtraSkip]%
%
\>[10]\AgdaFunction{X-P}\AgdaSpace{}%
\AgdaSymbol{:}\AgdaSpace{}%
\AgdaDatatype{Var}\AgdaSpace{}%
\AgdaFunction{Γ₀}\AgdaSpace{}%
\AgdaGeneralizable{A}\AgdaSpace{}%
\AgdaSymbol{→}\AgdaSpace{}%
\AgdaFunction{X-S}\AgdaSpace{}%
\AgdaSymbol{→}\AgdaSpace{}%
\AgdaFunction{Type}\<%
\\
%
\>[10]\AgdaFunction{X-P}\AgdaSpace{}%
\AgdaInductiveConstructor{vz}\AgdaSpace{}%
\AgdaInductiveConstructor{tt}\AgdaSpace{}%
\AgdaSymbol{=}\AgdaSpace{}%
\AgdaRecord{⊤}\<%
\\
%
\>[10]\AgdaFunction{X-P}\AgdaSpace{}%
\AgdaSymbol{(}\AgdaInductiveConstructor{vs}\AgdaSpace{}%
\AgdaInductiveConstructor{vz}\AgdaSymbol{)}\AgdaSpace{}%
\AgdaInductiveConstructor{tt}\AgdaSpace{}%
\AgdaSymbol{=}\AgdaSpace{}%
\AgdaFunction{⊥}\<%
\\
%
\\[\AgdaEmptyExtraSkip]%
%
\>[10]\AgdaFunction{X-R}\AgdaSpace{}%
\AgdaSymbol{:}\AgdaSpace{}%
\AgdaSymbol{(}\AgdaBound{x}\AgdaSpace{}%
\AgdaSymbol{:}\AgdaSpace{}%
\AgdaDatatype{Var}\AgdaSpace{}%
\AgdaFunction{Γ₀}\AgdaSpace{}%
\AgdaGeneralizable{A}\AgdaSymbol{)}\AgdaSpace{}%
\AgdaSymbol{(}\AgdaBound{s}\AgdaSpace{}%
\AgdaSymbol{:}\AgdaSpace{}%
\AgdaFunction{X-S}\AgdaSymbol{)}\AgdaSpace{}%
\AgdaSymbol{→}\AgdaSpace{}%
\AgdaFunction{X-P}\AgdaSpace{}%
\AgdaBound{x}\AgdaSpace{}%
\AgdaBound{s}\AgdaSpace{}%
\AgdaSymbol{→}\AgdaSpace{}%
\AgdaDatatype{Sp}\AgdaSpace{}%
\AgdaFunction{Γ₀}\AgdaSpace{}%
\AgdaGeneralizable{A}\AgdaSpace{}%
\AgdaInductiveConstructor{*}\<%
\\
%
\>[10]\AgdaFunction{X-R}\AgdaSpace{}%
\AgdaInductiveConstructor{vz}\AgdaSpace{}%
\AgdaInductiveConstructor{tt}\AgdaSpace{}%
\AgdaInductiveConstructor{tt}\AgdaSpace{}%
\AgdaSymbol{=}\AgdaSpace{}%
\AgdaInductiveConstructor{ε}\<%
\\
%
\>[10]\AgdaFunction{X-R}\AgdaSpace{}%
\AgdaSymbol{(}\AgdaInductiveConstructor{vs}\AgdaSpace{}%
\AgdaInductiveConstructor{vz}\AgdaSymbol{)}\AgdaSpace{}%
\AgdaInductiveConstructor{tt}\AgdaSpace{}%
\AgdaSymbol{()}\<%
\end{code}

% Bibliography
\addcontentsline{toc}{chapter}{References}
\bibliographystyle{acm}
\bibliography{references/bibliography}

\end{document}