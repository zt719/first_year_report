\begin{code}[hide]
{-# OPTIONS --cubical #-}
open import Cubical.Foundations.Prelude
\end{code}

\subsection*{Cubical TT}

Finally, cubical type theory!!!

\section{Category Theory}

\subsection*{Categories}

We define a category as follow:

% \begin{code}
% record Cat : Type₁ where
%   field
%     Obj : Type
%     Hom : Obj → Obj → Type
%     id  : ∀ {X} → Hom X X
%     _∘_ : ∀ {X Y Z} → Hom Y Z → Hom X Y → Hom X Z
%     idl : ∀ {X Y} (f : Hom X Y) → id ∘ f ≡ f
%     idr : ∀ {X Y} (f : Hom X Y) → f ∘ id ≡ f
%     ass : ∀ {X Y Z W} (f : Hom Z W) (g : Hom Y Z)  (h : Hom X Y)
%       → (f ∘ g) ∘ h ≡ f ∘ (g ∘ h)
%     isSetHom : ∀ {X Y} → isSet (Hom X Y)

%   ∣_∣ = Obj
%   _[_,_] = Hom
% \end{code}

\subsection*{Functors}

% \begin{code}
% record Func (ℂ 𝔻 : Cat) : Type₁ where
%   open Cat
%   field
%     F₀ : ∣ ℂ ∣ → ∣ 𝔻 ∣
%     F₁ : ∀ {X Y} → ℂ [ X , Y ] → 𝔻 [ F₀ X , F₀ Y ]
% \end{code}

\subsection*{Natural Transformations}

% \begin{code}
% record NatTrans {ℂ 𝔻} (F G : Func ℂ 𝔻) : Type where
%   open Cat hiding (_∘_)
%   open Cat 𝔻 using (_∘_)
%   open Func F
%   open Func G renaming (F₀ to G₀ ; F₁ to G₁)
%   field
%     η : (X : ∣ ℂ ∣) → 𝔻 [ F₀ X , G₀ X ]
%     com : ∀ {X Y} (f : ℂ [ X , Y ]) → (η Y ∘ F₁ f) ≡ (G₁ f ∘ η X)
% \end{code}

\subsection*{Algebra}

% \begin{code}
% record Alg {ℂ : Cat} (F : Func ℂ ℂ) : Type where
%   open Cat
%   open Func F
%   field
%     X : ∣ ℂ ∣
%     α : ℂ [ F₀ X , X ]
% \end{code}