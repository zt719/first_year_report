\chapter{Notations and Settings}

\begin{code}[hide]
{-# OPTIONS --cubical --guardedness #-}

open import Cubical.Foundations.Prelude
open import Cubical.Foundations.HLevels
open import Cubical.Categories.Category
open import Cubical.Categories.Functor
open import Cubical.Categories.Instances.Sets
\end{code}

In this section, we introduce syntax from \textit{agda-std} and \textit{cubical} Agda standard libraries used as notation throughout our formalizations. We assume the reader is familiar with Martin-Löf Type Theory (MLTT) and its variants, particularly Homotopy Type Theory (HoTT) and Cubical Type Theory (CTT). Additionally, we presuppose a basic understanding of category theory and do not formally introduce its foundational concepts.

\section{Notations}

\subsection{Type Theory}

\begin{itemize}
  \item{\AgdaBound{A} \AgdaSymbol{→} \AgdaBound{B} - function type}
  \item{\AgdaRecord{⊤} - unit type}
  \begin{itemize}
    \item{\AgdaInductiveConstructor{tt} \AgdaSymbol{:} \AgdaRecord{⊤}}
  \end{itemize}
  \item{\AgdaFunction{⊥} - empty type}
  \item{\AgdaBound{A} \AgdaRecord{×} \AgdaBound{B} - product type}
  \begin{itemize}
    \item{\AgdaInductiveConstructor{\_,\_} \AgdaSymbol{:} \AgdaBound{A} \AgdaSymbol{→} \AgdaBound{B} \AgdaSymbol{→} \AgdaBound{A} \AgdaRecord{×} \AgdaBound{B}}
    \item{\AgdaFunction
{proj₁} \AgdaSymbol{:} \AgdaBound{A} \AgdaRecord{×} \AgdaBound{B} \AgdaSymbol{→} \AgdaBound{A}}
    \item{\AgdaFunction
{proj₂} \AgdaSymbol{:} \AgdaBound{A} \AgdaRecord{×} \AgdaBound{B} \AgdaSymbol{→} \AgdaBound{B}}
  \end{itemize}
  \item{\AgdaBound{A} \AgdaDatatype{⊎} \AgdaBound{B} - coproduct type}
  \begin{itemize}
    \item{\AgdaInductiveConstructor{inj₁} \AgdaSymbol{:} \AgdaBound{A} \AgdaSymbol{→} \AgdaBound{A} \AgdaRecord{⊎} \AgdaBound{B}}
    \item{\AgdaInductiveConstructor{inj₂} \AgdaSymbol{:} \AgdaBound{B} \AgdaSymbol{→} \AgdaBound{A} \AgdaRecord{⊎} \AgdaBound{B}}
  \end{itemize} 
  \item{\AgdaFunction{Π} \AgdaBound{A B} or \AgdaSymbol{(}\AgdaBound{a} \AgdaSymbol{:} \AgdaBound{A}\AgdaSymbol{)} \AgdaSymbol{→} \AgdaBound{B a} - dependent function type}
  \item{\AgdaRecord{Σ} \AgdaBound{A B} or \AgdaFunction{Σ[} \AgdaBound{a} \AgdaFunction{∈} \AgdaBound{A} \AgdaFunction{]} \AgdaBound{B a} - dependent product type}
  \begin{itemize}
    \item{\AgdaInductiveConstructor{\_,\_} \AgdaSymbol{:} \AgdaSymbol{(}\AgdaBound{a} \AgdaSymbol{:} \AgdaBound{A}\AgdaSymbol{)} \AgdaSymbol{→} \AgdaBound{B a} \AgdaSymbol{→} \AgdaRecord{Σ} \AgdaBound{A B}}
    \item{\AgdaFunction
{proj₁} \AgdaSymbol{:} \AgdaRecord{Σ} \AgdaBound{A B} \AgdaSymbol{→} \AgdaBound{A}}
    \item{\AgdaFunction
{proj₂} \AgdaSymbol{:} \AgdaSymbol{(}\AgdaBound{ab} \AgdaSymbol{:} \AgdaRecord{Σ} \AgdaBound{A B}\AgdaSymbol{)} \AgdaSymbol{→} \AgdaBound{B} \AgdaSymbol{(}\AgdaFunction
{proj₁} \AgdaBound{ab}\AgdaSymbol{)}}
  \end{itemize}
\end{itemize}

\subsection{Category Theory}

\begin{itemize}
  \item{\AgdaBound{ℂ} \AgdaSymbol{:} \AgdaRecord{Category}} - Category
  \item{\AgdaFunction{∣} \AgdaBound{ℂ} \AgdaFunction{∣} \AgdaSymbol{:} \AgdaPrimitive{Type}} - Objects of category
  \item{\AgdaBound{ℂ} \AgdaFunction{[} \AgdaBound{X} \AgdaFunction{,} \AgdaBound{Y} \AgdaFunction{]} \AgdaSymbol{:} \AgdaPrimitive{Set}} - Morphisms of category
  \item{\AgdaBound{F} \AgdaSymbol{:} \AgdaBound{ℂ} \AgdaFunction{⇒} \AgdaBound{𝔻} or \AgdaRecord{Functor} \AgdaBound{ℂ} \AgdaBound{𝔻}} - Functor
  \item{\AgdaBound{F}\AgdaFunction{₀} \AgdaSymbol{:} \AgdaPrimitive{Type} \AgdaSymbol{→} \AgdaPrimitive{Type}} - Mapping objects part of functor
  \item{\AgdaBound{F}\AgdaFunction{₁} \AgdaSymbol{: (}\AgdaBound{X} \AgdaSymbol{→} \AgdaBound{Y}\AgdaSymbol{) →} \AgdaBound{F}\AgdaFunction{₀} \AgdaBound{X} \AgdaSymbol{→} \AgdaBound{F}\AgdaFunction{₀} \AgdaBound{Y}} - Mapping morphisms part of functor
  \item{\AgdaBound{α} \AgdaSymbol{:} \AgdaRecord{∫} \AgdaBound{F} \AgdaRecord{⇒} \AgdaBound{G} or \AgdaRecord{NatTrans} \AgdaBound{F} \AgdaBound{G}} - Natural transformation.
  \item{\AgdaFunction{[} \AgdaBound{f} \AgdaFunction{,} \AgdaBound{g} \AgdaFunction{]}} - The unique morphism to the product
  \[
  \begin{tikzcd}[row sep=huge, column sep=huge]
    & \AgdaBound{X} \arrow[dl, "\AgdaBound{f}"'] \arrow[dr, "\AgdaBound{g}"] \arrow[d, dashed, "\AgdaFunction{[}\AgdaBound{f}\AgdaFunction{,}\AgdaBound{g}\AgdaFunction{]}"] & \\
    \AgdaBound{A} & \AgdaBound{A × B} \arrow[l, "\AgdaFunction
{proj₁}"] \arrow[r, "\AgdaFunction
{proj₂}"'] & \AgdaBound{B}
  \end{tikzcd}
  \]
  \item{\AgdaFunction{<} \AgdaBound{f} \AgdaFunction{,} \AgdaBound{g} \AgdaFunction{>}} - The unique morphism from the coproduct
  \[
  \begin{tikzcd}[row sep=huge, column sep=huge]
    \AgdaBound{A} \arrow[r, "\AgdaFunction
{inj₁}"] \arrow[dr, "\AgdaBound{f}"'] & \AgdaBound{A + B} \arrow[d, dashed, "\AgdaFunction{<}\AgdaBound{f}\AgdaFunction{,}\AgdaBound{g}\AgdaFunction{>}"] & \AgdaBound{B} \arrow[l, "\AgdaFunction
{inj₂}"'] \arrow[dl, "\AgdaBound{g}"] \\
    & \AgdaBound{X} &
  \end{tikzcd}
  \]
\end{itemize}

\section{Formalization Settings}

We conduct theoretical research within HoTT, such as interpreting containers as endofunctors on h-level sets. This requires us to explicitly track h-level fields such as \AgdaFunction{isSet} in Cubical Agda, which can be quite bureaucratic and tedious in practice. Instead, to focus on mathematical ideas, we choose to do some post-rigorous math since we are already familiar with how to carry out such constructions rigorously. As such, our formalizations are carried out in plain Agda for intuitionistic purposes.

To be more specific, h-level definitions and checks are omitted. In this case, we are essentially working within a wild category setting. We also assume function extensionality and minimize the use of universe levels to improve both simplicity and readability.

For example, a normal container and container functor are defined as:

\begin{code}
record Cont : Type₁ where
  constructor _◃_&_&_
  field
    S : Type
    P : S → Type
    isSetS : isSet S
    isSetP : (s : S) → isSet (P s)

⟦_⟧ : Cont → Functor (SET ℓ-zero) (SET ℓ-zero)
⟦ S ◃ P & isSetS & isSetP ⟧
  = record
  { F-ob = λ (X , isSetX) → 
    (Σ[ s ∈ S ] (P s → X)) , isSetΣ isSetS (λ s → isSet→ isSetX)
  ; F-hom = λ f (s , k) → s , λ p → f (k p)
  ; F-id = λ i (s , k) → s , k
  ; F-seq = λ f g i (s , k) → s , λ p → g (f (k p))
  }
\end{code}

We would hide the field of \AgdaField{isSetS} and \AgdaField{isSetP} from \AgdaRecord{Cont}. Meanwhile we focus on the constructions and often leave \AgdaField{F-id} and \AgdaField{F-seq} implicit. \AgdaPrimitive{Set} means h-level set and \AgdaPrimitive{Type} means any type.